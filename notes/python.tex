\documentclass{amsart}
\usepackage{fullpage}
\usepackage{hyperref}
\usepackage{verbatim}
\title{Linear algebra in Python with the SymPy library}

\begin{document}

\maketitle

There are unfortunately a number of incompatibilities between
different versions of Python  in how they handle linear algebra.  The
notes below refer to the version which you can run in your browser by
visiting \url{http://live.sympy.org}.  If you install or update Python
and sympy on your own PC, then the instructions should work there
as well.  However, older versions may require different syntax.

You can enter the matrices 
\[ A = 
   \begin{pmatrix}
    5 & 6 & 7 \\
    4 & 3 & 2 
   \end{pmatrix}
   \hspace{5em}
   B = 
    \begin{pmatrix}
     9 & 5 \\ 5 & 9
    \end{pmatrix}
\]
like this:
\begin{verbatim}
A = Matrix([[5,6,7],[4,3,2]])
B = Matrix([[9,5],[5,9]])
\end{verbatim}

You can now do some calculations:
\begin{itemize}
 \item Enter \verb|B*A| to calculate $BA$.
 \item If you enter \verb|A*B| then you will get an error message,
  because $AB$ is not defined.  The error message is long and
  complicated, but the last line says
  \verb|ShapeError: Matrices size mismatch|, which should not be too
  hard to understand.
 \item Enter \verb|B**2| to calculate $B^2$, or \verb|B**-1| to
  calculate the inverse matrix $B^{-1}$.
 \item Enter \verb|transpose(A)| or \verb|A.T| to find the transpose
  $A^T$.
 \item Enter \verb|det(B)| or \verb|B.det()| to find the determinant
  of $B$.  The brackets are necessary here.  If you enter \verb|B.det|
  without the brackets, you will get 
\begin{verbatim}
 boundmethodMutableDenseMatrix.detofMatrix([[9, 5],[5, 9]])
\end{verbatim}
  which is not very helpful.  In general, if you get a result starting
  with \verb|boundmethod...|, it probably indicates that you left out
  a pair of brackets.  
 \item Enter \verb|eye(3)| for the $3\times 3$ identity matrix.
 \item Enter \verb|A.row(0)| to get the first row of $A$, or
  \verb|A.row(1)| to get the second row.  Python's indexing always
  starts with 0, so many things need to be shifted by $1$ compared to
  how they are in the notes or in Maple.
 \item Similarly, you can enter \verb|A.col(0)| or \verb|A.col(1)| or
  \verb|A.col(2)| to extract one of the columns of $A$.
 \item To get the top right entry in $A$ (which is called $A_{13}$ in
  the notes), enter \verb|A[0,2]|.
 \item In many places, it is convenient to give names to the columns
  of a matrix.  For example, we might take $u_1$, $u_2$ and $u_3$ to
  be the columns of the matrix $A$ above.  To do this in Python you can
  type 
\begin{verbatim}
u = [A.col(i) for i in range(3)]
\end{verbatim}
  However, this gives the usual shift in indexing, so the columns are
  \verb|u[0]|, \verb|u[1]| and \verb|u[2]|.
 \item To find the reduced row echelon form of $A$, enter
  \verb|A.rref()[0]|.  If you just enter \verb|A.rref()| instead you
  will get a list of two things, the first of which is the RREF of
  $A$, and the second of which is the list of pivot columns.  If you
  just enter \verb|A.rref| then you will get a result like
  \verb|boundmethod...| which you cannot use.  You might like to
  define a more convenient syntax by entering
\begin{verbatim}
def RREF(M):
    return M.rref()[0]
\end{verbatim}
  You can then enter \verb|RREF(A)| rather than \verb|A.rref()[0]|.
 \item If you are going to use the RREF of $A$ in further
  calculations, then you probably want to give it a name, like this:
\begin{verbatim}
A1 = A.rref()[0]
\end{verbatim}
  Note, however, that when you do this Python will not immediately
  print the result; you need to enter \verb|A1| on a separate line if
  you want to see the RREF matrix.
 \item To find and factor the characteristic polynomial of $B$, enter 
\begin{verbatim}
p = B.charpoly(t)
p
factor(p)
\end{verbatim}
  Python will print the characteristic polynomial as 
  \[ \text{PurePoly}(t^2-4t+3,t,domain=\mathbb{Z}) \]
  rather than just $t^2-4t+3$, but that should not cause a problem.
  It is best to stick with $t$ as the name of the variable here.  In
  particular, you should not try to use the symbol $\lambda$ for this
  or anything else, because it would clash with a completely different
  use of that symbol in Python; we will not take the space to explain
  the background here. 
 \item To find the eigenvalues and eigenvectors of $B$, enter 
  \verb|B.eigenvects()|.  The result is 
  \[
  \left(4,1,\left[\left[\begin{array}{c}-1\\1\end{array}\right]\right]\right),\quad
  \left(14,1,\left[\left[\begin{array}{c}1\\1\end{array}\right]\right]\right).
  \]
  The numbers $4$ and $14$ are the eigenvalues, and the vectors
  $\begin{pmatrix}-1\\1\end{pmatrix}$ and 
  $\begin{pmatrix}1\\1\end{pmatrix}$ are the corresponding
  eigenvectors, and the ones in between the eigenvalues and the
  eigenvectors can be ignored.  In cases where the matrix has repeated
  eigenvalues the result will need a bit more interpretation, but that
  will be discussed later.
\end{itemize}

\section*{Python with SymPy on your own machine}

If have installed Python (from \url{http://www.python.org/download/})
on your own PC then you can add SymPy by following the instructions at
\url{http://sympy.org}.  I recommend that you install version 2.7.5 of
Python and version 1.4 of SymPy.  After that you can start Python
under IDLE as usual, and enter  
\begin{verbatim}
from __future__ import division
from sympy import *
t = Symbol('t')
\end{verbatim}
The first line effectively tells Python that you want to use exact
fractions rather than decimal approximations everywhere.  The second
line says that you want to use SymPy.  The third line tells SymPy to
treat $t$ as an abstract symbol, which we need when working with
characteristic polynomials.  All of these things are done
automatically if you use \url{http://live.sympy.org}, but on your own
PC you need to do them yourself.

\end{document}
