\documentclass[a4paper]{article}
\usepackage[pdftex]{graphicx}
\usepackage{color}
\usepackage{amsmath,amssymb,amsthm}
\usepackage{tikz}
\usetikzlibrary{arrows}
\usepackage{verbatim}

\setlength{\topmargin}{-1cm}
\setlength{\oddsidemargin}{-1cm}
\setlength{\evensidemargin}{-1cm}
\setlength{\textwidth}{18cm}
\setlength{\textheight}{24.5cm}

\definecolor{olivegreen}{cmyk}{0.64,0,0.95,0.40}
\definecolor{rawsienna}{cmyk}{0,0.72,1,0.45}
\definecolor{lightgreen}{rgb}{0.85,1.0,0.85}

\newcommand{\GREENYELLOW}[1]{{\color{greenyellow}#1}}
\newcommand{\YELLOW}[1]{{\color{yellow}#1}}
\newcommand{\YLW}[1]{{\color{yellow}#1}}
\newcommand{\GOLDENROD}[1]{{\color{goldenrod}#1}}
\newcommand{\DANDELION}[1]{{\color{dandelion}#1}}
\newcommand{\APRICOT}[1]{{\color{apricot}#1}}
\newcommand{\PEACH}[1]{{\color{peach}#1}}
\newcommand{\MELON}[1]{{\color{melon}#1}}
\newcommand{\YELLOWORANGE}[1]{{\color{yelloworange}#1}}
\newcommand{\ORANGE}[1]{{\color{orange}#1}}
\newcommand{\BURNTORANGE}[1]{{\color{burntorange}#1}}
\newcommand{\BITTERSWEET}[1]{{\color{bittersweet}#1}}
\newcommand{\REDORANGE}[1]{{\color{redorange}#1}}
\newcommand{\MAHOGANY}[1]{{\color{mahogany}#1}}
\newcommand{\MAROON}[1]{{\color{maroon}#1}}
\newcommand{\BRICKRED}[1]{{\color{brickred}#1}}
\newcommand{\RED}[1]{{\color{red}#1}}
\newcommand{\ORANGERED}[1]{{\color{orangered}#1}}
\newcommand{\RUBINERED}[1]{{\color{rubinered}#1}}
\newcommand{\WILDSTRAWBERRY}[1]{{\color{wildstrawberry}#1}}
\newcommand{\SALMON}[1]{{\color{salmon}#1}}
\newcommand{\CARNATIONPINK}[1]{{\color{carnationpink}#1}}
\newcommand{\MAGENTA}[1]{{\color{magenta}#1}}
\newcommand{\VIOLETRED}[1]{{\color{violetred}#1}}
\newcommand{\RHODAMINE}[1]{{\color{rhodamine}#1}}
\newcommand{\MULBERRY}[1]{{\color{mulberry}#1}}
\newcommand{\REDVIOLET}[1]{{\color{redviolet}#1}}
\newcommand{\FUCHSIA}[1]{{\color{fuchsia}#1}}
\newcommand{\LAVENDER}[1]{{\color{lavender}#1}}
\newcommand{\THISTLE}[1]{{\color{thistle}#1}}
\newcommand{\ORCHID}[1]{{\color{orchid}#1}}
\newcommand{\DARKORCHID}[1]{{\color{darkorchid}#1}}
\newcommand{\PURPLE}[1]{{\color{purple}#1}}
\newcommand{\PLUM}[1]{{\color{plum}#1}}
\newcommand{\VIOLET}[1]{{\color{violet}#1}}
\newcommand{\ROYALPURPLE}[1]{{\color{royalpurple}#1}}
\newcommand{\BLUEVIOLET}[1]{{\color{blueviolet}#1}}
\newcommand{\PERIWINKLE}[1]{{\color{periwinkle}#1}}
\newcommand{\CADETBLUE}[1]{{\color{cadetblue}#1}}
\newcommand{\CORNFLOWERBLUE}[1]{{\color{cornflowerblue}#1}}
\newcommand{\MIDNIGHTBLUE}[1]{{\color{midnightblue}#1}}
\newcommand{\NAVYBLUE}[1]{{\color{navyblue}#1}}
\newcommand{\ROYALBLUE}[1]{{\color{royalblue}#1}}
\newcommand{\BLU}[1]{{\color{blue}#1}}
\newcommand{\BLUE}[1]{{\color{blue}#1}}
\newcommand{\CERULEAN}[1]{{\color{cerulean}#1}}
\newcommand{\CYAN}[1]{{\color{cyan}#1}}
\newcommand{\PROCESSBLUE}[1]{{\color{processblue}#1}}
\newcommand{\SKYBLUE}[1]{{\color{skyblue}#1}}
\newcommand{\TURQUOISE}[1]{{\color{turquoise}#1}}
\newcommand{\TEALBLUE}[1]{{\color{tealblue}#1}}
\newcommand{\AQUAMARINE}[1]{{\color{aquamarine}#1}}
\newcommand{\BLUEGREEN}[1]{{\color{bluegreen}#1}}
\newcommand{\EMERALD}[1]{{\color{emerald}#1}}
\newcommand{\JUNGLEGREEN}[1]{{\color{junglegreen}#1}}
\newcommand{\SEAGREEN}[1]{{\color{seagreen}#1}}
\newcommand{\GREEN}[1]{{\color{green}#1}}
\newcommand{\FORESTGREEN}[1]{{\color{forestgreen}#1}}
\newcommand{\PINEGREEN}[1]{{\color{pinegreen}#1}}
\newcommand{\LIMEGREEN}[1]{{\color{limegreen}#1}}
\newcommand{\YELLOWGREEN}[1]{{\color{yellowgreen}#1}}
\newcommand{\SPRINGGREEN}[1]{{\color{springgreen}#1}}
\newcommand{\OLIVEGREEN}[1]{{\color{olivegreen}#1}}
\newcommand{\OLG}[1]{{\color{olivegreen}#1}}
\newcommand{\RAWSIENNA}[1]{{\color{rawsienna}#1}}
\newcommand{\SEPIA}[1]{{\color{sepia}#1}}
\newcommand{\BROWN}[1]{{\color{brown}#1}}
\newcommand{\TAN}[1]{{\color{tan}#1}}
\newcommand{\GRAY}[1]{{\color{gray}#1}}
\newcommand{\LGRAY}[1]{{\color{gray!40}#1}}
\newcommand{\WHITE}[1]{{\color{white}#1}}
\newcommand{\BLACK}[1]{{\color{black}#1}}

\newcommand{\bbm}       {\left[\begin{matrix}}
\newcommand{\ebm}       {\end{matrix}\right]}
\newcommand{\bsm}       {\left[\begin{smallmatrix}}
\newcommand{\esm}       {\end{smallmatrix}\right]}
\newcommand{\bpm}       {\begin{pmatrix}}
\newcommand{\epm}       {\end{pmatrix}}
\newcommand{\bcf}[2]{\left(\begin{array}{c}{#1}\\{#2}\end{array}\right)}

\newcommand{\adj}       {\operatorname{adj}}
\newcommand{\ann}       {\operatorname{ann}}
\newcommand{\diag}      {\operatorname{diag}}
\newcommand{\img}       {\operatorname{img}}
\newcommand{\rnk}       {\operatorname{rank}}
\newcommand{\sgn}       {\operatorname{sgn}}
\newcommand{\spn}       {\operatorname{span}}
\newcommand{\trc}       {\operatorname{trace}}

\newcommand{\pp}{\hphantom{+}}
\newcommand{\tm}{\times}
\newcommand{\sse}{\subseteq}
\newcommand{\st}{\;|\;}
\newcommand{\sm}{\setminus}
\newcommand{\iffa}      {\Leftrightarrow}
\newcommand{\xra}{\xrightarrow}
\newcommand{\xla}{\xleftarrow}
\renewcommand{\:}{\colon}

\newcommand{\half}{\tfrac{1}{2}}

\newcommand{\N}         {{\mathbb{N}}}
\newcommand{\Z}         {{\mathbb{Z}}}
\newcommand{\Q}         {{\mathbb{Q}}}
\newcommand{\R}         {{\mathbb{R}}}
\newcommand{\C}         {{\mathbb{C}}}

\newcommand{\va}        {\mathbf{a}}
\newcommand{\vb}        {\mathbf{b}}
\newcommand{\vc}        {\mathbf{c}}
\newcommand{\vd}        {\mathbf{d}}
\newcommand{\ve}        {\mathbf{e}}
\newcommand{\vf}        {\mathbf{f}}
\newcommand{\vg}        {\mathbf{g}}
\newcommand{\vh}        {\mathbf{h}}
\newcommand{\vi}        {\mathbf{i}}
\newcommand{\vj}        {\mathbf{j}}
\newcommand{\vk}        {\mathbf{k}}
\newcommand{\vl}        {\mathbf{l}}
\newcommand{\vm}        {\mathbf{m}}
\newcommand{\vn}        {\mathbf{n}}
\newcommand{\vo}        {\mathbf{o}}
\newcommand{\vp}        {\mathbf{p}}
\newcommand{\vq}        {\mathbf{q}}
\newcommand{\vr}        {\mathbf{r}}
\newcommand{\vs}        {\mathbf{s}}
\newcommand{\vt}        {\mathbf{t}}
\newcommand{\vu}        {\mathbf{u}}
\newcommand{\vv}        {\mathbf{v}}
\newcommand{\vw}        {\mathbf{w}}
\newcommand{\vx}        {\mathbf{x}}
\newcommand{\vy}        {\mathbf{y}}
\newcommand{\vz}        {\mathbf{z}}

\newcommand{\vA}        {\mathbf{A}}
\newcommand{\vB}        {\mathbf{B}}
\newcommand{\vC}        {\mathbf{C}}
\newcommand{\vD}        {\mathbf{D}}
\newcommand{\vE}        {\mathbf{E}}
\newcommand{\vF}        {\mathbf{F}}
\newcommand{\vG}        {\mathbf{G}}
\newcommand{\vH}        {\mathbf{H}}
\newcommand{\vI}        {\mathbf{I}}
\newcommand{\vJ}        {\mathbf{J}}
\newcommand{\vK}        {\mathbf{K}}
\newcommand{\vL}        {\mathbf{L}}
\newcommand{\vM}        {\mathbf{M}}
\newcommand{\vN}        {\mathbf{N}}
\newcommand{\vO}        {\mathbf{O}}
\newcommand{\vP}        {\mathbf{P}}
\newcommand{\vQ}        {\mathbf{Q}}
\newcommand{\vR}        {\mathbf{R}}
\newcommand{\vS}        {\mathbf{S}}
\newcommand{\vT}        {\mathbf{T}}
\newcommand{\vU}        {\mathbf{U}}
\newcommand{\vV}        {\mathbf{V}}
\newcommand{\vW}        {\mathbf{W}}
\newcommand{\vX}        {\mathbf{X}}
\newcommand{\vY}        {\mathbf{Y}}
\newcommand{\vZ}        {\mathbf{Z}}

\newcommand{\al}        {\alpha}
\newcommand{\bt}        {\beta} 
\newcommand{\gm}        {\gamma}
\newcommand{\dl}        {\delta}
\newcommand{\ep}        {\epsilon}
\newcommand{\zt}        {\zeta}
\newcommand{\et}        {\eta}
\newcommand{\tht}       {\theta}
\newcommand{\io}        {\iota}
\newcommand{\kp}        {\kappa}
\newcommand{\lm}        {\lambda}
\newcommand{\ph}        {\phi}
\newcommand{\ch}        {\chi}
\newcommand{\ps}        {\psi}
\newcommand{\rh}        {\rho}
\newcommand{\sg}        {\sigma}
\newcommand{\om}        {\omega}

\newcommand{\Gm}        {\Gamma}
\newcommand{\Dl}        {\Delta}

\newcommand{\CA}        {\mathcal{A}}
\newcommand{\CB}        {\mathcal{B}}
\newcommand{\CC}        {\mathcal{C}}
\newcommand{\CD}        {\mathcal{D}}
\newcommand{\CE}        {\mathcal{E}}
\newcommand{\CF}        {\mathcal{F}}
\newcommand{\CG}        {\mathcal{G}}
\newcommand{\CH}        {\mathcal{H}}
\newcommand{\CI}        {\mathcal{I}}
\newcommand{\CJ}        {\mathcal{J}}
\newcommand{\CK}        {\mathcal{K}}
\newcommand{\CL}        {\mathcal{L}}
\newcommand{\CM}        {\mathcal{M}}
\newcommand{\CN}        {\mathcal{N}}
\newcommand{\CO}        {\mathcal{O}}
\newcommand{\CP}        {\mathcal{P}}
\newcommand{\CQ}        {\mathcal{Q}}
\newcommand{\CR}        {\mathcal{R}}
\newcommand{\CS}        {\mathcal{S}}
\newcommand{\CT}        {\mathcal{T}}
\newcommand{\CU}        {\mathcal{U}}
\newcommand{\CV}        {\mathcal{V}}
\newcommand{\CW}        {\mathcal{W}}
\newcommand{\CX}        {\mathcal{X}}
\newcommand{\CY}        {\mathcal{Y}}
\newcommand{\CZ}        {\mathcal{Z}}


\newcommand{\ov}        {\overline}
\newcommand{\ip}[1]     {\langle #1\rangle}
\renewcommand{\ss}      {\scriptstyle}

\newcommand{\EMPH}[1]{\emph{\RED{#1}}}
\newcommand{\DEFN}[1]{\emph{\PURPLE{#1}}}
\newcommand{\VEC}[1]    {\mathbf{#1}}

\newcommand{\ghost}{{\tiny $\color[rgb]{1,1,1}.$}}

\newcommand{\reminderbar}{\par\medskip\par\hrule\par\medskip\par}

\newcommand{\uc}{\uncover}

\newcommand{\bbox}[1]{
\[ \mbox{\begin{tikzpicture}%
   \draw(0,0) node[draw,thick,olivegreen,rectangle] {\color{black} #1};%
  \end{tikzpicture}} \]
}

\newcommand{\cbox}[1]{
\begin{center}\begin{tikzpicture}%
   \draw(0,0) node[draw,thick,olivegreen,rectangle] {\color{black} #1};%
\end{tikzpicture}\end{center}
}
%\newcounter{probcounter}
\newcounter{marksassigned}
\newcounter{marksawarded}
\newcounter{totalmarks}

\newcommand{\mrks}[1]{%
\addtocounter{marksassigned}{#1}%
\addtocounter{totalmarks}{#1}%
\textbf{(#1 marks)}}
\newcommand{\mrk}{%
\addtocounter{marksassigned}{1}%
\addtocounter{totalmarks}{1}%
\textbf{(1 mark)}}
\newcommand{\mks}[1]{%
\addtocounter{marksawarded}{#1}%
\textbf{\color{red}[#1]}}
\newcommand{\mk}{\mks{1}}

\renewcommand{\thesection}{}

\newenvironment{problem}[1][]{
\stepcounter{probcounter}
\setcounter{marksawarded}{0}
\bigskip\par\noindent\textbf{(\arabic{section}.\arabic{probcounter}; #1)}
}{
\typeout{Q\arabic{probcounter}: \arabic{marksassigned} marks assigned}
\setcounter{marksassigned}{0}
}

\newenvironment{solution}{
{\bigskip\par\noindent \bf Solution:}}{
\typeout{Q\arabic{probcounter}: \arabic{marksawarded} marks awarded}
}

\newcommand{\printtotalmarks}{%
\typeout{Total assigned: \arabic{totalmarks} marks}%
}

\newcommand{\sct}{\section}


\newcounter{probcounter}
\newcounter{marksassigned}
\newcounter{marksawarded}
\newcounter{totalmarks}

\newcommand{\mrks}[1]{%
\addtocounter{marksassigned}{#1}%
\addtocounter{totalmarks}{#1}%
\textbf{(#1 marks)}}
\newcommand{\mrk}{%
\addtocounter{marksassigned}{1}%
\addtocounter{totalmarks}{1}%
\textbf{(1 mark)}}
\newcommand{\mks}[1]{
\addtocounter{marksawarded}{#1}}
\newcommand{\mk}{\mks{1}}

\renewcommand{\thesection}{}

\newenvironment{problem}[1][]{
\stepcounter{probcounter}
\setcounter{marksawarded}{0}
\bigskip\par\noindent\textbf{(\arabic{section}.\arabic{probcounter}; #1)}
}{
\typeout{Q\arabic{probcounter}: \arabic{marksassigned} marks assigned}
\setcounter{marksassigned}{0}
}

\newenvironment{solution}{\comment}{\endcomment}

\newcommand{\printtotalmarks}{%
\typeout{Total assigned: \arabic{totalmarks} marks}%
}

\newcommand{\sct}{\section}



\begin{document}

\begin{center}
 {\Huge Linear Mathematics for Applications \\ All exam questions}
\end{center}
\vspace{4ex}

\section{Question 1}
\setcounter{probcounter}{0}

\begin{problem}[2013-14]
 You may assume the following row reductions.  Some of them are
 relevant, and some of them are not.  Some questions can be done more 
 easily without row-reduction.
 \[ \bbm
     0&2&0&-1&10\\
     0&-1&1&1&5\\
     -1&-3&2&2&-7\\
     -1&3&3&0&36\\
     \ebm
      \to \bbm
     1&0&0&0&9\\
     0&1&0&0&8\\
     0&0&1&0&7\\
     0&0&0&1&6\\
     \ebm
    \hspace{4em}
     \bbm
     0&0&-1&-1\\
     2&-1&-3&3\\
     0&1&2&3\\
     -1&1&2&0\\
     10&5&-7&36\\
     \ebm
      \to \bbm
     1&0&0&0\\
     0&1&0&0\\
     0&0&1&0\\
     0&0&0&1\\
     0&0&0&0\\
     \ebm
  \]
  \[ \bbm
     0&0&1&0\\
     1&0&0&1\\
     0&1&0&0\\
     0&0&1&0\\
     \ebm
      \to \bbm
     1&0&0&1\\
     0&1&0&0\\
     0&0&1&0\\
     0&0&0&0\\
     \ebm
     \hspace{4em}
     \bbm
     0&1&0&0\\
     0&0&1&0\\
     1&0&0&1\\
     0&1&0&0\\
     \ebm
      \to \bbm
     1&0&0&1\\
     0&1&0&0\\
     0&0&1&0\\
     0&0&0&0\\
     \ebm
   \]
 \begin{itemize}
  \item[(a)] Give examples of the following: \mrks{6}
   \begin{itemize}
    \item[(i)] A $3\tm 3$ RREF matrix $A$ such that $A^T$ is also in RREF.
    \item[(ii)] A $2\tm 4$ RREF matrix $B$ that is no longer in RREF
     if you delete the second column.
    \item[(iii)] A $3\tm 3$ RREF matrix $C$ in which four of the
     entries are not zero.
   \end{itemize}
  \item[(b)] Find the general solution for the following system of
   linear equations, or prove that there is no solution. \mrks{4}
   \begin{align*}
    2b &= 10+d \\
    c+d &= b+5 \\
    2c+2d &= a+3b-7 \\
    3b+3c &= a+36.
   \end{align*}
  \item[(c)] Consider the vectors
   \[
    v   = \bbm 4\\5\\6\\7 \ebm \qquad
    u_1 = \bbm 1\\2\\3\\4 \ebm \qquad
    u_2 = \bbm 5\\6\\7\\8 \ebm \qquad
    u_3 = \bbm 8\\7\\6\\5 \ebm \qquad
    u_4 = \bbm 4\\3\\2\\1 \ebm \qquad
   \]
   Either express $v$ as a linear combination of $u_1$, $u_2$, $u_3$ and
   $u_4$, or prove that that is impossible. \mrks{4}
  \item[(d)] Do the vectors $u_i$ in part~(c) form a basis for $\R^4$?
   Justify your answer. \mrks{2}
  \item[(e)] By performing row operations on $C-tI$ or otherwise, evaluate
   the characteristic polynomial of the following matrix: \mrks{4}
   \[ C = \bbm
           0&0&1&0\\
           1&0&0&1\\
           0&1&0&0\\
           0&0&1&0\\
          \ebm.
   \]
 \end{itemize}
\end{problem}
\begin{solution}
 \begin{itemize}
  \item[(a)]
   \begin{itemize}
    \item[(i)] The only possibilities are as follows:
     \[ \bbm 0&0&0 \\ 0&0&0 \\ 0&0&0 \ebm \qquad
        \bbm 1&0&0 \\ 0&0&0 \\ 0&0&0 \ebm \qquad
        \bbm 1&0&0 \\ 0&1&0 \\ 0&0&0 \ebm \qquad
        \bbm 1&0&0 \\ 0&1&0 \\ 0&0&1 \ebm.
     \]
     Any one is worth two marks. \mks{2}
    \item[(ii)] The simplest example is
     \[ B = \bbm 0&1&0&0 \\ 0&0&1&0 \ebm. \]
     If we delete the second column, then we get a matrix with a zero
     row above a nonzero row, which is therefore not in RREF. \mks{2}
    \item[(iii)] All possible answers have the form
     \[ C = \bbm 1 & 0 & a \\ 0 & 1 & b \\ 0 & 0 & 0 \ebm \]
     with $a,b\neq 0$. \mks{2}
   \end{itemize}
  \item[(b)] We write the equations more tidily, convert to an
   augmented matrix, row reduce (using one of the given
   row-reductions), and convert back to a system of equations \mks{3}.
   \[
    \begin{array}{rl}
     0a +2b + 0c -  d &= 10 \\
     0a - b +  c +  d &=  5 \\
     -a -3b + 2c + 2d &= -7 \\
     -a +3b + 3c + 0d &= 36 
    \end{array}
    \to 
    \left[\begin{array}{cccc|c}
      0 &  2 &  0 & -1 & 10 \\
      0 & -1 &  1 &  1 &  5 \\
     -1 & -3 &  2 &  2 & -7 \\
     -1 &  3 &  3 &  0 & 36
    \end{array}\right]
    \to 
    \left[\begin{array}{cccc|c}
      1 & 0 & 0 & 0 & 9 \\
      0 & 1 & 0 & 0 & 8 \\
      0 & 0 & 1 & 0 & 7 \\
      0 & 0 & 0 & 1 & 6
    \end{array}\right]
    \to
    \begin{array}{rl}
     a &= 9 \\
     b &= 8 \\
     c &= 7 \\
     d &= 6.
    \end{array}
   \]
   This gives the unique solution of the original system. \mk
  \item[(c)] It is easiest to do this by inspection.  We have
   \[ u_2 - u_1 = \bbm 4\\4\\4\\4 \ebm \qquad \text{ and } \qquad
      v - u_1 = \bbm 3\\3\\3\\3 \ebm = 
      \frac{3}{4}(u_2-u_1),
   \]
   which gives
   \[ v = u_1 + \frac{3}{4}(u_2-u_1) = \frac{1}{4}u_1 + \frac{3}{4}u_2.
     \mks{4}
   \]
   Alternatively, we can row-reduce the matrix
   $[u_1|u_2|u_3|u_4|v]$:
   \[ \hspace{-8em}
    \bbm
    1&5&8&4&4\\
    2&6&7&3&5\\
    3&7&6&2&6\\
    4&8&5&1&7\\
    \ebm
    \to
    \bbm
    1&5&8&4&4\\
    0&-4&-9&-5&-3\\
    0&-8&-18&-10&-6\\
    0&-12&-27&-15&-9\\
    \ebm
    \to
    \bbm
    1&5&8&4&4\\
    0&1&9/4&5/4&3/4\\
    0&-8&-18&-10&-6\\
    0&-12&-27&-15&-9\\
    \ebm
    \to
    \bbm
    1&0&-13/4&-9/4&1/4\\
    0&1&9/4&5/4&3/4\\
    0&0&0&0&0\\
    0&0&0&0&0\\
    \ebm
   \]
   This shows that $v=\lm_1u_1+\dotsb+\lm_4u_4$ whenever 
   \begin{align*}
    \lm_1 - \tfrac{13}{4}\lm_3-\tfrac{9}{4}\lm_4 &= \tfrac{1}{4} \\
    \lm_2 + \tfrac{9}{4}\lm_3+\tfrac{5}{4}\lm_4 &= \tfrac{3}{4}.
   \end{align*}
   Here $\lm_3$ and $\lm_4$ are independent variables, which can take
   any value.  In particular, we can take $\lm_3=\lm_4=0$, giving
   $\lm_1=1/4$ and $\lm_2=3/4$.  This gives
   $v=\tfrac{1}{4}u_1+\tfrac{3}{4}u_2$, just as before. 
  \item[(d)] It is easy to see that $u_1+u_3=u_2+u_4$, or equivalently
   $u_1-u_2+u_3-u_4=0$.  This nontrivial relation shows that the list
   $u_1,u_2,u_3,u_4$ is linearly dependent and therefore not a basis. \mks{2}
   Alternatively, we can delete the last column from the row-reduction
   in~(d) to see that the matrix $[u_1|u_2|u_3|u_4]$ does not reduce to
   $I_4$, which again proves that the list is not a basis.
  \item[(e)] We write down $C-tI$ and perform some row operations as
   follows.
   \[  \hspace{-8em}
    \bbm
    -t&0&1&0\\
    1&-t&0&1\\
    0&1&-t&0\\
    0&0&1&-t\\
    \ebm
    \to
    \bbm
    0&-t^2&1&t\\
    1&-t&0&1\\
    0&1&-t&0\\
    0&0&1&-t\\
    \ebm
    \to
    \bbm
    0&0&-t^3+1&t\\
    1&-t&0&1\\
    0&1&-t&0\\
    0&0&1&-t\\
    \ebm
    \to
    \bbm
    0&0&0&-t^4+2t\\
    1&-t&0&1\\
    0&1&-t&0\\
    0&0&1&-t\\
    \ebm = D \mks{2}
   \]
   We add $t$ times row $2$ to row $1$, then add $t^2$ times row $3$ to
   row $1$, then add $t^3-1$ times row $4$ to row $1$, giving the
   matrix $D$ shown above.  None of these operations change the
   determinant, so $\det(C-tI)=\det(D)$.  We can expand $D$ along the
   top row to get 
   \[ \det(C-tI) = (-1)^{1+4}(-t^4+2t)
       \det\bbm 1 & -t & 0 \\ 0 & 1 & -t \\ 0 & 0 & 1 \ebm.
   \]
   The $3\tm 3$ matrix here is upper triangular, so the determinant is
   the product of the diagonal entries, which is one.  It follows that
   $\det(C-tI)=t^4-2t$. \mks{2} 
 \end{itemize}
\end{solution}

\begin{problem}[2012-13 resit]
 You may assume the following row reductions.  Some of them are
 relevant, and some of them are not.  
 \[ \bbm 0&0&0&1\\ 0&0&1&10\\ 0&1&0&12\\ 1&11&-11&11\\ \ebm \to 
    \bbm 1&0&0&0\\ 0&1&0&0\\ 0&0&1&0 \\ 0&0&0&1 \ebm \hspace{4em}
    \bbm 0&0&0&1\\ 0&0&1&11\\ 0&1&0&-11\\ 1&10&12&11 \ebm \to
    \bbm 1&0&0&0\\ 0&1&0&0\\ 0&0&1&0 \\ 0&0&0&1 \ebm
 \]
 \[
 \bbm 2&3&1\\ 0&3&-3\\ 1&2&0\\ 3&2&4\\ \ebm \to
 \bbm 1&0&2\\ 0&1&-1\\ 0&0&0\\ 0&0&0\\ \ebm \hspace{3em}
 \bbm 2&0&1&3\\ 3&3&2&2\\ 1&-3&0&4\\ \ebm \to
 \bbm 1&0&1/2&3/2\\ 0&1&1/6&-5/6\\ 0&0&0&0\\ \ebm
 \]
 \[
 \bbm -1&-2&1&1&2\\ 0&0&2&1&1\\ 1&2&2&0&1\\ \ebm \to
 \bbm 1&2&0&0&-3\\ 0&0&1&0&2\\ 0&0&0&1&-3\\ \ebm \hspace{3em}
 \bbm -1&0&1\\ -2&0&2\\ 1&2&2\\ 1&1&0\\ 2&1&1\\ \ebm \to
 \bbm 1&0&0\\ 0&1&0\\ 0&0&1\\ 0&0&0\\ 0&0&0\\ \ebm
 \]
 \begin{itemize}
  \item[(a)] Explain what it means for a matrix to be in reduced row
   echelon form (RREF). \mrks{4}
  \item[(b)] List all the $2\tm 2$ matrices in RREF for which every
   entry is $0$ or $1$.  \mrks{4}
  \item[(c)] Find the general solution for the following system of
   linear equations, or prove that there is no solution. \mrks{3}
   \begin{align*}
    r + s &= 2q+p+2 \\
    2r+s &= 1 \\
    2r+p+2q &= 1.
   \end{align*}
  \item[(d)] Consider the vectors
   \[
    u_1 = \bbm 2\\0\\1\\3\ebm \qquad
    u_2 = \bbm 3\\3\\2\\2 \ebm \qquad
    u_3 = \bbm 1\\ -3\\ 0\\ 4\ebm.
   \]
   Are they linearly independent?  Justify your answer. \mrks{3}
  \item[(e)] Either find the inverse of the following matrix, or show
   that it has no inverse: \mrks{4}
   \[ A = \bbm
           0&0&0&1\\
           0&0&1&10\\
           0&1&0&12\\
           1&11&-11&11\\
          \ebm
   \]
  \item[(f)] Is zero an eigenvalue of $A$?  Justify your answer. \mrks{2} 
 \end{itemize}
\end{problem}
\begin{solution}
 \begin{itemize}
  \item[(a)] A matrix $A$ is in RREF if 
   \begin{itemize}
    \item Any rows of zeros are at the bottom, after all the nonzero
     rows. \mk
    \item In every nonzero row, the first nonzero entry (called the
     \emph{pivot}) is equal to one. \mk
    \item Each pivot is further to the right than the pivot in the
     previous row. \mk
    \item All entries above or below a pivot are zero. \mk
   \end{itemize}
  \item[(b)] The possibilities are as follows:
   \[ \bbm 0&0\\0&0 \ebm \hspace{3em}
      \bbm 0&1\\0&0 \ebm \hspace{3em}
      \bbm 1&0\\0&0 \ebm \hspace{3em}
      \bbm 1&1\\0&0 \ebm \hspace{3em}
      \bbm 1&0\\0&1 \ebm. \mks{4}
   \]
  \item[(c)] We write the equations more tidily, convert to an
   augmented matrix, row reduce (using one of the given
   row-reductions), and convert back to a system of equations \mks{2}.
   \[ \hspace{-4em}
    \begin{array}{rl}
     -p-2q+r+s &= 2 \\
     2r+s &= 1 \\
     p+2q+2r &= 1
    \end{array}
    \to 
    \left[\begin{array}{cccc|c}
      -1 & -2 &  1 &  1 &  2 \\
       0 &  0 &  2 &  1 &  1 \\
       1 &  2 &  1 &  0 &  1
    \end{array}\right]
    \to 
    \left[\begin{array}{cccc|c}
       1&2&0&0&-3\\
       0&0&1&0&2\\
       0&0&0&1&-3\\
    \end{array}\right]
    \to
    \begin{array}{rl}
     p+2q &= -3 \\
     r &= 2 \\
     s &= -3.
    \end{array}
   \]
   The variable $q$ is independent so we move it to the right hand
   side.  The general solution is $p=-3-2q$ with $r=2$ and $s=-3$ and $q$
   arbitrary. \mk
  \item[(d)] Put 
   \[ A = [u_1|u_2|u_3] =
       \bbm
       2&3&1\\
       0&3&-3\\
       1&2&0\\
       3&2&4\\
       \ebm
   \]
   From the given row-reductions, we see that 
   \[ A \to 
       \bbm
       1&0&2\\
       0&1&-1\\
       0&0&0\\
       0&0&0\\
       \ebm
     %   \hspace{4em}
     % A^T \to 
     %   \bbm
     %   1&0&1/2&3/2\\
     %   0&1&1/6&-5/6\\
     %   0&0&0&0\\
     %   \ebm.
   \]
   The row-reduction of $A$ does not have a pivot in every column, so
   the vectors $u_i$ are linearly dependent.  \mks{3} (In fact we have a
   linear relation $-2u_1+u_2+u_3=0$, but students are not required to
   find that.)
%   Also, the row-reduction of $A^T$ does not have a
%   pivot in every column, so the vectors $u_i$ do not span $\R^4$.
%   Alternatively, we can recall that no list of $3$ vectors can ever
%   span $\R^4$, because $3<4$.
  \item[(e)] We row-reduce $[A|I_4]$ \mks{2} as follows:
   \[
    \left[\begin{array}{cccc|cccc}
     0&0&0&1&1&0&0&0\\
     0&0&1&10&0&1&0&0\\
     0&1&0&12&0&0&1&0\\
     1&11&-11&11&0&0&0&1\\
    \end{array}\right]
   \to
    \left[\begin{array}{cccc|cccc}
   0& 0&  0&1&  1&0&0&0\\
   0& 0&  1&0&-10&1&0&0\\
   0& 1&  0&0&-12&0&1&0\\
   1&11&-11&0&-11&0&0&1\\
   \end{array}\right]
   \to
   \] 
   \[
    \left[\begin{array}{cccc|cccc}
   0& 0&0&1&   1& 0&0&0\\
   0& 0&1&0& -10& 1&0&0\\
   0& 1&0&0& -12& 0&1&0\\
   1&11&0&0&-121&11&0&1\\
   \end{array}\right]
   \to
    \left[\begin{array}{cccc|cccc}
   0&0&0&1&  1& 0&  0&0\\
   0&0&1&0&-10& 1&  0&0\\
   0&1&0&0&-12& 0&  1&0\\
   1&0&0&0& 11&11&-11&1\\
   \end{array}\right]
   \to
   \] 
   \[
    \left[\begin{array}{cccc|cccc}
   1&0&0&0&11&11&-11&1\\
   0&1&0&0&-12&0&1&0\\
   0&0&1&0&-10&1&0&0\\
   0&0&0&1&1&0&0&0\\
   \end{array}\right] \mk
   \]
   In step $1$ we subtract multiples of the first row from the other
   three rows.  In step 2 we add $11$ times row $2$ to row $4$, and in
   step 3 we subtract $11$ times row $3$ from row $4$.  In step $4$ we
   swap the top and bottom rows, and we also swap the middle two
   rows.  The left hand block is now $I_4$, so $A$ is invertible, with
   inverse given by the right hand block:
   \[ A^{-1} = 
       \bbm
       11&11&-11&1\\
       -12&0&1&0\\
       -10&1&0&0\\
       1&0&0&0\\
       \ebm. \mk
   \]
  \item[(f)] For any square matrix, zero is an eigenvalue if and only
   if the matrix is not invertible.  We have just seen that $A$ is
   invertible, so zero is not an eigenvalue of $A$ \mks{2}.  
 \end{itemize}
\end{solution}


\begin{problem}[2012-13]
 You may assume the following row reductions.  Some of them are
 relevant, and some of them are not.  
 \begin{align*}
    \bbm
   2&-1&-1&3\\
   -2&2&3&7\\
   -2&12&23&107\\
   3&-2&-3&-2\\
   \ebm
   &\to
    \bbm
   1&0&0&5\\
   0&1&0&4\\
   0&0&1&3\\
   0&0&0&0\\
   \ebm
    &
    \bbm
   2&-2&-2&3\\
   -1&2&12&-2\\
   -1&3&23&-3\\
   3&7&107&-2\\
   \ebm
    &\to
    \bbm
   1&0&10&0\\
   0&1&11&0\\
   0&0&0&1\\
   0&0&0&0\\
   \ebm
    \\
   \bbm
   1&-1&0&1&3\\
   3&-3&1&0&11\\
   -2&2&0&-3&-7
   \ebm
    &\to
    \bbm
   1&-1&0&0&2\\
   0&0&1&0&5\\
   0&0&0&1&1
   \ebm
    & 
    \bbm
   1&3&-2\\
   -1&-3&2\\
   0&1&0\\
   1&0&-3\\
   3&11&-7
   \ebm
   &\to
    \bbm
   1&0&0\\
   0&1&0\\
   0&0&1\\
   0&0&0\\
   0&0&0
   \ebm
    \\
 \end{align*}

 \begin{itemize}
  \item[(a)] Explain what it means for a matrix to be in reduced row
   echelon form (RREF). \mrks{4}
  \item[(b)] Give an example of a $4\tm 4$ RREF matrix with pivots in
   columns $1$ and $3$, and precisely five nonzero entries. \mrks{2}
  \item[(c)] Find the general solution for the following system of
   linear equations, or prove that there is no solution. \mrks{4}
   \begin{align*}
    a+d &= b+3 \\
    3a+c &= 3b+11 \\
    -2a+2b &= 3d-7.
   \end{align*}
  \item[(d)] Consider the vectors
   \[
    v = \bbm 3\\ -2\\-3\\-2 \ebm \qquad
    u_1 = \bbm 2\\-1\\-1\\3\ebm \qquad
    u_2 = \bbm -2\\2\\3\\7 \ebm \qquad
    u_3 = \bbm -2\\ 12\\ 23\\ 107\ebm
   \]
   Either express $v$ as a linear combination of $u_1$, $u_2$ and
   $u_3$, or prove that that is impossible. \mrks{3}
  \item[(e)] By performing row operations or otherwise, evaluate the
   determinant of the following matrix: \mrks{4}
   \[ A = \bbm
       2 & 2 & 0 & 0 \\
       2 & 4 & 3 & 0 \\
       2 & 4 & 6 & 4 \\
       1 & 2 & 3 & 4 
      \ebm
   \]
  \item[(f)] Do the columns of $A$ span $\R^4$?  Justify your
   answer. \mrks{3} 
 \end{itemize}
\end{problem}
\begin{solution}
 \begin{itemize}
  \item[(a)] A matrix $A$ is in RREF if 
   \begin{itemize}
    \item Any rows of zeros are at the bottom, after all the nonzero
     rows. \mk
    \item In every nonzero row, the first nonzero entry (called the
     \emph{pivot}) is equal to one. \mk
    \item Each pivot is further to the right than the pivot in the
     previous row. \mk
    \item All entries above or below a pivot are zero. \mk
   \end{itemize}
  \item[(b)] There are many possible answers: the simplest one is the
   matrix 
   \[ \bbm 1&1&0&1 \\ 0&0&1&1 \\ 0&0&0&0 \\ 0&0&0&0 \ebm. \mks{2} \]
  \item[(c)] We write the equations more tidily, convert to an
   augmented matrix, row reduce (using one of the given
   row-reductions), and convert back to a system of equations \mks{3}.
   \[
    \begin{array}{rl}
     a-b+d &= 3 \\
     3a-3b+c &= 11 \\
     -2a+2b-3d &= -7 
    \end{array}
    \to 
    \left[\begin{array}{cccc|c}
     1&-1&0&1&3\\
     3&-3&1&0&11\\
     -2&2&0&-3&-7
    \end{array}\right]
    \to 
    \left[\begin{array}{cccc|c}
     1&-1&0&0&2\\
     0&0&1&0&5\\
     0&0&0&1&1
    \end{array}\right]
    \to
    \begin{array}{rl}
     a-b &= 2 \\
     c &= 5 \\
     d &= 1.
    \end{array}
   \]
   The variable $b$ is independent so we move it to the right hand
   side.  The general solution is $a=b+2$ with $c=5$ and $d=1$ and $b$
   arbitrary. \mk
  \item[(d)] We construct the matrix $[u_1|u_2|u_3|v]$, and observe
   that it can be reduced by one of the given row-reductions:
   \[
     [u_1|u_2|u_3|v] = 
    \left[\begin{array}{ccc|c}
   2&-2&-2&3\\
   -1&2&12&-2\\
   -1&3&23&-3\\
   3&7&107&-2\\
    \end{array}\right]
    \to
    \left[\begin{array}{ccc|c}
   1&0&10&0\\
   0&1&11&0\\
   0&0&0&1\\
   0&0&0&0\\
    \end{array}\right]. \mks{2}
   \] 
   The resulting matrix has a pivot in the last column, which
   indicates that $v$ cannot be expressed as a linear combination of
   $u_1$, $u_2$ and $u_3$. \mk
  \item[(e)] We row-reduce $A$ as follows:
   \[
   \bbm
   2&2&0&0\\
   2&4&3&0\\
   2&4&6&4\\
   1&2&3&4\\
   \ebm
   \xra{1}
   \bbm
   1&1&0&0\\
   2&4&3&0\\
   2&4&6&4\\
   1&2&3&4\\
   \ebm
   \xra{2}
   \bbm
   1&1&0&0\\
   0&2&3&0\\
   0&2&6&4\\
   0&1&3&4\\
   \ebm
   \xra{3}
   \bbm
   1&1&0&0\\
   0&0&-3&-8\\
   0&0&0&-4\\
   0&1&3&4\\
   \ebm
   \xra{4}
   \bbm
   1&1&0&0\\
   0&1&3&4\\
   0&0&-3&-8\\
   0&0&0&-4\\
   \ebm \mks{2}
   \]
   In step $1$ we multiply the first row by $1/2$, so we remember a
   factor of $1/2$.  In step $2$ we add multiples of row $1$ to rows
   $2$, $3$ and $4$; this does not contribute any factors to the
   determinant.  In step $3$ we add multiples of row $4$ to rows $2$
   and $3$; again, this does not contribute any factors.  In step $4$
   we swap the last two rows, then swap the middle two rows.  This
   gives two factors of $-1$, which cancel each other out.  The final
   matrix is upper triangular, so the determinant is the product of
   the diagonal entries, which is $12$.  After dividing by the factor
   of $1/2$ from the first step, we see that $\det(A)=24$. \mks{2}

   An alternative method is to expand along the top row:
   \begin{align*}
    \det(A) &= 2\det\bbm 4&3&0 \\ 4&6&4 \\ 2&3&4 \ebm 
              -2\det\bbm 2&3&0 \\ 2&6&4 \\ 1&3&4 \ebm \\
    \det\bbm 4&3&0 \\ 4&6&4 \\ 2&3&4 \ebm 
     &= 4(6\tm 4 - 3\tm 4) 
       -3(4\tm 4 - 2\tm 4) = 24 \\
    \det\bbm 2&3&0 \\ 2&6&4 \\ 1&3&4 \ebm
     &= 2(6\tm 4 - 3\tm 4)
       -3(2\tm 4 - 1\tm 4) = 12 \\
    \det(A) &= 2\tm 24 - 2\tm 12 = 24.
   \end{align*}
  \item[(f)] As $\det(A)\neq 0$, we see that $A$ is invertible
   \mks{2}.  By a standard theorem, this means that the columns of $A$
   form a basis for $\R^4$, so they certainly span $\R^4$.\mk
 \end{itemize}
\end{solution}


\begin{problem}[2011-12 resit]
 \begin{itemize}
  \item[(a)] Which of the following matrices are in reduced row echelon form (RREF)? 
   Explain your answers. \mrks{3}
   \[ 
    A = \bbm 2&0&0&1 \\ 0&1&1&1 \\ 0&0&0&0 \ebm \qquad
    B = \bbm 0&0&1&0&0 \\ 0&0&0&1&0 \\ 0&0&0&0&1 \ebm \qquad
    C = \bbm 1&0&1&0&1 \\ 0&0&1&0&1 \\ 0&0&0&0&1 \ebm   
   \]
  \item[(b)] Row-reduce the following matrix. \mrks{6}
   \[ D = \bbm 7 & 14 &  3 & 15 & -12 \\ 
               5 & 10 &  5 & 25 &  0 \\
               3 &  6 &  7 & 35 & 12 \ebm 
   \]
  \item[(c)]
   You may assume the row-reduction 
   \[ 
    \bbm
      1 &  2 & -3 & 7  & 7 \\ 
     -3 &  1 &  2 & 14 & -14 \\
      2 & -3 &  1 & 7  & 7
    \ebm
    \to
    \bbm
     1 & 0 & -1 & 0 & 5 \\
     0 & 1 & -1 & 0 & 1 \\
     0 & 0 &  0 & 1 & 0
    \ebm 
   \]
   Solve the following two systems of equations (the first
   system on the left, and the second system on the right) : 
   \begin{align*}
    x+2y-3z  &= 7  &    x+2y-3z  &= 7  \\
    -3x+y+2z &= 14 &    -3x+y+2z &= -14 \\
    2x-3y+z  &= 7  &    2x-3y+z  &= 7
   \end{align*}
   In each case say whether the system has a unique solution, an
   infinite family of solutions, or no solution.  \mrks{6}
  \item[(d)] Find the determinant of the following matrix: \mrks{3}
   \[ E = \bbm 
       0 & a & 0 & b \\
       c & d & 0 & e \\ 
       f & 0 & 0 & g \\
       h & 0 & i & j 
      \ebm
   \]
  \item[(e)] State, with justification, which of the following
   matrices are invertible. \mrks{7}
   \[ \hspace{-4em}
    F = \bbm 1&2&3&4 \\ 11&21&31&41 \\ 101&201&301&401 \\ 1001&2001&3001&4001 \ebm 
    \hspace{2em}
    G = \bbm 6&2&1&5 \\ 7&3&1&4 \\ 9&4&4&3 \ebm 
    \hspace{2em}
    H = \bbm 2&1&-2\\ 3&1&-3 \\ -2&-1&1 \ebm
    \hspace{2em}
    J = \bbm 1&1&1&1 \\ 1&1&1&0 \\ 1&1&0&0 \\ 1&0&0&0 \ebm
   \]
 \end{itemize}
\end{problem}
\begin{solution}
 \begin{itemize}
  \item[(a)] The matrix $A$ is not in RREF because the first nonzero
   entry in the first row is not equal to one \mk.  The matrix $B$ is
   in RREF \mk.  The matrix $C$ is not in RREF because there is a
   nonzero entry in the first row lying above the pivot in the second
   row. \mk
  \item[(b)] If we start by subtracting twice row $3$ from row $1$, we
   avoid all denominators:
   \[
    \bbm
    7&14&3&15&-12\\
    5&10&5&25&0\\
    3&6&7&35&12\\
    \ebm
    \to
    \bbm
    1&2&-11&-55&-36\\
    5&10&5&25&0\\
    3&6&7&35&12\\
    \ebm
    \to
    \bbm
    1&2&-11&-55&-36\\
    0&0&60&300&180\\
    0&0&40&200&120\\
    \ebm
    \to
   \] \[
    \bbm
    1&2&-11&-55&-36\\
    0&0&1&5&3\\
    0&0&40&200&120\\
    \ebm
    \to
    \bbm
    1&2&0&0&-3\\
    0&0&1&5&3\\
    0&0&0&0&0\\
    \ebm
   \]
   Alternatively, we can proceed as follows:
   \[
    \bbm
    7&14&3&15&-12\\
    5&10&5&25&0\\
    3&6&7&35&12\\
    \ebm
    \to
    \bbm
    1&2&\tfrac{3}{7}&\tfrac{15}{7}&\tfrac{-12}{7}\\
    5&10&5&25&0\\
    3&6&7&35&12\\
    \ebm
    \to
    \bbm
    1&2&\tfrac{3}{7}&\tfrac{15}{7}&\tfrac{-12}{7}\\
    0&0&\tfrac{20}{7}&\tfrac{100}{7}&\tfrac{60}{7}\\
    0&0&\tfrac{40}{7}&\tfrac{200}{7}&\tfrac{120}{7}\\
    \ebm
    \to
   \] \[
    \bbm
    1&2&\tfrac{3}{7}&\tfrac{15}{7}&\tfrac{-12}{7}\\
    0&0&1&5&3\\
    0&0&\tfrac{40}{7}&\tfrac{200}{7}&\tfrac{120}{7}\\
    \ebm
    \to
    \bbm
    1&2&0&0&-3\\
    0&0&1&5&3\\
    0&0&0&0&0\\
    \ebm
   \]
   Either way: \mks{6}.
  \item[(c)] 
   The left hand system corresponds to the first augmented matrix
   shown below:
   \[ 
    \left[\begin{array}{ccc|c}
     1 & 2 & -3 & 7  \\
     -3 & 1 & 2 & 14  \\
     2 & -3 & 1 & 7 
    \end{array}\right]
    \to
    \left[\begin{array}{ccc|c}
     1 & 0 & -1 & 0 \\
     0 & 1 & -1 & 0 \\
     0 & 0 &  0 & 1 
    \end{array}\right]
   \]
   By deleting the last column from the row-reduction given in the
   question, we see that our matrix row-reduces as indicated, so the
   left hand system is equivalent to the system
   \[ x-z = 0 \hspace{4em} y-z = 0 \hspace{4em} 0=1, \mks{2}\]
   so there are no solutions \mk.

   Similarly, we can delete the fourth column from the given
   row-reduction to get 
   \[ 
    \left[\begin{array}{ccc|c}
     1 & 2 & -3 & 7  \\
     -3 & 1 & 2 & -14  \\
     2 & -3 & 1 & 7 
    \end{array}\right]
    \to
    \left[\begin{array}{ccc|c}
     1 & 0 & -1 & 5 \\
     0 & 1 & -1 & 1 \\
     0 & 0 &  0 & 0 
    \end{array}\right]
   \]
   This shows that the right-hand system is equivalent to the system
   \[ x-z = 5 \hspace{4em} y-z = 1 \hspace{4em} 0=0. \mk \]
   The solutions have the form
   $\bbm x\\y\\z\ebm=\bbm 5+z\\1+z\\ z\ebm$ with $z$ arbitrary \mk.  In
   particular, there are infinitely many solutions, one for each
   possible value of $z$ \mk.
  \item[(d)] We expand down the third column, then along the top row:
   \begin{align*}
    \det(E) &= -i \det\bbm 0&a&b \\ c&d&e \\ f&0&g \ebm \\
     &= -i\left(-a\det\bbm c&e \\ f&g \ebm 
                +b\det\bbm c&d \\ f&0 \ebm\right) \\
     &= ai(cg-ef) - ib(-df)
      = acgi - aefi + bdfi. \mks{3}
   \end{align*}
  \item[(e)]
   The matrix $F$ can be partially row-reduced as follows:
   \[
   \bbm
   1&2&3&4\\
   11&21&31&41\\
   101&201&301&401\\
   1001&2001&3001&4001
   \ebm
   \to
   \bbm
   1&2&3&4\\
   1&1&1&1\\
   1&1&1&1\\
   1&1&1&1
   \ebm
   \to
   \bbm
   0&1&2&3\\
   1&1&1&1\\
   0&0&0&0\\
   0&0&0&0
   \ebm
   \]
   The resulting matrix has a row of zeros so it is not invertible,
   and it follows that the original matrix $F$ is not invertible
   either \mks{2}.  The matrix $G$ is not square and so cannot be
   invertible \mk.  Next, the matrix $H$ can be row-reduced to the
   identity as follows: 
   \[
     \bbm
     2&1&-2\\
     3&1&-3\\
     -2&-1&1\\
     \ebm
     \to
     \bbm
     1&1/2&-1\\
     3&1&-3\\
     -2&-1&1\\
     \ebm
     \to
     \bbm
     1&1/2&-1\\
     0&-1/2&0\\
     0&0&-1\\
     \ebm
     \to
     \bbm
     1&1/2&-1\\
     0&1&0\\
     0&0&1\\
     \ebm
     \to
     \bbm
     1&0&0\\
     0&1&0\\
     0&0&1\\
     \ebm.
   \]
   This shows that $H$ is invertible.  Alternatively, we can calculate
   that $\det(H)=1\neq 0$, which also implies that $H$ is
   invertible \mks{2}.  Finally, the matrix is $J$ is also invertible.
   Indeed, it can easily be row-reduced to the identity, or we can
   calculate that $\det(J)=1$, or we can note that
   \[ \bbm 1&1&1&1 \\ 1&1&1&0 \\ 1&1&0&0 \\ 1&0&0&0 \ebm
      \bbm 0&0&0&1 \\ 0&0&1&-1 \\ 0&1&-1&0 \\ 1&-1&0&0 \ebm 
      = 
      \bbm 1&0&0&0 \\ 0&1&0&0 \\ 0&0&1&0 \\ 0&0&0&1 \ebm.\mks{2}
   \]
 \end{itemize}
\end{solution}


\begin{problem}[2011-12]
 \begin{itemize}
  \item[(a)] Which of the following matrices are in reduced row echelon form (RREF)? 
   Explain your answers. \mrks{3}
   \[ 
    A = \bbm 1&2&0&3 \\ 0&0&0&0 \\ 0&0&1&4 \ebm \qquad
    B = \bbm 0&0&0&0&1 \\ 0&0&1&0&0 \\ 1&0&0&0&0 \ebm \qquad
    C = \bbm 1&1&1 \\ 0&1&1 \\ 0&0&1 \ebm   
   \]
  \item[(b)] Row-reduce the following matrix. \mrks{6}
   \[ D = \bbm 11 & 10 &  1 &  1 & 11 \\ 
               11 &  1 & 10 & 10 &  1 \\
                1 &  1 &  0 &  0 & 10 \ebm 
   \]
  \item[(c)]
   You may assume the row-reduction 
   \[ 
    \bbm
     7 & -3 & 1 & -1 & 1 \\ 
     3 &  2 & 7 & 16 & 16 \\
     4 & -1 & 2 &  3 & -3
    \ebm
    \to
    \bbm
     1 & 0 & 1 & 2 & 0 \\
     0 & 1 & 2 & 5 & 0 \\
     0 & 0 & 0 & 0 & 1
    \ebm 
   \]
   Solve the following two systems of equations (the first
   system on the left, and the second system on the right) : 
   \begin{align*}
    7x-3y+z  &= -1 &
    7x-3y+z  &=  1 \\
    3x+2y+7z &= 16 &
    3x+2y+7z &= 16 \\
    4x-y+2z  &=  3 &
    4x-y+2z  &= -3 
   \end{align*}
   In each case say whether the system has a unique solution, an
   infinite family of solutions, or no solution.  \mrks{6}
  \item[(d)] Find the determinant of the following matrix: \mrks{3}
   \[ E = \bbm 
       1 & 0 & 2 & 0 \\
       0 & 0 & 3 & 0 \\ 
       0 & 4 & 0 & 4 \\
       0 & 4 & 0 & 5 
      \ebm
   \]
  \item[(e)] State, with justification, which of the following
   matrices are invertible. \mrks{7}
   \[ 
    F = \bbm 1&1&1&1 \\ 2&3&4&5 \\ 6&7&8&9 \\ 9&9&9&9 \ebm 
    \hspace{2em}
    G = \bbm 1&2&5 \\ 6&4&3 \\ 5&1&2 \\ 7&9&1 \ebm 
    \hspace{2em}
    H = \bbm -2&-2&-1\\ -1&0&1 \\ 2&-2&1 \ebm
    \hspace{2em}
    J = \bbm 0&1&0&0 \\ 1&0&0&0 \\ 0&0&0&1 \\ 0&0&1&0 \ebm
   \]
 \end{itemize}
\end{problem}
\begin{solution}
 \begin{itemize}
  \item[(a)] None of the matrices are in RREF.  The matrix $A$ is not
   in RREF because it has a row of zeros that does not occur after all
   the nonzero rows.\mk  The matrix $B$ is not in RREF because the pivot
   in the second row is to the left of the pivot in the first row.\mk
   The matrix $C$ is not in RREF because there are nonzero entries
   above the pivot in the third row.\mk
  \item[(b)] 
   \[
      \bbm 11 & 10 &  1 &  1 & 11 \\ 
           11 &  1 & 10 & 10 &  1 \\
            1 &  1 &  0 &  0 & 10 \ebm 
      \!\to\! 
      \bbm  0 & -1 &  1 &  1 &-99 \\ 
            0 &-10 & 10 & 10 &-109 \\
            1 &  1 &  0 &  0 & 10 \ebm 
      \!\to\! 
      \bbm  0 &  1 & -1 & -1 & 99 \\ 
            0 &  0 &  0 &  0 & 881 \\
            1 &  0 &  1 &  1 &-89 \ebm 
      \!\to\! 
      \bbm  1 &  0 &  1 &  1 & 0 \\
            0 &  1 & -1 & -1 & 0 \\ 
            0 &  0 &  0 &  0 & 1 \ebm. \mks{6}
   \]
  \item[(c)] 
   The left hand system corresponds to the first augmented matrix
   shown below:
   \[ 
    \left[\begin{array}{ccc|c}
     7 & -3 & 1 & -1 \\ 
     3 &  2 & 7 & 16 \\
     4 & -1 & 2 &  3
    \end{array}\right]
    \to
    \left[\begin{array}{ccc|c}
     1 & 0 & 1 & 2 \\
     0 & 1 & 2 & 5 \\
     0 & 0 & 0 & 0 
    \end{array}\right]
   \]
   By deleting the last column from the row-reduction given in the
   question, we see that our matrix row-reduces as indicated, so the
   left hand system is equivalent to the system
   \[ x+z = 2 \hspace{4em} y+2z = 5 \hspace{4em} 0 = 0. \mk \]
   The solutions have the form
   $\bbm x\\y\\z\ebm=\bbm 2-z\\5-2z\\ z\ebm$ with $z$ arbitrary \mk.  In
   particular, there are infinitely many solutions, one for each
   possible value of $z$ \mk.

   Similarly, we can delete the fourth column from the given
   row-reduction to get 
   \[ 
    \left[\begin{array}{ccc|c}
     7 & -3 & 1 & 1 \\ 
     3 &  2 & 7 & 16 \\
     4 & -1 & 2 & -3
    \end{array}\right]
    \to
    \left[\begin{array}{ccc|c}
     1 & 0 & 1 & 0 \\
     0 & 1 & 2 & 0 \\
     0 & 0 & 0 & 1
    \end{array}\right]
   \]
   This shows that the right-hand system is equivalent to the system
   \[ x+z = 0 \hspace{4em} y+2z = 0 \hspace{4em} 0=1, \mks{2}\]
   so there are no solutions \mk.
  \item[(d)] There are enough zeros in $E$ that a direct expansion is
   painless: 
   \begin{align*}
    \det(E) &= \det\bbm 0&3&0 \\ 4&0&4 \\ 4&0&5 \ebm 
               +2 \det\bbm 0&0&0 \\ 0&4&4 \\ 0&4&5 \ebm \mk \\
            &= -3\det\bbm 4&4\\ 4&5\ebm + 2\tm 0 \mk \\
            &= -3\tm(20-16) = -12. \mk
   \end{align*}
  \item[(e)]
   In matrix $F$ the last row is $9$ times the first row, so the rows
   are linearly dependent, so $F$ is not invertible \mks{2}.  The matrix $G$
   is not square and so cannot be invertible \mk.  Next, the matrix $H$
   can be row-reduced to the identity as follows:
   \[ \bbm -2 & -2 & -1 \\
           -1 &  0 &  1 \\
            2 & -2 &  1 \ebm
      \to
      \bbm  0 & -2 & -3 \\
            1 &  0 & -1 \\
            0 & -2 &  3 \ebm
      \to
      \bbm  0 & -2 & -3 \\
            1 &  0 & -1 \\
            0 &  0 &  6 \ebm
      \to
      \bbm  1 &  0 & -1 \\
            0 &  1 & 3/2 \\
            0 &  0 &  1 \ebm
      \to
      \bbm  1 &  0 &  0 \\
            0 &  1 &  0 \\
            0 &  0 &  1 \ebm
   \]
   This shows that $H$ is invertible.  Alternatively, we can calculate
   that $\det(H)=-12\neq 0$, which also implies that $H$ is
   invertible \mks{2}.  Finally, the matrix is $J$ is also invertible.  One of
   many ways to see this is to note that $J^2=I$, so $J$ is its own
   inverse. \mks{2}
 \end{itemize}
\end{solution}

\begin{problem}[Mock 1]
 \begin{itemize}
  \item[(a)] Which of the following matrices are in reduced row echelon form (RREF)? 
   Explain your answers. \mrks{3}
   \[ 
    A = \bbm 1&0&0&1&0&0 \\
             0&1&0&0&1&0 \\
             0&0&1&0&0&1 \ebm \qquad
    B = \bbm 1&0&0&1 \\
             0&1&1&0 \\
             0&0&0&1 \ebm \qquad
    C = \bbm 0&0&1 \\ 
             0&1&0 \\
             1&0&0 \ebm   
   \]
  \item[(b)] Row-reduce the following matrix. \mrks{6}
   \[ D = \bbm  2 &  2 &  2 &  4 &  6 \\ 
                3 &  3 &  3 &  5 &  7 \\
                5 &  5 &  5 &  8 &  9 \ebm 
   \]
  \item[(c)]
   You may assume the row-reduction 
   \[ \bbm
        1 & -2 &  0 &  1 & -3 &  2\\
       -2 &  4 &  0 & -2 &  6 & -4\\
        1 & -2 & -1 &  5 & -4 &  0
      \ebm \to
      \bbm
        1 & -2 &  0 &  1 & -3 &  2\\
        0 &  0 &  1 & -4 &  1 &  2\\
        0 &  0 &  0 &  0 &  0 &  0
      \ebm.
   \]
   Find the general solution of the system of equations 
   \begin{align*}
      v -2w     +y - 3z &= 2\\
    -2v +4w    -2y + 6z &= -4\\
      v -2w -x +5y - 4z &= 0
   \end{align*}
   Then find a specific solution (with no free variables) where
   $x=0$. \mrks{6}
  \item[(d)] Find the determinant of the following matrix: \mrks{3}
   \[ E = \bbm
           0&a&0&0\\
           b&c&d&e\\
           f&g&0&h\\
           i&j&0&k\\
          \ebm
   \]
  \item[(e)] State, with justification, which of the following
   matrices are invertible. \mrks{7}
   \[ 
    F = \bbm
    1&1&1&1\\
    1&1&1&2\\
    1&1&2&2\\
    1&2&2&2
    \ebm \qquad
    G = \bbm
    1&2&4&8\\
    1&1&1&1\\
    8&4&2&1
    \ebm \qquad
    H = \bbm
    -3& 1&-2\\
     1& 2& 2\\
     0&-1&-1
    \ebm \qquad
    J = \bbm
     100&20&3&123\\
     300&10&7&317\\
     500&70&1&571\\
     200&60&9&269\\
    \ebm
   \]
 \end{itemize}
\end{problem}
\begin{solution}
 \begin{itemize}
  \item[(a)] The matrix $A$ is in RREF \mk.  The matrix $B$ is not
   in RREF, because it has a nonzero entry above the pivot in the
   last column \mk.  The matrix $C$ is not in RREF because the pivot
   in the second row is to the left of the pivot in the first row \mk.
  \item[(b)] 
  \[
   \bbm
   2&2&2&4&6\\
   3&3&3&5&7\\
   5&5&5&8&9\\
   \ebm
   \to
   \bbm
   1&1&1&2&3\\
   3&3&3&5&7\\
   5&5&5&8&9\\
   \ebm
   \to
   \bbm
   1&1&1&2&3\\
   0&0&0&-1&-2\\
   0&0&0&-2&-6\\
   \ebm
   \to
   \bbm
   1&1&1&2&3\\
   0&0&0&1&2\\
   0&0&0&-2&-6\\
   \ebm
  \] \[
   \to
   \bbm
   1&1&1&0&-1\\
   0&0&0&1&2\\
   0&0&0&0&-2\\
   \ebm
   \to
   \bbm
   1&1&1&0&-1\\
   0&0&0&1&2\\
   0&0&0&0&1\\
   \ebm
   \to
   \bbm
   1&1&1&0&0\\
   0&0&0&1&0\\
   0&0&0&0&1\\
   \ebm \mks{6}
  \]
  \item[(c)] 
   The given system of equations corresponds to the augmented matrix
   \[ A = \left[\begin{array}{ccccc|c}
        1 & -2 &  0 &  1 & -3 &  2\\
       -2 &  4 &  0 & -2 &  6 & -4\\
        1 & -2 & -1 &  5 & -4 &  0
      \end{array}\right]. \mk
   \]
   We are told that this row-reduces to the matrix 
   \[ A' = \left[\begin{array}{ccccc|c}
        1 & -2 &  0 &  1 & -3 &  2\\
        0 &  0 &  1 & -4 &  1 &  2\\
        0 &  0 &  0 &  0 &  0 &  0
      \end{array}\right],
   \]
   so our original system is equivalent to the system
   \begin{align*}
    v-2w+y-3z &= 2 \\
    x-4y+z &= 2. \mks{2}
   \end{align*}
   The general solution is $v=2+2w-y+3z$ and $x=2+4y-z$ with $w$, $y$
   and $z$ arbitrary \mk.  To find a specific solution with $x=0$ we can
   take $w=y=0$ and $z=2$; this gives $v=2+2w-y+3z=8$ and $x=2+4y-z=0$
   so $\bbm v&w&x&y&z\ebm=\bbm 8&0&0&0&2\ebm$ \mks{2}.
  \item[(d)] Expand along the first row, then down the middle column:
   \[ \det(E) = -a\det\bbm b&d&e\\ f&0&h\\ i&0&k\ebm 
              = (-a)(-d)\det\bbm f&h\\i&k\ebm 
              = ad(fk-hi) = adfk-adhi. \mks{3}
   \]
  \item[(e)]
   The matrix $F$ can be row-reduced to the identity:
   \[ \bbm 1&1&1&1 \\ 1&1&1&2 \\ 1&1&2&2 \\ 1&2&2&2 \ebm \to 
      \bbm 1&1&1&1 \\ 0&0&0&1 \\ 0&0&1&1 \\ 0&1&1&1 \ebm \to 
      \bbm 1&1&1&0 \\ 0&0&0&1 \\ 0&0&1&0 \\ 0&1&1&0 \ebm \to 
      \bbm 1&1&0&0 \\ 0&0&0&1 \\ 0&0&1&0 \\ 0&1&0&0 \ebm \to 
      \bbm 1&0&0&0 \\ 0&0&0&1 \\ 0&0&1&0 \\ 0&1&0&0 \ebm \to 
      \bbm 1&0&0&0 \\ 0&1&0&0 \\ 0&0&1&0 \\ 0&0&0&1 \ebm.
   \]
   It follows that $F$ is invertible \mks{2}. The matrix $G$
   is not square and so cannot be invertible \mk.  Next, we can
   evaluate $\det(H)$ by expanding down the first column:
   \[ \det(H) = 
       -3\det\bbm 2&2\\-1&-1\ebm - \det\bbm 1&-2\\-1&-1\ebm 
       = -3\tm 0 - (-1-2) = 3.
   \]
   As $\det(H)\neq 0$, we see that $H$ is invertible.\mks{2}
   In the matrix $J$ the last column is the sum of the first three
   columns, so the columns are linearly dependent, so $J$ is not
   invertible. \mks{2}
 \end{itemize}
\end{solution}

\begin{problem}[Mock 2]
 \begin{itemize}
  \item[(a)] Which of the following matrices are in reduced row echelon form (RREF)? 
   Explain your answers. \mrks{4}
   \[ 
     A = \bbm
      1&0&0&2\\
      0&2&1&0\\
      0&0&0&0\\
     \ebm \qquad 
     B = \bbm
      1&2&0\\
      0&0&1\\
      0&0&0\\
      0&0&0\\
     \ebm \qquad
     C = \bbm
      1&1&0&0\\
      0&1&1&0\\
      0&0&1&1\\
     \ebm \qquad
     D = \bbm
      0&1&2\\
      1&0&0\\
      0&0&0\\
     \ebm
   \]
  \item[(b)] Row-reduce the following matrix. \mrks{6}
   \[ E = \bbm  2 &  2 &  2 &  4 &  6 \\ 
                3 &  3 &  3 &  5 &  7 \\
                5 &  5 &  5 &  8 &  9 \ebm 
   \]
  \item[(c)]
   Consider the vectors
   \[ 
     v_1 = \bbm 1\\ 3\\ 7\\ 2 \ebm \quad
     v_2 = \bbm 4\\ 1\\ 0\\ 2 \ebm \quad
     v_3 = \bbm 2\\ 1\\ 1\\ 1 \ebm \quad
     v_4 = \bbm 1\\ 2\\ 5\\ 2 \ebm \quad
     v_5 = \bbm 7\\ 6\\ 4\\ 5 \ebm \quad
     v_6 = \bbm 7\\ 4\\ 6\\ 5 \ebm \quad
   \]
   You may assume the row-reduction 
   \[ 
     \bbm
     1&4&2&1&7&7\\
     3&1&1&2&6&4\\
     7&0&1&5&4&6\\
     2&2&1&2&5&5\\
     \ebm 
     \to
     \bbm
     1&0&0&1&0&1\\
     0&1&0&1&0&2\\
     0&0&1&-2&0&-1\\
     0&0&0&0&1&0\\
     \ebm
   \]
   \begin{itemize}
    \item[(i)] Either express $v_5$ as a linear combination of
     $v_1,v_2,v_3,v_4$, or explain why that is not possible. \mrks{3}
    \item[(ii)] Either express $v_6$ as a linear combination of
     $v_1,v_2,v_3,v_4$, or explain why that is not possible. \mrks{3}
   \end{itemize}
  \item[(d)] Find inverses for the matrices $F$, $G$ and $H$ below,
   and explain why $J$ does not have an inverse. \mrks{9}
   \[
    F = \bbm
    1&2\\
    3&4\\
    \ebm \qquad
    G = \bbm
    0&0&1\\
    0&1&2\\
    1&2&3\\
    \ebm \qquad
    H = \bbm
    0&0&4&0\\
    3&0&0&0\\
    0&0&0&2\\
    0&1&0&0\\
    \ebm \qquad
    J = \bbm
    1&100&10&111\\
    10&1&100&111\\
    10&100&1&111\\
    100&1&10&111\\
    \ebm
   \]
   \textbf{Hint:} for $H$, trial and error may be easier than a
   systematic method.  
 \end{itemize}
\end{problem}
\begin{solution}
 \begin{itemize}
  \item[(a)] The matrix $A$ is not in RREF because the first nonzero
   entry in the second row is equal to $2$ \mk.  The matrix $B$ is 
   in RREF \mk.  The matrix $C$ is not in RREF because there are
   nonzero entries above the second and third pivots \mk.  The matrix
   $D$ is not in RREF because the pivot in the second row is to the
   left of the pivot in the first row \mk.
  \item[(b)] 
   \[
   \bbm
    2&0&4&-1& 0\\
   -2&1&0& 0&-1\\
   -1&1&2& 0& 2\\
   \ebm
   \to
   \bbm
    2&0&4&-1&0\\
   -2&1&0&0&-1\\
    1&-1&-2&0&-2\\
   \ebm
   \to
   \bbm
   0&2&8&-1&4\\
   0&-1&-4&0&-5\\
   1&-1&-2&0&-2\\
   \ebm
   \to
   \bbm
   0&2&8&-1&4\\
   0&1&4&0&5\\
   1&-1&-2&0&-2\\
   \ebm
  \] \[
   \to
   \bbm
   0&0&0&-1&-6\\
   0&1&4&0&5\\
   1&0&2&0&3\\
   \ebm
   \to
   \bbm
   0&0&0&1&6\\
   0&1&4&0&5\\
   1&0&2&0&3\\
   \ebm
   \to
   \bbm
   1&0&2&0&3\\
   0&1&4&0&5\\
   0&0&0&1&6\\
   \ebm \mks{6}
   \]
  \item[(c)] By deleting the last column in the given row-reduction we
   get 
   \[ [v_1\:v_2\:v_3\:v_4|v_5] = 
     \left[\begin{array}{cccc|c}
     1&4&2&1&7\\
     3&1&1&2&6\\
     7&0&1&5&4\\
     2&2&1&2&5\\
     \end{array}\right]
     \to
     \left[\begin{array}{cccc|c}
     1&0&0&1&0\\
     0&1&0&1&0\\
     0&0&1&-2&0\\
     0&0&0&0&1\\
     \end{array}\right] \mk
   \]
   As there is a pivot in the last column, we see that $v_5$ is not a
   linear combination of $v_1,v_2,v_3,v_4$.  \mks{2}

   Now instead delete the fifth column, to get 
   \[ [v_1\:v_2\:v_3\:v_4|v_6] = 
     \left[\begin{array}{cccc|c}
     1&4&2&1&7\\
     3&1&1&2&4\\
     7&0&1&5&6\\
     2&2&1&2&5\\
     \end{array}\right]
     \to
     \left[\begin{array}{cccc|c}
     1&0&0& 1& 1\\
     0&1&0& 1& 2\\
     0&0&1&-2&-1\\
     0&0&0& 0& 0\\
     \end{array}\right] \mk
   \]
   As there is no pivot in the last column we see that $v_6$ can be
   written as $\al_1v_1+\al_2v_2+\al_3v_3+\al_4v_4$, where the
   coefficients $\al_i$ must satisfy
   \begin{align*}
    \al_1+ \al_4 &= 1 \\
    \al_2+ \al_4 &= 2 \\
    \al_3-2\al_4 &= -1.
   \end{align*}
   We can take $\al_4=0$ and then we get $\al_1=1$ and $\al_2=2$ and
   $\al_3=-1$, which gives
   \[ v_6 = v_1+2v_2-v_3. \mks{2} \]
  \item[(d)]
   \[
    F = \bbm
    1&2\\
    3&4\\
    \ebm \qquad
    G = \bbm
    0&0&1\\
    0&1&2\\
    1&2&3\\
    \ebm \qquad
    H = \bbm
    0&0&4&0\\
    3&0&0&0\\
    0&0&0&2\\
    0&1&0&0\\
    \ebm \qquad
    J = \bbm
    1&100&10&111\\
    10&1&100&111\\
    10&100&1&111\\
    100&1&10&111\\
    \ebm
   \]

   \begin{itemize}
    \item For $F$ we just use the standard formula for $2\tm 2$
     matrices:
     \[ \bbm a&b\\ c&d\ebm^{-1} = \frac{1}{ad-bc}\bbm d&-b\\-c&a\ebm 
        \hspace{5em}
        F^{-1} = \frac{1}{-2} \bbm 4&-2\\-3&1\ebm 
           = \bbm -2 & 1 \\ 3/2 & -1/2 \ebm. \mks{2}
     \]
    \item For $G$ we row-reduce the matrix $[G|I_3]$ \mk:
     \[
      \left[\begin{array}{ccc|ccc}
       0&0&1&1&0&0\\
       0&1&2&0&1&0\\
       1&2&3&0&0&1\\
      \end{array}\right]
      \to
      \left[\begin{array}{ccc|ccc}
       1&2&3&0&0&1\\
       0&1&2&0&1&0\\
       0&0&1&1&0&0\\
      \end{array}\right]
      \to
      \left[\begin{array}{ccc|ccc}
       1&0&-1&0&-2&1\\
       0&1&2&0&1&0\\
       0&0&1&1&0&0\\
      \end{array}\right]
      \to
      \left[\begin{array}{ccc|ccc}
       1&0&0&1&-2&1\\
       0&1&0&-2&1&0\\
       0&0&1&1&0&0\\
      \end{array}\right]
     \]
     The final matrix is $[I_3|G^{-1}]$, so 
     $G^{-1}=\bbm 1&-2&1\\ -2&1&0 \\ 1&0&0\ebm$ \mks{2}.
    \item For $H$ we simply observe that 
     \[ \bbm
         0&0&4&0\\
         3&0&0&0\\
         0&0&0&2\\
         0&1&0&0\\
        \ebm
        \bbm 
         0&1/3&0&0 \\
         0&0&0&1 \\
         1/4&0&0&0 \\
         0&0&1/2&0
        \ebm
        = 
        \bbm
         1&0&0&0 \\
         0&1&0&0 \\
         0&0&1&0 \\
         0&0&0&1
        \ebm
     \]
     which shows that 
     \[ H^{-1} = 
        \bbm 
         0&1/3&0&0 \\
         0&0&0&1 \\
         1/4&0&0&0 \\
         0&0&1/2&0
        \ebm \mks{2}
     \]
    \item In $J$, the last column is the sum of the first three
     columns, so the columns are linearly dependent.  It follows that
     $J$ is not invertible \mks{2}.
   \end{itemize}
 \end{itemize}
\end{solution}

%%%%%%%%%%%%%%%%%%%%%%%%%%%%%%%%%%%%%%%%%%%%%%%%%%%%%%%%%%%%%%%%%%%%%%

\section{Question 2}
\setcounter{probcounter}{0}


\begin{problem}[2013-14]
 Consider the following Markov chain:
 \begin{center}
  \begin{tikzpicture}[scale=4,auto,shorten >= 4pt,shorten <= 4pt]
   \node (state1) at (  0:1) [circle,draw] {\Large 1}; 
   \node (state2) at ( 72:1) [circle,draw] {\Large 2}; 
   \node (state3) at (144:1) [circle,draw] {\Large 3}; 
   \node (state4) at (216:1) [circle,draw] {\Large 4}; 
   \node (state5) at (288:1) [circle,draw] {\Large 5}; 
   \draw[-angle 90,bend right] (state1) to (state2);
   \draw[-angle 90,bend right] (state2) to (state3);
   \draw[-angle 90,bend right] (state3) to (state4);
   \draw[-angle 90,bend right] (state4) to (state5);
   \draw[-angle 90,bend right] (state5) to (state1);
   \draw[-angle 90,bend right] (state1) to  (state3);
   \draw[-angle 90,bend right] (state2) to  (state4);
   \draw[-angle 90,bend right] (state3) to  (state5);
   \draw[-angle 90,bend right] (state4) to  (state1);
   \draw[-angle 90,bend right] (state5) to  (state2);
   \draw[-angle 90] (state1) to  (state4);
   \draw[-angle 90] (state2) to  (state5);
   \draw[-angle 90] (state3) to  (state1);
   \draw[-angle 90] (state4) to  (state2);
   \draw[-angle 90] (state5) to  (state3);
   \draw[-angle 90,bend left=10] (state1) to  (state5);
   \draw[-angle 90,bend left=10] (state2) to  (state1);
   \draw[-angle 90,bend left=10] (state3) to  (state2);
   \draw[-angle 90,bend left=10] (state4) to  (state3);
   \draw[-angle 90,bend left=10] (state5) to  (state4);
  \end{tikzpicture}
 \end{center}
 There is an arrow from every state to every other state, but no
 arrows from any state to itself.  All the arrows have the same
 probability $p$. 

 \begin{itemize}
  \item[(a)] What must $p$ be? \mrks{2}
  \item[(b)] Write down the transition matrix $P$. \mrks{2}
  \item[(c)] Calculate $P^2$, and thus find constants $\al$ and $\bt$
   such that $P^2=\al P+\bt I_5$. \mrks{4}
  \item[(d)] Show that if $v$ is an eigenvector for $P$ with
   eigenvalue $\lm$, then $\lm^2=\al\lm+\bt$. \mrks{2}
  \item[(e)] Use~(d) to find the eigenvalues of $P$. \mrks{3}
  \item[(f)] You may assume that $P$ has a unique stationary
   distribution.  What is it? \mrks{3} \\
   \textbf{Hint:} you could use row-reduction, but other methods
    are much easier.
  \item[(g)] Find a basis for $\R^5$ consisting of eigenvectors for
   $P$.  \mrks{4}
 \end{itemize}
\end{problem}
\begin{solution}
 \begin{itemize}
  \item[(a)] Each vertex has four outgoing arrows, each with the same
   probability $p$.  The sum of the outgoing probabilities must be
   one, so $p=1/4$. \mks{2}
  \item[(b)] There are no loops from a vertex back to itself, so the
   diagonal entries in $P$ are zero.  For any two states that are not
   the same, we have an arrow labelled with $p=1/4$.  Thus, all the
   off-diagonal entries in $P$ are $1/4$, so 
   \[ P = \frac{1}{4} \bbm
       0&1&1&1&1 \\
       1&0&1&1&1 \\
       1&1&0&1&1 \\
       1&1&1&0&1 \\
       1&1&1&1&0
      \ebm. \mks{2}
   \]
  \item[(c)] We have 
   \[ P^2 = \frac{1}{16}
      \bbm
       0&1&1&1&1 \\
       1&0&1&1&1 \\
       1&1&0&1&1 \\
       1&1&1&0&1 \\
       1&1&1&1&0
      \ebm 
      \bbm
       0&1&1&1&1 \\
       1&0&1&1&1 \\
       1&1&0&1&1 \\
       1&1&1&0&1 \\
       1&1&1&1&0
      \ebm = 
     \frac{1}{16}
      \bbm
       4&3&3&3&3 \\
       3&4&3&3&3 \\
       3&3&4&3&3 \\
       3&3&3&4&3 \\
       3&3&3&3&4
      \ebm. \mks{2}
   \]
   We want this to be the same as $\al P + \bt I$.  Looking at the
   diagonal entries, we see that $4/16=\bt$, or in other words
   $\bt=1/4$ \mk.  Looking at the off-diagonal entries, we see that
   $3/16=\al\tm(1/4)$, so $\al=3/4$ \mk.
  \item[(d)] If $v$ is an eigenvector for $P$ with eigenvalue $\lm$,
   we have 
   \begin{align*}
    Pv &= \lm v \\
    P^2v &= \lm^2 v \\
    (\al P + \bt I)v &= \al\lm v + \bt v = (\al\lm+\bt)v.
   \end{align*}
   As $P^2=\al P+\bt I$, it follows that $\lm^2v=(\al\lm+\bt)v$, or
   equivalently $(\lm^2-\al\lm-\bt)v=0$.  As $v$ is an eigenvector it
   is nonzero, so $\lm^2-\al\lm-\bt=0$. \mks{2}
  \item[(e)] We know that $\al=3/4$ and $\bt=1/4$, so the equation
   in~(d) becomes $\lm^2-3\lm/4-1/4=0$, which factors as
   $(\lm-1)(\lm+1/4)$.  We see from this that the eigenvalues are $1$
   and $-1/4$. \mks{3}
  \item[(f)] It is clear by symmetry that all entries in the
   stationary distribution must be the same.  There are five entries
   and they must add up to one, so the stationary distribution is 
   \[ u_1 = \bbm 1/5 & 1/5 & 1/5 & 1/5 & 1/5 \ebm. \mks{3} \]
   One could also obtain this by row-reducing the matrix $P-I$:
   {\tiny \[
   \bbm
   -1&1/4&1/4&1/4&1/4\\
   1/4&-1&1/4&1/4&1/4\\
   1/4&1/4&-1&1/4&1/4\\
   1/4&1/4&1/4&-1&1/4\\
   1/4&1/4&1/4&1/4&-1\\
   \ebm
   \to
   \bbm
   1&-1/4&-1/4&-1/4&-1/4\\
   0&-15/16&5/16&5/16&5/16\\
   0&5/16&-15/16&5/16&5/16\\
   0&5/16&5/16&-15/16&5/16\\
   0&5/16&5/16&5/16&-15/16\\
   \ebm
   \to
   \bbm
   1&-1/4&-1/4&-1/4&-1/4\\
   0&1&-1/3&-1/3&-1/3\\
   0&5/16&-15/16&5/16&5/16\\
   0&5/16&5/16&-15/16&5/16\\
   0&5/16&5/16&5/16&-15/16\\
   \ebm
   \to
  \] \[
   \bbm
   1&0&-1/3&-1/3&-1/3\\
   0&1&-1/3&-1/3&-1/3\\
   0&0&-5/6&5/12&5/12\\
   0&0&5/12&-5/6&5/12\\
   0&0&5/12&5/12&-5/6\\
   \ebm
   \to
   \bbm
   1&0&-1/3&-1/3&-1/3\\
   0&1&-1/3&-1/3&-1/3\\
   0&0&1&-1/2&-1/2\\
   0&0&5/12&-5/6&5/12\\
   0&0&5/12&5/12&-5/6\\
   \ebm
   \to
   \bbm
   1&0&0&-1/2&-1/2\\
   0&1&0&-1/2&-1/2\\
   0&0&1&-1/2&-1/2\\
   0&0&0&-5/8&5/8\\
   0&0&0&5/8&-5/8\\
   \ebm
   \to
  \] \[
   \bbm
   1&0&0&-1/2&-1/2\\
   0&1&0&-1/2&-1/2\\
   0&0&1&-1/2&-1/2\\
   0&0&0&1&-1\\
   0&0&0&5/8&-5/8\\
   \ebm
   \to
   \bbm
   1&0&0&0&-1\\
   0&1&0&0&-1\\
   0&0&1&0&-1\\
   0&0&0&1&-1\\
   0&0&0&0&0\\
   \ebm
   \]}
   From this last matrix we see again that all entries in the
   stationary distribution must be the same.
  \item[(g)] The stationary distribution $u_1$ is an eigenvector, of
   eigenvalue $1$ \mk.  The eigenvectors of eigenvalue $-1/4$ must satisfy
   $(P+\frac{1}{4}I)u=0$.  However, every entry in $P+\frac{1}{4}I$ is
   $1/4$, so a vector $u=\bbm v&w&x&y&z\ebm^T$ satisfies
   $(P+\frac{1}{4}I)u=0$ if and only if $(v+w+x+y+z)/4=0$ \mk.  We therefore
   have linearly independent eigenvectors as follows:
   \[ u_2 = \bbm 1 \\ 0 \\ 0 \\ 0 \\ -1 \ebm
      u_3 = \bbm 0 \\ 1 \\ 0 \\ 0 \\ -1 \ebm
      u_4 = \bbm 0 \\ 0 \\ 1 \\ 0 \\ -1 \ebm
      u_5 = \bbm 0 \\ 0 \\ 0 \\ 1 \\ -1 \ebm.
   \]
   The full list $u_1,\dotsc,u_5$ is now a basis for $\R^5$ consisting
   of eigenvectors for $P$. \mks{2}
 \end{itemize}
\end{solution}

\begin{problem}[2012-13 resit]
 Consider the following Markov chain:
 \begin{center}
  \begin{tikzpicture}[scale=3,auto,shorten >= 1pt]
   \node (state1) at ( 0.0,1.0) [circle,draw] {$1$}; 
   \node (state2) at ( 1.0,1.0) [circle,draw] {$2$}; 
   \node (state3) at ( 0.0,0.0) [circle,draw] {$3$}; 
   \node (state4) at ( 1.0,0.0) [circle,draw] {$4$}; 
   \draw[-triangle 90,loop left ] (state1) to node {$\tfrac{1}{3}$} (state1);
   \draw[-triangle 90,loop right] (state2) to node {$\tfrac{3}{4}$} (state2);
   \draw[-triangle 90,loop left ] (state3) to node {$\tfrac{3}{4}$} (state3);
   \draw[-triangle 90,loop right] (state4) to node {$\tfrac{11}{12}$} (state4);
   \draw[-triangle 90] (state1) to node        {$\tfrac{1}{3}$ } (state2);
   \draw[-triangle 90] (state1) to node [swap] {$\tfrac{1}{3}$ } (state3);
   \draw[-triangle 90] (state2) to node        {$\tfrac{1}{4}$ } (state4);
   \draw[-triangle 90] (state3) to node [swap] {$\tfrac{1}{4}$ } (state4);
   \draw[-triangle 90] (state4) to node        {$\tfrac{1}{12}$} (state1);
  \end{tikzpicture}
 \end{center}
 \begin{itemize}
  \item[(a)] Write down the transition matrix $P$. \mrks{2}
  \item[(b)] Consider the following vectors:
   \[ u_1 = \bbm 1 \\ -1 \\ -1 \\ 1 \ebm \qquad
      u_2 = \bbm 1 \\ -2 \\ -2 \\ 3 \ebm
   \]
   Show that these are eigenvectors for $P$, and find the
   corresponding eigenvalues.\\  \mrks{4}

   \textbf{Note:} it is not necessary to calculate the characteristic
   polynomial.
  \item[(c)] Find another eigenvector of the form
   \[ u_3 = \bbm 0 \\ x \\ y \\ 0 \ebm, \]
   and state the corresponding eigenvalue. \mrks{4}
  \item[(c)] Using the general theory of Markov chains, write down one
   more eigenvalue; then find a corresponding eigenvector.  \mrks{5}
  \item[(d)] Write down an eigenvector of $P^T$.  \mrks{2}
  \item[(e)] What is the long-run average probability of being in
   state 3? \mrks{3}
 \end{itemize}
\end{problem}
\begin{solution}
 \begin{itemize}
  \item[(a)] $\displaystyle P=
    \bbm 1/3 & 0   & 0   & 1/12  \\
         1/3 & 3/4 & 0   & 0     \\
         1/3 & 0   & 3/4 & 0     \\
         0   & 1/4 & 1/4 & 11/12 
    \ebm = \frac{1}{12} 
    \bbm 4 & 0 & 0 & 1 \\
         4 & 9 & 0 & 0 \\
         4 & 0 & 9 & 0 \\
         0 & 3 & 3 & 11 
    \ebm 
    $. \mks{2}
  \item[(b)] We have 
   \begin{align*}
    Pu_1 &= \frac{1}{12} 
    \bbm 4 & 0 & 0 & 1 \\
         4 & 9 & 0 & 0 \\
         4 & 0 & 9 & 0 \\
         0 & 3 & 3 & 11 
    \ebm \bbm 1 \\ -1 \\ -1 \\ 1 \ebm =
    \frac{1}{12} \bbm 5 \\ -5 \\ -5 \\ 5 \ebm = 
    \frac{5}{12} u_1 \mk \\
    Pu_2 &= \frac{1}{12} 
    \bbm 4 & 0 & 0 & 1 \\
         4 & 9 & 0 & 0 \\
         4 & 0 & 9 & 0 \\
         0 & 3 & 3 & 11 
    \ebm \bbm 1 \\ -2 \\ -2 \\ 3 \ebm =
    \frac{1}{12} \bbm 7 \\ -14 \\ -14 \\ 21 \ebm = 
    \frac{7}{12} u_2 \mk
   \end{align*}
   This shows that $u_1$ and $u_2$ are eigenvectors for $P$,
   with eigenvalues $\lm_1=5/12$ and $\lm_2=7/12$. \mks{2}
  \item[(c)] If $u_3=\bbm 0 & x & y & 0\ebm^T$ then
   \[ Au_3 = \frac{1}{12} 
    \bbm 4 & 0 & 0 & 1 \\
         4 & 9 & 0 & 0 \\
         4 & 0 & 9 & 0 \\
         0 & 3 & 3 & 11 
    \ebm \bbm 0 \\ x \\ y \\ 0 \ebm =
    \frac{1}{12} \bbm 0 \\ 9x \\ 9y \\ 3(x+y) \ebm.  \mk
   \]
   For $u_3$ to be an eigenvector, $Au_3$ must be a scalar multiple of
   $u_3$, so the last entry must be zero, so $y=-x$. \mk Taking $x=1$ and
   $y=-1$ we get 
   \[ u_3 = \bbm 0 \\ 1 \\ -1 \\ 0 \ebm \hspace{4em}
      Au_3 = \frac{1}{12} \bbm 0 \\ 9 \\ 9 \\ 0 \ebm = 
      \frac{9}{12} u_3 = \frac{3}{4} u_3,
   \]
   so $u_3$ is an eigenvector of eigenvalue $\lm_3=3/4$. \mks{2}
  \item[(c)] By the general theory of Markov chains, the matrix $P$ is
   stochastic, so $\lm_4=1$ is another eigenvalue of $P$ \mks{2}.  To find a
   corresponding eigenvector, we row-reduce the matrix $P-I$:
   \[
   \bbm
   -2/3&0&0&1/12\\
   1/3&-1/4&0&0\\
   1/3&0&-1/4&0\\
   0&1/4&1/4&-1/12\\
   \ebm
   \to
   \bbm
   1&0&0&-1/8\\
   1/3&-1/4&0&0\\
   1/3&0&-1/4&0\\
   0&1/4&1/4&-1/12\\
   \ebm
   \to
   \bbm
   1&0&0&-1/8\\
   0&-1/4&0&1/24\\
   0&0&-1/4&1/24\\
   0&1/4&1/4&-1/12\\
   \ebm
   \to 
  \] \[
   \bbm
   1&0&0&-1/8\\
   0&1&0&-1/6\\
   0&0&-1/4&1/24\\
   0&1/4&1/4&-1/12\\
   \ebm
   \to
   \bbm
   1&0&0&-1/8\\
   0&1&0&-1/6\\
   0&0&1&-1/6\\
   0&1/4&1/4&-1/12\\
   \ebm
   \to
   \bbm
   1&0&0&-1/8\\
   0&1&0&-1/6\\
   0&0&1&-1/6\\
   0&0&0&0\\
   \ebm \mks{2}
   \]
   We deduce that the desired eigenvector has the form
   $u_4=\bbm a&b&c&d\ebm^T$ with $a-d/8=b-d/6=c-d/6=0$, so
   $u_4=\bbm d/8&d/6&d/6&d\ebm^T$ with $d$ arbitrary.  We choose $d=24$,
   giving $u_4=\bbm 3&4&4&24\ebm^T$. \mk
  \item[(d)] As $P$ is stochastic, the row vector
   $e=\bbm 1 & \dotsb & 1\ebm$ satisfies $eP=e$ and so $P^Te^T=e^T$.
   In other words, $e^T$ is an eigenvector of $P^T$, with eigenvalue
   $1$. \mks{2}
  \item[(e)] The stationary distribution \mk is the eigenvector of
   eigenvalue $1$ where the sum of the entries is $1$.  As the sum of
   the entries in $u_4$ is $35$, the stationary distribution is
   $u_4/35=\bbm 3/35 & 4/35 & 4/35 & 24/35\ebm^T$ \mk.  The long run
   average probability of being in state $3$ is the third entry in the
   stationary distribution, which is $4/35$ \mk.
 \end{itemize}
\end{solution}

\begin{problem}[2012-13]
 Consider the following Markov chain:
 \begin{center}
  \begin{tikzpicture}[scale=3,auto,shorten >= 1pt]
   \node (state1) at ( 0.0,1.0) [circle,draw] {$1$}; 
   \node (state2) at ( 1.0,1.0) [circle,draw] {$2$}; 
   \node (state3) at ( 1.0,0.0) [circle,draw] {$3$}; 
   \node (state4) at ( 0.0,0.0) [circle,draw] {$4$}; 
   \draw[->,loop left]  (state1) to node {$\tfrac{1}{4}$} (state1);
   \draw[->,loop right] (state2) to node {$\tfrac{1}{4}$} (state2);
   \draw[->,loop right] (state3) to node {$\tfrac{7}{8}$} (state3);
   \draw[->,loop left]  (state4) to node {$\tfrac{7}{8}$} (state4);
   \draw[->] (state1) to node {$\tfrac{3}{4}$} (state2);
   \draw[->] (state2) to node {$\tfrac{3}{4}$} (state3);
   \draw[->] (state3) to node {$\tfrac{1}{8}$} (state4);
   \draw[->] (state4) to node {$\tfrac{1}{8}$} (state1);
  \end{tikzpicture}
 \end{center}
 \begin{itemize}
  \item[(a)] Write down the transition matrix $P$. \mrks{2}
  \item[(b)] Consider the following vectors:
   \[ u_1 = \bbm 1 \\ -6 \\ 6 \\ -1 \ebm \qquad
      u_2 = \bbm 1 \\ 3 \\ -6 \\ 2 \ebm \qquad
      u_3 = \bbm 1 \\ 2 \\ -6 \\ 3 \ebm
   \]
   Show that these are all eigenvectors for $P$, and find the
   corresponding eigenvalues.\\  \mrks{6}

   \textbf{Note:} it is not necessary to calculate the characteristic
   polynomial.
  \item[(c)] Using the general theory of Markov chains, write down one
   more eigenvalue; then find a corresponding eigenvector.  \mrks{6}
  \item[(d)] Give an invertible matrix $U$ and a diagonal matrix $D$
   such that $P=UDU^{-1}$.  Explain how this can be used to calculate
   $P^n$. \mrks{3}
  \item[(e)] What is the long run average probability of being in
   state 1? \mrks{3}
 \end{itemize}
\end{problem}
\begin{solution}
 \begin{itemize}
  \item[(a)] $\displaystyle P=
    \bbm 1/4 & 0   & 0   & 1/8 \\
         3/4 & 1/4 & 0   & 0   \\
         0   & 3/4 & 7/8 & 0   \\
         0   & 0   & 1/8 & 7/8 
    \ebm = \frac{1}{8} 
    \bbm 2 & 0 & 0 & 1 \\
         6 & 2 & 0 & 0 \\
         0 & 6 & 7 & 0 \\
         0 & 0 & 1 & 7 
    \ebm 
    $. \mks{2}
  \item[(b)] We have 
   \begin{align*}
    Pu_1 &= \frac{1}{8}
     \bbm 2 & 0 & 0 & 1 \\
          6 & 2 & 0 & 0 \\
          0 & 6 & 7 & 0 \\
          0 & 0 & 1 & 7 
     \ebm 
     \bbm 1 \\ -6 \\ 6 \\ -1 \ebm 
     = \frac{1}{8} \bbm 1 \\ -6 \\ 6 \\ -1 \ebm 
     = \frac{1}{8} u_1 \mk \\
    Pu_2 &= \frac{1}{8}
     \bbm 2 & 0 & 0 & 1 \\
          6 & 2 & 0 & 0 \\
          0 & 6 & 7 & 0 \\
          0 & 0 & 1 & 7 
     \ebm 
     \bbm 1 \\ 3 \\ -6 \\ 2 \ebm 
     = \frac{1}{8} \bbm 4 \\ 12 \\ -24 \\ 8 \ebm 
     = \frac{1}{2} u_2 \mk \\
    Pu_3 &= \frac{1}{8}
     \bbm 2 & 0 & 0 & 1 \\
          6 & 2 & 0 & 0 \\
          0 & 6 & 7 & 0 \\
          0 & 0 & 1 & 7 
     \ebm 
     \bbm 1 \\ 2 \\ -6 \\ 3 \ebm 
     = \frac{1}{8} \bbm 5 \\ 10 \\ -30 \\ 15 \ebm 
     = \frac{5}{8} u_3. \mk
   \end{align*}
   This shows that $u_1$, $u_2$ and $u_3$ are eigenvectors for $P$,
   with eigenvalues $\lm_1=1/8$ and $\lm_2=1/2$ and $\lm_3=5/8$. \mks{3}
  \item[(c)] By the general theory of Markov chains, the matrix $P$ is
   stochastic, so $\lm_4=1$ is another eigenvalue of $P$ \mks{2}.  To find a
   corresponding eigenvector, we row-reduce the matrix $P-I$:
   \[
   \bbm
   -3/4&0&0&1/8\\
   3/4&-3/4&0&0\\
   0&3/4&-1/8&0\\
   0&0&1/8&-1/8\\
   \ebm
   \to
   \bbm
   -3/4&0&0&1/8\\
   1&-1&0&0\\
   0&3/4&-1/8&0\\
   0&0&1/8&-1/8\\
   \ebm
   \to
   \bbm
   0&-3/4&0&1/8\\
   1&-1&0&0\\
   0&3/4&-1/8&0\\
   0&0&1/8&-1/8\\
   \ebm
   \to
   \] \[
   \bbm
   0&1&0&-1/6\\
   1&-1&0&0\\
   0&3/4&-1/8&0\\
   0&0&1/8&-1/8\\
   \ebm
   \to
   \bbm
   0&1&0&-1/6\\
   1&0&0&-1/6\\
   0&0&-1/8&1/8\\
   0&0&1/8&-1/8\\
   \ebm
   \to
   \bbm
   0&1&0&-1/6\\
   1&0&0&-1/6\\
   0&0&-1/8&1/8\\
   0&0&0&0\\
   \ebm
   \to
  \] \[
   \bbm
   0&1&0&-1/6\\
   1&0&0&-1/6\\
   0&0&1&-1\\
   0&0&0&0\\
   \ebm
   \to
   \bbm
   1&0&0&-1/6\\
   0&1&0&-1/6\\
   0&0&1&-1\\
   0&0&0&0\\
   \ebm \mks{3}
   \]
   We deduce that the desired eigenvector has the form
   $u_4=\bbm a&b&c&d\ebm^T$ with $a-d/6=b-d/6=c-d=0$, so
   $u_4=\bbm d/6&d/6&d&d\ebm^T$ with $d$ arbitrary.  We choose $d=6$,
   giving $u_4=\bbm 1&1&6&6\ebm^T$. \mk
  \item[(d)] We now take
   \begin{align*}
    U &= [u_1|u_2|u_3|u_4] 
       = \bbm 
           1 &  1 &  1 &  1 \\ 
          -6 &  3 &  2 &  1 \\
           6 & -6 & -6 &  6 \\
          -1 &  2 &  3 &  6
         \ebm \mk \\
    D &= \bbm 
          \lm_1 & 0 & 0 & 0 \\
          0 & \lm_2 & 0 & 0 \\
          0 & 0 & \lm_3 & 0 \\
          0 & 0 & 0 & \lm_4 
         \ebm = 
         \bbm
          1/8 & 0 & 0 & 0 \\
          0 & 1/2 & 0 & 0 \\
          0 & 0 & 5/8 & 0 \\
          0 & 0 & 0 & 1 
         \ebm. \mk
   \end{align*}
   The general theory tells us that $P=UDU^{-1}$.  It follows that 
   $P^n=UD^nU^{-1}$, where 
   \[ D^n = 
         \bbm
          (1/8)^n & 0 & 0 & 0 \\
          0 & (1/2)^n & 0 & 0 \\
          0 & 0 & (5/8)^n & 0 \\
          0 & 0 & 0 & 1 
         \ebm. \mk
   \]   
  \item[(e)] The stationary distribution \mk is the eigenvector of
   eigenvalue $1$ where the sum of the entries is $1$.  As the sum of
   the entries in $u_4$ is $14$, the stationary distribution is
   $u_4/14=\bbm 1/14 & 1/14 & 3/7 & 3/7\ebm^T$ \mk.  The long run average
   probability of being in state $1$ is the first entry in the
   stationary distribution, which is $1/14$ \mk.
 \end{itemize}
\end{solution}


\begin{problem}[2011-12 resit]
 Consider the following Markov chain:
 \begin{center}
  \begin{tikzpicture}[scale=2,auto,shorten >= 1pt]
   \node (state1) at ( 0.0,0.0) [circle,draw] {$1$}; 
   \node (state2) at ( 1.0,0.0) [circle,draw] {$2$}; 
   \node (state3) at ( 2.0,0.0) [circle,draw] {$3$}; 
   \node (state4) at ( 3.0,0.0) [circle,draw] {$4$}; 
   \draw[->,loop above] (state1) to node {$\tfrac{1}{5}$} (state1);
   \draw[->,loop above] (state2) to node {$\tfrac{2}{5}$} (state2);
   \draw[->,loop above] (state3) to node {$\tfrac{3}{5}$} (state3);
   \draw[->,loop above] (state4) to node {$1$}            (state4);
   \draw[->]            (state1) to node {$\tfrac{4}{5}$} (state2);
   \draw[->]            (state2) to node {$\tfrac{3}{5}$} (state3);
   \draw[->]            (state3) to node {$\tfrac{2}{5}$} (state4);
  \end{tikzpicture}
 \end{center}
 \begin{itemize}
  \item[(a)] Write down the associated transition matrix $P$. \mrks{2}
  \item[(b)] Show that the following vectors are eigenvectors for $P$,
   and find the corresponding eigenvalues. \mrks{4}
   \[ u_3 = \bbm 0 \\ 0 \\ 1 \\ -1 \ebm \hspace{5em}
      u_4 = \bbm 0 \\ 0 \\ 0 \\ 1 \ebm
   \]
  \item[(c)] Find all the other eigenvalues and eigenvectors. \mrks{9}
  \item[(d)] At time $t=0$ the system is in state $2$.  What is the
   probability that it is in state $3$ at $t=8$? \mrks{10}
 \end{itemize}
\end{problem}
\begin{solution}
 \begin{itemize}
  \item[(a)]
   $P = \bbm
         1/5 &   0 &   0 & 0 \\
         4/5 & 2/5 &   0 & 0 \\
           0 & 3/5 & 3/5 & 0 \\
           0 &   0 & 2/5 & 1 
        \ebm$
   \mks{2}
  \item[(b)] We have
   \begin{align*}
    Pu_3 &= 
        \bbm
         1/5 &   0 &   0 & 0 \\
         4/5 & 2/5 &   0 & 0 \\
           0 & 3/5 & 3/5 & 0 \\
           0 &   0 & 2/5 & 1 
        \ebm
        \bbm 0 \\ 0 \\ 1 \\ -1 \ebm
        = 
        \bbm 0 \\ 0 \\ 3/5 \\ -3/5 \ebm = \tfrac{3}{5} u_3 \mk \\
    Pu_4 &= 
        \bbm
         1/5 &   0 &   0 & 0 \\
         4/5 & 2/5 &   0 & 0 \\
           0 & 3/5 & 3/5 & 0 \\
           0 &   0 & 2/5 & 1 
        \ebm
        \bbm 0 \\ 0 \\ 0 \\ 1 \ebm
        = 
        \bbm 0 \\ 0 \\ 0 \\ 1 \ebm = u_4 \mk.
   \end{align*}
   This shows that $u_3$ and $u_4$ are eigenvectors, with eigenvalues
   $\lm_3=3/5$ and $\lm_4=1$ \mks{2}.
  \item[(c)] The characteristic polynomial is
   \[ \chi_P(t) =
        \det\bbm
         1/5-t &   0 &   0 & 0 \\
         4/5 & 2/5-t &   0 & 0 \\
           0 & 3/5 & 3/5-t & 0 \\
           0 &   0 & 2/5 & 1-t 
        \ebm
        = 
        (1/5-t)(2/5-t)(3/5-t)(1-t)
   \]
   (because the determinant of a lower-triangular matrix is just the
   product of the diagonal entries) \mks{2}.  It follows that the eigenvalues
   are $\lm_1=1/5$ and $\lm_2=2/5$ and $\lm_3=3/5$ and $\lm_4=1$ \mk.  We
   already have eigenvectors $u_3$ and $u_4$ of eigenvalues $\lm_3$
   and $\lm_4$.  To find an eigenvector $u_1$ of eigenvalue $1/5$, we
   must solve the matrix equation shown on the left below, or the
   equivalent system of equations shown on the right.
   \[ 
        \bbm
         1/5 &   0 &   0 & 0 \\
         4/5 & 2/5 &   0 & 0 \\
           0 & 3/5 & 3/5 & 0 \\
           0 &   0 & 2/5 & 1 
        \ebm
        \bbm w \\ x \\ y \\ z \ebm
        = 
        \bbm w/5 \\ x/5 \\ y/5 \\ z/5 \ebm 
        \hspace{5em}
        \begin{array}{rl}
         w &= w \\
         4w+2x &= x \\
         3x+3y &= y \\
         2y+5z &= z.
        \end{array}
   \]
   It is clear that the vector $u_1=\bbm 1&-4&6&-3\ebm^T$ is a
   solution. \mks{3}

   Similarly, to find an eigenvector $u_1$ of eigenvalue $2/5$, we
   must solve the matrix equation shown on the left below, or the
   equivalent system of equations shown on the right.
   \[ 
        \bbm
         1/5 &   0 &   0 & 0 \\
         4/5 & 2/5 &   0 & 0 \\
           0 & 3/5 & 3/5 & 0 \\
           0 &   0 & 2/5 & 1 
        \ebm
        \bbm w \\ x \\ y \\ z \ebm
        = 
        \bbm 2w/5 \\ 2x/5 \\ 2y/5 \\ 2z/5 \ebm 
        \hspace{5em}
        \begin{array}{rl}
         w &= 2w \\
         4w+2x &= 2x \\
         3x+3y &= 2y \\
         2y+5z &= 2z.
        \end{array}
   \]
   The first equation gives $w=0$, we can take $x$ to be $1$, and then
   the remaining equations give $y=-3$ and $z=2$.  Thus, the vector
   $u_2=\bbm 0&1&-3&2\ebm^T$ is a solution.  \mks{3}
  \item[(d)] We are told that the initial distribution is 
   $r_0=\bbm 0&1&0&0\ebm^T$.  We need to express this as a linear
   combination of the vectors $u_i$, say
   $\al_1u_1+\al_2u_2+\al_3u_3+\al_4u_4=r_0$ \mk.  This reduces to 
   \begin{align*}
      \al_1 &= 0 \\
    -4\al_1 +  \al_2 &= 1 \\
     6\al_1 - 3\al_2 + \al_3 &= 0 \\
    -3\al_1 + 2\al_2 - \al_3 + \al_4 &= 0 \mk.
   \end{align*}
   These equations can be solved in an obvious way to give $\al_1=0$
   and $\al_2=1$ and $\al_3=3$ and $\al_4=1$, so 
   \[ r_0 = u_2 + 3u_3 + u_4 \mks{2}. \]
   It follows that 
   \begin{align*}
    r_n &= P^n r_0 \mk
         = P^nu_2 + 3 P^nu_3 + P^nu_4 \\
        &= (2/5)^nu_2 + 3(3/5)^nu_3 + u_4 \mk
         = \bbm 
            0 \\
            (2/5)^n \\
            -3(2/5)^n + 3(3/5)^n  \\
            2(2/5)^n - 3(3/5)^n + 1
           \ebm \mks{2}
   \end{align*}
   The probability of being in state $3$ at time $n$ is the third
   entry in this vector \mk.  Taking $n=8$ we get 
   \[ 3((3/5)^8 -(2/5)^8) = \frac{3783}{78125} = 0.0484 \mk. \]
 \end{itemize}
\end{solution}


\begin{problem}[2011-12]
 Consider the following Markov chain:
 \begin{center}
  \begin{tikzpicture}[scale=2,auto,shorten >= 1pt]
   \node (state1) at ( 0.0,1.5) [circle,draw] {$1$}; 
   \node (state2) at (-1.0,0.0) [circle,draw] {$2$}; 
   \node (state3) at ( 1.0,0.0) [circle,draw] {$3$}; 
   \draw[->,bend right] (state1) to node [swap] {$0.6$} (state2);
   \draw[->,bend left]  (state1) to node {$0.2$} (state3);
   \draw[->,loop above] (state1) to node {$0.2$} (state1);
   \draw[->,bend left]  (state2) to node {$1.0$} (state3);
   \draw[->,bend left]  (state3) to node {$1.0$} (state2);
  \end{tikzpicture}
 \end{center}
 \begin{itemize}
  \item[(a)] Write down the associated transition matrix. \mrks{2}
  \item[(b)] Find a stationary distribution for the system. \mrks{6}
  \item[(c)] If the system is in state $1$ at $t=0$, what is the
   probability that it is in state $2$ at $t=4$? \mrks{17}
 \end{itemize}
\end{problem}
\begin{solution}
 \begin{itemize}
  \item[(a)] $P = \bbm 0.2 & 0 & 0 \\ 0.6 & 0 & 1 \\ 0.2 & 1 & 0 \ebm$. 
   \mks{2}
  \item[(b)] A stationary distribution $p$ is in particular an
   eigenvector of eigenvalue $1$.  Such eigenvectors can be found by
   row-reducing $P-I$ \mk:
   \[ \bbm -0.8 & 0 & 0 \\ 0.6 & -1 & 1 \\ 0.2 & 1 & -1 \ebm
      \to 
      \bbm 1 & 0 & 0 \\ 0 & -1 & 1 \\ 0 & 1 & -1 \ebm
      \to 
      \bbm 1 & 0 & 0 \\ 0 & 1 & -1 \\ 0 & 0 & 0 \ebm \mks{2}
   \]
   We conclude that if $p=\bbm p_1&p_2&p_3\ebm^T$ then we must have 
   \[ \bbm 1 & 0 & 0 \\ 0 & 1 & -1 \\ 0 & 0 & 0 \ebm
       \bbm p_1\\ p_2\\ p_3\ebm = \bbm 0\\0\\0 \ebm,
   \]
   or equivalently $p_1=0$ and $p_2=p_3$ \mk.  For a stationary
   distribution we also need $p_1,p_2,p_3\geq 0$ and $p_1+p_2+p_3=1$,
   so we must take $p_2=p_3=0.5$ \mk.  It follows that the stationary
   distribution is $\bbm 0&0.5&0.5\ebm^T$ \mk.
  \item[(c)] We are given that the initial distribution is
   $r_0=\bbm 1\\0\\0\ebm^T$, and we need to calculate $r_4=P^4r_0$.
   As $4$ is not very large, it is easy enough to do this directly:
   \begin{align*}
    P^2 &= \bbm -0.8 & 0 & 0 \\ 0.6 & -1 & 1 \\ 0.2 & 1 & -1 \ebm
           \bbm -0.8 & 0 & 0 \\ 0.6 & -1 & 1 \\ 0.2 & 1 & -1 \ebm
         = \bbm 0.04 & 0 & 0 \\ 0.32 & 1 & 0 \\ 0.64 & 0 & 1 \ebm \\
    P^4 &= (P^2)^2 
         = \bbm 0.04 & 0 & 0 \\ 0.32 & 1 & 0 \\ 0.64 & 0 & 1 \ebm
           \bbm 0.04 & 0 & 0 \\ 0.32 & 1 & 0 \\ 0.64 & 0 & 1 \ebm 
         = \bbm 0.0016 & 0 & 0 \\ 0.3328 & 1 & 0 \\ 0.6656 & 0 & 1 \ebm \\
    r_4 &= P^4r_0 
         = \bbm 0.0016 & 0 & 0 \\ 0.3328 & 1 & 0 \\ 0.6656 & 0 & 1 \ebm 
           \bbm 1 \\ 0 \\ 0 \ebm 
         = \bbm 0.0016 \\ 0.3328 \\ 0.6656 \ebm 
   \end{align*}
   The probability of being in state $2$ at $t=4$ is the second
   component of $r_4$, which is $0.3328$.  Full credit will be given
   for this approach.

   However, most students will probably use the following method.  We
   first find the remaining eigenvalues and eigenvectors for $P$ \mk.  The
   characteristic polynomial is
   \[ \chi_P(t) = 
      \det\bbm 0.2-t & 0 & 0 \\ 0.6 & -t & 1 \\ 0.2 & 1 & -t \ebm
      = (0.2-t)(t^2-1) \mk = (t-1)(t+1)(0.2-t),
   \]
   so the eigenvalues are $1$ and $-1$ and $0.2$ \mk.  We have already
   found an eigenvector $u_1=\bbm 0&0.5&0.5\ebm^T$ of eigenvalue $1$.
   To find an eigenvector of eigenvalue $-1$, we row-reduce $P+I$:
   \[ 
    \bbm 1.2 & 0 & 0 \\ 0.6 & 1 & 1 \\ 0.2 & 1 & 1 \ebm
    \to 
    \bbm 1 & 0 & 0 \\ 0 & 1 & 1 \\ 0 & 1 & 1 \ebm
    \to 
    \bbm 1 & 0 & 0 \\ 0 & 1 & 1 \\ 0 & 0 & 0 \ebm \mk
   \]
   This shows that the eigenvectors of eigenvalue $-1$ are the vectors
   of the form $\bbm x&y&z\ebm$ with $x=y+z=0$, or in other words the
   vectors of the form $\bbm 0&y&-y\ebm$ \mk.  For compatibility with
   $u_1$ it will be convenient to take $y=0.5$ giving
   $u_2=\bbm 0&0.5&-0.5\ebm$ \mk.  Next, to find an eigenvector of
   eigenvalue $0.2$ we row-reduce $P-0.2I$:
   \[
    \bbm 0 & 0 & 0 \\ 0.6 & -0.2 & 1 \\ 0.2 & 1 & -0.2 \ebm
    \to
    \bbm 1 & 5 & -1 \\ 0.6 & -0.2 & 1 \\ 0 & 0 & 0 \ebm
    \to
    \bbm 1 & 5 & -1 \\ 0 & -3.2 & 1.6 \\ 0 & 0 & 0 \ebm
    \to
    \bbm 1 & 5 & -1 \\ 0 & 1 & -0.5 \\ 0 & 0 & 0 \ebm
    \to
    \bbm 1 & 0 & 1.5 \\ 0 & 1 & -0.5 \\ 0 & 0 & 0 \ebm \mks{2}
   \]
   From this we see that a suitable eigenvector is
   $u_3=\bbm -1.5&0.5&1\ebm$ \mk.  We next need to write the vector
   $r_0=\bbm 1&0&0\ebm^T$ as a linear combination of the eigenvectors
   $u_i$.  We can do this by row-reducing the matrix
   $[u_1|u_2|u_3|r_0]$: \mk 
   \[ 
    \bbm
     0   &  0   & -1.5 & 1 \\ 
     0.5 &  0.5 &  0.5 & 0 \\
     0.5 & -0.5 &  1   & 0 
    \ebm
    \to
    \bbm
     0   &  0   & -1.5 & 1 \\ 
     0.5 &  0.5 &  0.5 & 0 \\
     0   & -1   &  0.5 & 0 
    \ebm
    \to
    \bbm
     1   &  1   &  1   & 0 \\
     0   &  1   & -0.5 & 0 \\
     0   &  0   &  1   & -2/3  
    \ebm
   \] \[
    \to
    \bbm
     1   &  1   &  0   & 2/3 \\
     0   &  1   &  0   & -1/3 \\
     0   &  0   &  1   & -2/3  
    \ebm
    \to
    \bbm
     1   &  0   &  0   & 1 \\
     0   &  1   &  0   & -1/3 \\
     0   &  0   &  1   & -2/3  
    \ebm \mks{2}
   \]
   The required coefficients appear in the last column, so
   $r_0=u_1-\tfrac{1}{3}u_2-\tfrac{2}{3}u_3$ \mk.  This gives
   \begin{align*}
     r_4 &= P^4r_0
          = P^4u_1-\tfrac{1}{3}P^4u_2-\tfrac{2}{3}P^4u_3 \mk
          = 1^4u_1-\tfrac{1}{3}(-1)^4u_2-\tfrac{2}{3}(0.2)^4u_3 \mk \\
         &= u_1 - \tfrac{1}{3}u_2 -\tfrac{2}{3}(0.2)^4u_3 
          = \bbm 0\\ 1/2 \\ 1/2 \ebm +
            \bbm 0\\-1/6\\ 1/6 \ebm +
            (0.2)^4 \bbm 1\\-1/3\\-2/3 \ebm
          = \bbm 0.0016 \\ 0.3328 \\ 0.6656 \ebm \mk
   \end{align*}
   Again, the probability of being in state $2$ at $t=4$ is the second
   component of $r_4$, which is $0.3328$. \mk
 \end{itemize}
\end{solution}

\begin{problem}[Mock 1]
 Consider the following Markov chain:
 \begin{center}
  \begin{tikzpicture}[scale=2,auto,shorten >= 1pt]
   \node (state1) at ( 0.0,1.5) [circle,draw] {$1$}; 
   \node (state2) at (-1.0,0.0) [circle,draw] {$2$}; 
   \node (state3) at ( 1.0,0.0) [circle,draw] {$3$}; 
   \draw[->,loop above] (state1) to node {$\tfrac{1}{2}$} (state1);
   \draw[->,bend left]  (state1) to node {$\tfrac{1}{2}$} (state3);
   \draw[->,bend left]  (state2) to node {$\tfrac{1}{4}$} (state1);
   \draw[->]            (state2) to node {$\tfrac{3}{4}$} (state3);
   \draw[->]            (state3) to node {$\tfrac{1}{4}$} (state1);
   \draw[->,bend left]  (state3) to node {$\tfrac{3}{4}$} (state2);
  \end{tikzpicture}
 \end{center}
 \begin{itemize}
  \item[(a)] Write down the associated transition matrix $P$. \mrks{2}
  \item[(b)] You may assume that the eigenvalues of $P$ are $\lm_1=1$
   and $\lm_2=-3/4$ and $\lm_3=1/4$.  Find corresponding eigenvectors
   $u_1$, $u_2$ and $u_3$.  \mrks{9}
  \item[(c)] Find a stationary distribution for the system. \mrks{2}
  \item[(d)] Show that the vector $v=\bbm 1&1&1\ebm^T$ can be
   expressed as a linear combination of $u_1$ and $u_2$. \mrks{4}
  \item[(e)] At time $t=0$ the system has an equal probability of
   being in any of the three states.  What is the probability of 
   being in state $1$ at time $t=10$? \mrks{8}
 \end{itemize}
\end{problem}
\begin{solution}
 \begin{itemize}
  \item[(a)]
   $P = \bbm 1/2 & 1/4 & 1/4 \\ 0 & 0 & 3/4 \\ 1/2 & 3/4 & 0 \ebm
      = \frac{1}{4} \bbm 2&1&1\\0&0&3\\2&3&0\ebm$. 
   \mks{2}
  \item[(b)] To find an eigenvector of eigenvalue $1$ we row-reduce $P-I$:
   \[
    \bbm
    -1/2&1/4&1/4\\
    0&-1&3/4\\
    1/2&3/4&-1\\
    \ebm
    \to
    \bbm
    1&-1/2&-1/2\\
    0&1&-3/4\\
    1/2&3/4&-1\\
    \ebm
    \to
    \bbm
    1&-1/2&-1/2\\
    0&1&-3/4\\
    0&1&-3/4\\
    \ebm
    \to
    \bbm
    1&0&-7/8\\
    0&1&-3/4\\
    0&0&0\\
    \ebm \mks{2}
   \]
   Any eigenvector $u_1=\bbm x&y&z\ebm^T$ of eigenvalue $1$ must
   therefore satisfy $x-(7/8)z=y-(3/4)z=0$, so
   $u_1=\bbm (7/8)z & (3/4)z & z \ebm^T$ with $z$ arbitrary \mk.  It
   will be convenient to take $z=8$ giving $u_1=\bbm 7&6&8\ebm^T$.

   To find an eigenvector $u_2$ of eigenvalue $-3/4$ we row-reduce
   $P+\frac{3}{4}I$: 
   \[
    \bbm
    5/4&1/4&1/4\\
    0&3/4&3/4\\
    1/2&3/4&3/4\\
    \ebm
    \to
    \bbm
    1&1/5&1/5\\
    0&1&1\\
    1/2&3/4&3/4\\
    \ebm
    \to
    \bbm
    1&1/5&1/5\\
    0&1&1\\
    0&13/20&13/20\\
    \ebm
    \to
    \bbm
    1&0&0\\
    0&1&1\\
    0&0&0\\
    \ebm \mks{2}
   \]
   From this we see that we can take $u_2=\bbm 0&1&-1\ebm^T$ \mk.
   Similarly, to find an eigenvector $u_2$ of eigenvalue $-3/4$ we
   row-reduce $P-\frac{1}{4}I$:
   \[
    \bbm
    1/4&1/4&1/4\\
    0&-1/4&3/4\\
    1/2&3/4&-1/4\\
    \ebm
    \to
    \bbm
    1&1&1\\
    0&1&-3\\
    1/2&3/4&-1/4\\
    \ebm
    \to
    \bbm
    1&1&1\\
    0&1&-3\\
    0&1/4&-3/4\\
    \ebm
    \to
    \bbm
    1&0&4\\
    0&1&-3\\
    0&0&0\\
    \ebm \mks{2}
   \]
   From this we see that we can take $u_3=\bbm -4&3&1\ebm^T$ \mk.
  \item[(c)] A stationary distribution $q$ must be an eigenvector of
   eigenvalue $1$, so it must be a multiple of $u_1$ \mk, say
   $q=\bbm 7t&6t&8t\ebm^T$.  For a probability vector the sum of the
   entries must be equal to one, which means that $21t=1$ so $t=1/21$
   and $q=\bbm 1/3&2/7&8/21\ebm^T$ \mk.
  \item[(d)] By inspection we have $u_1+u_2=\bbm 7&7&7\ebm^T$ so
   $v=(u_1+u_2)/7$.  For a more systematic approach, we need to find
   $\al_1,\al_2$ and $\al_3$ such that $\al_1u_1+\al_2u_2+\al_3u_3=v$, or
   equivalently
   \[ \al_1 \bbm 7\\6\\8\ebm +
      \al_2 \bbm 0\\1\\-1\ebm +
      \al_3 \bbm -4\\3\\1\ebm
       = \bbm 1\\1\\1\ebm \hspace{5em}
      \begin{array}{rl}
       7\al_1-4\al_3 &= 1 \\
       6\al_1+\al_2+3\al_3 &= 1 \\
       8\al_1-\al_2+\al_3 &= 1
      \end{array}
      \qquad \mks{2}
   \]
   These equations are easily solved to give $\al_1=\al_2=1/7$ and $\al_3=0$ so 
   $v=(u_1+u_2)/7$ as before \mks{2}.
  \item[(e)] We are told that at $t=0$ the three states all have equal
   probability, which must be $1/3$, so the initial distribution is 
   \[ r_0 = \bbm 1/3 \\ 1/3 \\ 1/3 \ebm
          = \frac{v}{3} = \frac{u_1+u_2}{21}. \mks{2}
   \]
   It follows that 
   \[ r_{10} = P^{10}r_0 \mk =
         \tfrac{1}{21}(P^{10}u_1+P^{10}u_2) = 
         \tfrac{1}{21}\left(u_1+(-\tfrac{3}{4})^{10}u_2\right) \mks{2} = 
         \bbm 7/21 \\ 6/21 + (-3/4)^{10}/21 \\ 8/21 - (-3/4)^{10}/21 \ebm
         \mk.
   \]
   The probability of being in state $1$ at $t=10$ is the first entry
   in $r_{10}$, which is just $7/21=1/3$ \mks{2}.
 \end{itemize}
\end{solution}

\begin{problem}[Mock 2]
 Consider the following Markov chain:
 \begin{center}
  \begin{tikzpicture}[scale=2,auto,shorten >= 1pt]
   \node (state1) at ( 0.0,0.0) [circle,draw] {$1$}; 
   \node (state2) at ( 1.0,0.0) [circle,draw] {$2$}; 
   \node (state3) at ( 2.0,0.0) [circle,draw] {$3$}; 
   \node (state4) at ( 3.0,0.0) [circle,draw] {$4$}; 
   \draw[->,loop above] (state1) to node {$\tfrac{1}{4}$} (state1);
   \draw[->,loop above] (state2) to node {$\tfrac{1}{3}$} (state2);
   \draw[->,loop above] (state3) to node {$\tfrac{1}{2}$} (state3);
   \draw[->,loop above] (state4) to node {$1$}            (state4);
   \draw[->]            (state1) to node {$\tfrac{3}{4}$} (state2);
   \draw[->]            (state2) to node {$\tfrac{2}{3}$} (state3);
   \draw[->]            (state3) to node {$\tfrac{1}{2}$} (state4);
  \end{tikzpicture}
 \end{center}
 \begin{itemize}
  \item[(a)] Write down the associated transition matrix $P$. \mrks{2}
  \item[(b)] Show that the following vectors are eigenvectors for $P$,
   and find the corresponding eigenvalues. \mrks{4}
   \[ u_1 = \bbm 1 \\ -9 \\ 24 \\ -16 \ebm \hspace{5em}
      u_2 = \bbm 0 \\ 1 \\ -4 \\ 3 \ebm
   \]
  \item[(c)] Find all the other eigenvalues and eigenvectors. \mrks{9}
  \item[(e)] At time $t=0$ the system is in state $1$.  What is the
   probability that it is in state $4$ at $t=5$? \mrks{10}
 \end{itemize}
\end{problem}
\begin{solution}
 \begin{itemize}
  \item[(a)]
   $P = \bbm
         1/4 &   0 &   0 & 0 \\
         3/4 & 1/3 &   0 & 0 \\
           0 & 2/3 & 1/2 & 0 \\
           0 &   0 & 1/2 & 1 
        \ebm$
   \mks{2}
  \item[(b)] We have
   \begin{align*}
    Pu_1 &= 
        \bbm
         1/4 &   0 &   0 & 0 \\
         3/4 & 1/3 &   0 & 0 \\
           0 & 2/3 & 1/2 & 0 \\
           0 &   0 & 1/2 & 1 
        \ebm
        \bbm 1 \\ -9 \\ 24 \\ -16 \ebm
        = 
        \bbm 1/4 \\ -9/4 \\ 6 \\ -4 \ebm = \tfrac{1}{4} u_1 \\
    Pu_2 &= 
        \bbm
         1/4 &   0 &   0 & 0 \\
         3/4 & 1/3 &   0 & 0 \\
           0 & 2/3 & 1/2 & 0 \\
           0 &   0 & 1/2 & 1 
        \ebm
        \bbm 0 \\ 1 \\ -4 \\ 3 \ebm
        =
        \bbm 0 \\ 1/3 \\ -4/3 \\ 1 \ebm = \tfrac{1}{3} u_2 \mks{2}.
   \end{align*}
   This shows that $u_1$ and $u_2$ are eigenvectors, with eigenvalues
   $\lm_1=1/4$ and $\lm_2=1/3$ \mks{2}.
  \item[(c)] The characteristic polynomial is
   \[ \chi_P(t) =
        \det\bbm
         1/4-t &   0 &   0 & 0 \\
         3/4 & 1/3-t &   0 & 0 \\
           0 & 2/3 & 1/2-t & 0 \\
           0 &   0 & 1/2 & 1-t 
        \ebm
        = 
        (1/4-t)(1/3-t)(1/2-t)(1-t)
   \]
   (because the determinant of a lower-triangular matrix is just the
   product of the diagonal entries) \mks{2}.  It follows that the eigenvalues
   are $\lm_1=1/4$ and $\lm_2=1/3$ and $\lm_3=1/2$ and $\lm_4=1$ \mk.  We
   already have eigenvectors $u_1$ and $u_2$ of eigenvalues $\lm_1$
   and $\lm_2$.  To find an eigenvector $u_3$ of eigenvalue $1/2$, we
   row-reduce $P-\tfrac{1}{2}I$:
   \[
    \bbm
    -1/4&0&0&0\\
    3/4&-1/6&0&0\\
    0&2/3&0&0\\
    0&0&1/2&1/2\\
    \ebm
    \to
    \bbm
    1&0&0&0\\
    0&-1/6&0&0\\
    0&2/3&0&0\\
    0&0&1/2&1/2\\
    \ebm
    \to
    \bbm
    1&0&0&0\\
    0&1&0&0\\
    0&0&0&0\\
    0&0&1/2&1/2\\
    \ebm
    \to
    \bbm
    1&0&0&0\\
    0&1&0&0\\
    0&0&1&1\\
    0&0&0&0\\
    \ebm \mks{2}
   \]
   We conclude that if $u_3=\bbm x&x&y&z\ebm^T$ then we must have
   $w=x=y+z=0$.  We can take $y=1$ and so $u_3=\bbm 0&0&1&-1\ebm$ \mks{2}.
   
   One can also see by inspection that the vector
   $u_4=\bbm 0&0&0&1\ebm^T$ is an eigenvector of eigenvalue $1$ \mks{2}.
   (This is essentially the observation that the system will almost
   surely reach state $4$, and then it will stay there.) This could
   also be obtained by row-reducing $P-I$, but that would be more work
   than necessary.  The steps are as follows:
   \[
    \bbm
    -3/4&0&0&0\\
    3/4&-2/3&0&0\\
    0&2/3&-1/2&0\\
    0&0&1/2&0\\
    \ebm
    \to
    \bbm
    -3/4&0&0&0\\
    0&-2/3&0&0\\
    0&2/3&-1/2&0\\
    0&0&1/2&0\\
    \ebm
    \to
    \bbm
    -3/4&0&0&0\\
    0&-2/3&0&0\\
    0&0&-1/2&0\\
    0&0&1/2&0\\
    \ebm
   \] \[
    \to
    \bbm
    -3/4&0&0&0\\
    0&-2/3&0&0\\
    0&0&-1/2&0\\
    0&0&0&0\\
    \ebm
    \to
    \bbm
    1&0&0&0\\
    0&1&0&0\\
    0&0&1&0\\
    0&0&0&0\\
    \ebm
   \]
  \item[(e)] We are told that the initial distribution is 
   $r_0=\bbm 1&0&0&0\ebm^T$.  We need to express this as a linear
   combination of the vectors $u_i$, say
   $\al_1u_1+\al_2u_2+\al_3u_3+\al_4u_4=r_0$ \mk.  This reduces to 
   \begin{align*}
    \al_1 &= 1 \\
    -9\al_1 + \al_2 &= 0 \\
    24\al_1 - 4\al_2 + \al_3 &= 0 \\
    -16\al_1 + 3\al_2 - \al_3 + \al_4 &= 0 \mk.
   \end{align*}
   These equations can be solved in an obvious way to give $\al_1=1$
   and $\al_2=9$ and $\al_3=12$ and $\al_4=1$, so 
   \[ r_0 = u_1 + 9u_2 + 12 u_3 + u_4 \mks{2}. \]
   It follows that 
   \begin{align*}
    r_n &= P^n r_0 \mk
         = P^nu_1 + 9P^nu_2 + 12 P^nu_3 + P^nu_4 \\
        &= (1/4)^nu_1 + 9(1/3)^nu_2 + 12(1/2)^nu_3 + u_4 \mk
         = \bbm 
            (1/4)^n \\
            -9\tm (1/4)^n + 9\tm (1/3)^n \\
            24\tm (1/4)^n - 36\tm(1/3)^n + 12\tm (1/2)^n \\
            -16\tm(1/4)^n + 27\tm(1/3)^n - 12\tm (1/2)^n + 1.
           \ebm \mks{2}
   \end{align*}
   The probability of being in state $4$ at time $n$ is the bottom
   entry in this vector.  taking $n=5$ we get 
   \[ -16/4^5 +27/3^5 - 12/2^5 + 1 = 415/576 \simeq 0.7205 \mks{2}. \]
 \end{itemize}
\end{solution}

%%%%%%%%%%%%%%%%%%%%%%%%%%%%%%%%%%%%%%%%%%%%%%%%%%%%%%%%%%%%%%%%%%%%%%

\section{Question 3}
\setcounter{probcounter}{0}

\begin{problem}[2013-14]
 \begin{itemize}
  \item[(a)] Are the following statements true or false?  Justify your
   answers. \mrks{10}
   \begin{itemize}
    \item[(i)] There are subspaces $V,W\leq\R^6$ with
     $\dim(V)=\dim(W)=4$ and $\dim(V\cap W)=1$.
    \item[(ii)] There are subspaces $V,W\leq\R^6$ with
     $\dim(V)=\dim(W)=4$ and $\dim(V\cap W)=2$.
    \item[(iii)] The following list is linearly independent:
     \[ \bbm 1\\1\\ 1\\  1 \ebm,\quad
        \bbm 1\\2\\ 4\\  8 \ebm,\quad
        \bbm 1\\3\\ 9\\ 27 \ebm,\quad
        \bbm 1\\4\\16\\ 81 \ebm,\quad
        \bbm 1\\5\\25\\125 \ebm.
     \]
    \item[(iv)] If $A$ is a $3\tm 3$ matrix with only $2$ distinct
     eigenvalues, then it cannot be diagonalised.
    \item[(v)] There is a $4\tm 4$ symmetric matrix with
     characteristic polynomial $t^4+1$.
   \end{itemize}
  \item[(b)] Give examples of the following: \mrks{10}
   \begin{itemize}
    \item[(i)] A list $u_1,\dotsc,u_4$ of vectors in $\R^2$ such that
     $u_1,u_2$ is a basis and $u_2,u_3$ is a basis and $u_3,u_4$ is a
     basis but $u_4,u_1$ is not a basis.
    \item[(ii)] A pair of subspaces $V,W\leq\R^4$ such that
     $\dim(V)=\dim(W)=2$ and 
     \[ V\cap W = \{\bbm w&x&y&z\ebm^T \st w+x=x+y=y+z=0\}. \]
    \item[(iii)] A non-diagonalisable $3\tm 3$ matrix whose only
     eigenvalue is $111$.
    \item[(iv)] A stochastic matrix with eigenvalues $1$, $1/2$ and
     $1/3$. 
    \item[(v)] A $3\tm 2$ matrix of rank $1$ that is not in RREF.
   \end{itemize}
 \end{itemize}
\end{problem}
\begin{solution}
 \begin{itemize}
  \item[(a)]
   \begin{itemize}
    \item[(i)] This is false \mk.  The dimension formula says that
     $\dim(V+W)=\dim(V)+\dim(W)-\dim(V\cap W)$, which would give
     $\dim(V+W)=4+4-1=7$, but it is impossible for a subspace of
     $\R^6$ to have dimension $7$. \mk
    \item[(ii)] This is true. \mk As an example, we could have
     $V=\spn(e_1,e_2,e_3,e_4)$ and $W=\spn(e_1,e_2,e_5,e_6)$ so
     $V\cap W=\spn(e_1,e_2)$. \mk
    \item[(iii)] This is false. \mk No list of $5$ vectors in $\R^4$
     can be linearly independent. \mk
    \item[(iv)] This is false. \mk The matrix 
     \[ A = \bbm 0&0&0 \\ 0&0&0 \\ 0&0&1 \ebm \]
     has only two eigenvalues (namely $0$ and $1$), but it is diagonal
     and so is certainly diagonalisable. \mk
    \item[(v)] This is false. \mk Any symmetric matrix has real
     eigenvalues, but the roots of $t^4+1$ are not real. \mk
   \end{itemize}
  \item[(b)] In each case, there are many possible correct answers.
   \begin{itemize}
    \item[(i)] Note that a list $v,w$ in $\R^2$ is a basis iff $v$ and
     $w$ are nonzero, and not multiples of each other.  Given this, we
     see that the list $e_1,e_2,e_1+e_2,e_1$ is as required. \mks{2}
    \item[(ii)] Put $u_1=\bbm 1&-1&1&-1\ebm$, so 
     \[ \{\bbm w&x&y&z\ebm^T \st w+x=x+y=y+z=0\} = \spn(u_1). \]
     Then the spaces $V=\spn(u_1,e_1)$ and $W=\spn(u_1,e_2)$ have
     dimension two with $V\cap W=\spn(u_1)$ as required. \mks{2}
    \item[(iii)] The standard example is the Jordan block
     $\displaystyle 
       \bbm 111 & 1 & 0 \\ 0 & 111 & 1 \\ 0 & 0 & 111 \ebm.
     $   \mks{2}
    \item[(iv)] The obvious example is
     $\displaystyle
        \bbm 1 & 1/2 & 1/3 \\ 0 & 1/2 & 1/3 \\ 0 & 0 & 1/3 \ebm.
     $ \mks{2}
    \item[(v)] One example is 
     $\displaystyle \bbm 0 & 0 \\ 0 & 0 \\ 1 & 0 \ebm $
     \mks{2}
   \end{itemize}
 \end{itemize}
\end{solution}

\begin{problem}[2012-13 resit]
 \begin{itemize}
  \item[(a)] Are the following statements true or false?  Justify your
   answers. \mrks{10}
   \begin{itemize}
    \item[(i)] If $V$ and $W$ are subspaces of $\R^n$, then
     $\dim(V+W)\geq\dim(V)+\dim(W)$.
    \item[(ii)] If the list $v_1,\dotsc,v_5$ spans $\R^4$, then it is
     also linearly independent.
    \item[(iii)] If $w_1$ cannot be expressed as a linear combination of
     $w_2$, $w_3$ and $w_4$, then the list $w_1,w_2,w_3,w_4$ is
     linearly independent.
    \item[(iv)] If $a_1,a_2,a_3,b\in\R^4$ and the matrix
     $[a_1|a_2|a_3|b]$ row-reduces to the identity matrix, then $b$ is
     a linear combination of $a_1$, $a_2$ and $a_3$.
    \item[(v)] If $M$ is a $3\tm 3$ symmetric matrix and $u$ and $v$
     are vectors in $\R^3$ with $Mu=2u$ and $Mv=3v$, then $u.v=0$.
   \end{itemize}
  \item[(b)] Give examples of the following: \mrks{10}
   \begin{itemize}
    \item[(i)] A linearly independent list of $3$ vectors in $\R^4$.
    \item[(ii)] A pair of subspaces $V,W\leq\R^4$ such that
     $\dim(V)=\dim(W)=2$ and $V\cap W=\spn(e_1+e_2+e_3+e_4)$.
    \item[(iii)] A two-dimensional subspace $U\leq\R^4$ that contains
     the vector $\bbm 1&2&3&4\ebm^T$ but not the vector 
     $\bbm 4&3&2&1\ebm^T$.
    \item[(iv)] A matrix whose characteristic polynomial
     is $t^2+2$.
    \item[(v)] A $3\tm 3$ matrix of rank $2$ that is not in RREF.
   \end{itemize}
 \end{itemize}
\end{problem}
\begin{solution}
 \begin{itemize}
  \item[(a)]
   \begin{itemize}
    \item[(i)] This is false. \mk For example, if
     $V=W=\spn(e_1)\leq\R^2$ then $V+W$ is also $\spn(e_1)$, so
     $\dim(V+W)=1$ but $\dim(V)+\dim(W)=2$. \mk (However, the reversed
     inequality $\dim(V+W)\leq\dim(V)+\dim(W)$ is always valid.)
    \item[(ii)] This is false.  \mk In fact, no list of five vectors
     in $\R^4$ can ever be linearly independent.  For a specific
     example, the list $e_1,e_2,e_3,e_4,0$ spans $\R^4$ and is not
     linearly independent. \mk
    \item[(iii)] This is false. \mk For example, if $w_1=e_1$ and
     $w_2=w_3=w_4=0$ then $w_1$ is not a linear combination of $w_2$,
     $w_3$ and $w_4$, but nonetheless the list $w_1,w_2,w_3,w_4$ is
     linearly dependent. \mk
    \item[(iv)] This is false. \mk The general method for such problems
     is as follows: we row-reduce the matrix $[a_1|a_2|a_3|b]$ to get
     a matrix $M$; then $b$ is a linear combination of $a_1$, $a_2$
     and $a_3$ if and only if $M$ has no pivot in the last column.  As
     the identity matrix has a pivot in the last column, we see that
     $b$ is \emph{not} a linear combination of $a_1$, $a_2$ and
     $a_3$. \mk
    \item[(v)] This is true. \mk The equations $Mu=2u$ and $Mv=3v$
     mean that $u$ and $v$ are eigenvectors of the symmetric matrix
     $M$ with distinct eigenvalues, namely $2$ and $3$.  A general
     theorem says that such eigenvectors are orthogonal. \mk
   \end{itemize}
  \item[(b)] In each case, there are many possible correct answers.
   \begin{itemize}
    \item[(i)] The simplest example is the list $e_1,e_2,e_3$. \mks{2}
    \item[(ii)] One example is to take
     \begin{align*}
      V &= \spn(e_1,e_2+e_3+e_4) 
         = \{\bbm x&y&y&y\ebm^T\st x,y\in\R\} \\
      W &= \spn(e_1+e_2+e_3,e_4) 
         = \{\bbm x&x&x&y\ebm^T\st x,y\in\R\}. \mks{2}
     \end{align*}
    \item[(iii)] The simplest example is to take
     $U=\spn(e_1,2e_2+3e_3+4e_4)$. \mks{2}
    \item[(iv)] One example is $\bbm 0&1\\-2&0\ebm$. \mks{2}
    \item[(v)] Examples include $\bbm 2&0&0\\0&2&0\\0&0&0\ebm$ and
     $\bbm 1&1&1\\0&1&1\\0&1&1\ebm$. \mks{2}
   \end{itemize}
 \end{itemize}
\end{solution}

\begin{problem}[2012-13]
 \begin{itemize}
  \item[(a)] Are the following statements true or false?  Justify your
   answers. \mrks{10}
   \begin{itemize}
    \item[(i)] If $V$ and $W$ are subspaces of $\R^n$, then
     $\dim(V+W)\leq\dim(V)+\dim(W)$.
    \item[(ii)] If the list $v_1,\dotsc,v_4$ spans $\R^4$, then it is
     also linearly independent.
    \item[(iii)] If $w_1$ can be expressed as a linear combination of
     $w_2$, $w_3$ and $w_4$, then the list $w_1,w_2,w_3,w_4$ is
     linearly independent.
    \item[(iv)] If $a_1,a_2,a_3,b\in\R^4$ and the matrix
     $[a_1|a_2|a_3|b]$ row-reduces to the identity matrix, then $b$ is
     a linear combination of $a_1$, $a_2$ and $a_3$.
    \item[(v)] If $M$ is a square matrix with $M^T=M$, and $u$ and $v$
     are vectors with $u+Mu=v-Mv=0$, then $u.v=0$.
   \end{itemize}
  \item[(b)] Give examples of the following: \mrks{10}
   \begin{itemize}
    \item[(i)] A spanning set for $\R^3$ that is not a basis.
    \item[(ii)] A pair of subspaces $V,W\leq\R^4$ such that
     $\dim(V)=\dim(W)=2$ and $\dim(V+W)=3$.
    \item[(iii)] A two-dimensional subspace $U\leq\R^4$ such that
     $w+x+y+z=0$ for all vectors $\bbm w&x&y&z\ebm^T\in U$.
    \item[(iv)] A non-diagonal matrix whose characteristic polynomial
     is $t^2-1$.
    \item[(v)] A $2\tm 3$ matrix of rank $1$ that is not in RREF.
   \end{itemize}
 \end{itemize}
\end{problem}
\begin{solution}
 \begin{itemize}
  \item[(a)]
   \begin{itemize}
    \item[(i)] This is true. \mk The dimension formula says that
     $\dim(V+W)=\dim(V)+\dim(W)-\dim(V\cap W)$, and this is clearly
     less than or equal to $\dim(V)+\dim(W)$. \mk
    \item[(ii)] This is true.  \mk In general, if we have a list of $n$
     vectors in $\R^n$ (so the number of vectors is the same as the
     size of each vector) then the list is linearly independent iff it
     spans iff it is a basis. \mk
    \item[(iii)] This is false. \mk If $w_1$ is a linear combination of
     $w_2$, $w_3$ and $w_4$, then $w_1=\al w_2+\bt w_3+\gm w_4$ for
     some scalars $\al$, $\bt$ and $\gm$.  This means that we have a
     nontrivial linear relation $(-1)w_1+\al w_2+\bt w_3+\gm w_4$ on
     the list $w_1,w_2,w_3,w_4$, so that list is linearly dependent
     (not independent). \mk
    \item[(iv)] This is false. \mk The general method for such problems
     is as follows: we row-reduce the matrix $[a_1|a_2|a_3|b]$ to get
     a matrix $M$; then $b$ is a linear combination of $a_1$, $a_2$
     and $a_3$ if and only if $M$ has no pivot in the last column.  As
     the identity matrix has a pivot in the last column, we see that
     $b$ is \emph{not} a linear combination of $a_1$, $a_2$ and
     $a_3$. \mk
    \item[(v)] This is true. \mk The equations $u+Mu=v-Mv=0$ mean that
     $u$ and $v$ are eigenvectors of the symmetric matrix $M$ with
     distinct eigenvalues, namely $-1$ and $+1$.  A general theorem
     says that such eigenvectors are orthogonal. \mk
   \end{itemize}
  \item[(b)] In each case, there are many possible correct answers.
   \begin{itemize}
    \item[(i)] The simplest example is the list $e_1,e_2,e_3,0$. \mks{2}
    \item[(ii)] The simplest example is to take $V=\spn(e_1,e_2)$ and
     $W=\spn(e_1,e_3)$ (so that $V+W=\spn(e_1,e_2,e_3)$). \mks{2}
    \item[(iii)] The simplest example is to take
     $U=\spn(e_1-e_2,e_3-e_4)$. \mks{2}
    \item[(iv)] Possible examples include $\bbm 0&1\\1&0\ebm$ and
     $\bbm 1&1\\0&-1\ebm$. \mks{2}
    \item[(v)] Examples include $\bbm 2&0&0\\0&0&0\ebm$ and
     $\bbm 1&1&1\\1&1&1\ebm$. \mks{2}
   \end{itemize}
 \end{itemize}
\end{solution}


\begin{problem}[2011-12 resit]
 \begin{itemize}
  \item[(1)]
   Are the following statements true or false?  Justify your answers
   carefully. \mrks{10}
   \begin{itemize}
    \item[(a)] Let $A$ be a $4\tm 4$ matrix with exactly two nonzero
     rows.  Then the rank of $A$ is $2$.
    \item[(b)] No list of five vectors in $\R^6$ can span $\R^6$.
    \item[(c)] The following vectors are linearly independent:
     \[ \bbm 1\\ 7\\ 9\\ 1 \ebm \hspace{3em}
        \bbm 4\\ 8\\ 1\\ 4 \ebm \hspace{3em}
        \bbm 7\\ 0\\ 3\\ 6 \ebm \hspace{3em}
        \bbm 4\\ 1\\ 5\\ 4 \ebm \hspace{3em}
        \bbm 0\\-2\\ 6\\-1 \ebm
     \]
    \item[(d)] Let $V$ and $W$ be subspaces of $\R^n$; then
     $\dim(V+W)=\dim(V)+\dim(W)$. 
    \item[(e)] Every subspace of $\R^3$ is either a line or a plane.
   \end{itemize}
  \item[(2)]
   Which of the following sets is a subspace of $\R^4$?  Justify your
   answers. \mrks{8}
   \begin{align*}
    V_1 &= \{\bbm w&x&y&z\ebm^T \in\R^4 \st wxyz=0\} \\
    V_2 &= \{\bbm w&x&y&z\ebm^T \in\R^4 \st w^4+z^4=0\} \\
    V_3 &= \{\bbm w&x&y&z\ebm^T \in\R^4 \st w+x+y+z\geq 0\} \\
    V_4 &= \{\bbm w&x&y&z\ebm^T \in\R^4 \st w=x=y=z\}.
   \end{align*}
  \item[(3)] Give examples of the following. \mrks{7}
   \begin{itemize}
    \item[(a)] A list of four vectors in $\R^3$ such that any two of
     them are linearly independent, but any three of them are linearly
     dependent. 
    \item[(b)] A linearly dependent list that spans $\R^2$.
    \item[(c)] A $3\tm 3$ matrix that has only two distinct
     eigenvalues. 
    \item[(d)] A $2\tm 2$ matrix that is invertible but not orthogonal.
   \end{itemize}
 \end{itemize}
\end{problem}
\begin{solution}
 \begin{itemize}
  \item[(1)]
   \begin{itemize}
    \item[(a)] This is false \mk; for example, the matrix
     $A=\bbm 1&0&0&0\\ 1&0&0&0 \\ 0&0&0&0 \\ 0&0&0&0\ebm$ only has
     rank one. \mk
    \item[(b)] This is true \mk.  Every spanning list in $\R^n$ has
     length at least $n$. \mk.
    \item[(c)] This is false \mk: any list of five vectors in $\R^4$ is
     automatically linearly dependent \mk.  (In fact, we have
     $v_1-v_2+v_3-v_4-v_5=0$, but it is not necessary to work this out.) 
    \item[(d)] This is false \mk.  We have
     $\dim(V+W)=\dim(V)+\dim(W)-\dim(V\cap W)$, which is usually less
     than $\dim(V)+\dim(W)$. \mk
    \item[(e)] This is false \mk: there are two more possibilities,
     namely the zero subspace and $\R^3$ itself. \mk
   \end{itemize}
  \item[(2)]
   \begin{itemize}
    \item[(a)] The set $V_1$ is not a subspace \mk.  Indeed, the vectors
     $a=\bbm 1&1&0&0\ebm^T$ and $b=\bbm 0&0&1&1\ebm^T$ both lie in
     $V_1$, but the vector $a+b=\bbm 1&1&1&1\ebm^T$ does not lie in
     $V_1$, so $V_1$ is not closed under addition. \mk
    \item[(b)] The set $V_2$ is a subspace \mk.  Indeed, as all fourth
     powers are nonnegative, the only way we can have $w^4+z^4=0$ is if
     $w=z=0$.  Thus $V_2=\{(0,x,y,0)\st x,y\in\R\}$, which is clearly
     a subspace \mk. 
    \item[(c)] The set $V_3$ is not a subspace \mk.  Indeed, the vector
     $a=\bbm 1&0&0&0\ebm^T$ lies in $V_3$, but the vector
     $(-1).a=\bbm -1&0&0&0\ebm$ does not lie in $V_3$, so $V_3$ is not
     closed under scalar multiplication. \mk.
    \item[(d)] The set $V_4$ is a subspace \mk, because it can be
     described as $\spn(\bbm 1&1&1&1\ebm^T)$ \mk. (A mark will be
     given for any reasonable argument.)
   \end{itemize}
  \item[(3)] 
   \begin{itemize}
    \item[(a)] The simplest example is the list
     \[ \bbm 1\\ 0\\ 0\ebm,\quad 
        \bbm 1\\ 1\\ 0\ebm,\quad
        \bbm 1\\ 2\\ 0\ebm,\quad
        \bbm 1\\ 3\\ 0\ebm.  \mks{2}
     \]
    \item[(b)] The simplest example is the list $e_1,e_2,0$.  For a
     slightly less degenerate example, we could use
     $e_1,e_2,e_1+e_2$.  \mks{2}
    \item[(c)] The simplest example is the matrix
     $\bbm 0&0&0 \\ 0&0&0\\ 0&0&1\ebm$.  \mk
    \item[(d)] The simplest example is the matrix 
     $\bbm 1&1\\0&1\ebm$.  \mks{2}
   \end{itemize}
 \end{itemize}
\end{solution}


\begin{problem}[2011-12]
 \begin{itemize}
  \item[(1)]
   Are the following statements true or false?  Justify your answers
   carefully. \mrks{9}
   \begin{itemize}
    \item[(a)] Any list of four vectors in $\R^3$ spans $\R^3$.
    \item[(b)] There exists a linearly dependent list of vectors that
     spans $\R^3$.
    \item[(c)] The following vectors are linearly independent:
     \[ \bbm 1\\ 2\\ 3\\ 4 \ebm \hspace{3em}
        \bbm 2\\ 2\\ 3\\ 4 \ebm \hspace{3em}
        \bbm 3\\ 3\\ 3\\ 4 \ebm \hspace{3em}
        \bbm 4\\ 4\\ 4\\ 4 \ebm \hspace{3em}
        \bbm 4\\ 3\\ 2\\ 1 \ebm
     \]
    \item[(d)] The following vectors form a basis of $\R^3$:
     \[ \bbm 1 \\ 1 \\ 1 \ebm \hspace{4em}
        \bbm 1 \\ 10 \\ 100 \ebm \hspace{4em}
        \bbm 1 \\ 11 \\ 101 \ebm
     \]
   \end{itemize}
  \item[(2)]
   Which of the following sets is a subspace of $\R^4$?  Justify your
   answers. \mrks{9}
   \begin{align*}
    V_1 &= \{\bbm w&x&y&z\ebm^T \in\R^4 \st w+x+y+z=0\} \\
    V_2 &= \{\bbm w&x&y&z\ebm^T \in\R^4 \st w^2+x^2+y^2+z^2=0\} \\
    V_3 &= \{\bbm w&x&y&z\ebm^T \in\R^4 \st w^3+x^3+y^3+z^3=0\} \\
    V_4 &= \{\bbm w&x&y&z\ebm^T \in\R^4 \st w+x+y+z=1\}.
   \end{align*}
  \item[(3)] Give examples of the following. \mrks{7}
   \begin{itemize}
    \item[(a)] A list of $4$ vectors in $\R^3$ such that any three of
     them form a basis.
    \item[(b)] A pair of subspaces $V,W\leq\R^6$ with
     $\dim(V)=\dim(W)=3$ and $\dim(V+W)=4$.
    \item[(c)] A list of three subspaces $P,Q,R\leq\R^3$ such that
     $\dim(P)=\dim(Q)=\dim(R)=2$ and $\dim(P\cap Q\cap R)=1$.
   \end{itemize}
 \end{itemize}
\end{problem}
\begin{solution}
 \begin{itemize}
  \item[(1)]
   \begin{itemize}
    \item[(a)] This is false \mk.  For the most extreme example, the list
     $0,0,0,0$ clearly does not span $\R^3$.  For a less degenerate
     example in which the four vectors are all different, we can take
     $\bbm 1\\0\\0\ebm,\bbm 2\\0\\0\ebm,\bbm 3\\0\\0\ebm,\bbm 4\\0\\0\ebm$
     \mk.
    \item[(b)] This is true \mk.  The simplest example is the list $e_1$,
     $e_2$, $e_3$, $0$ \mk.
    \item[(c)] This is false \mk: any list of five vectors in $\R^4$ is
     automatically linearly dependent \mk.  If we call the vectors $v_1$ to
     $v_5$, then we have
     \[ 4v_1-5v_4+4v_5 = 
         \bbm 4\\8\\12\\16\ebm  -
         \bbm 20\\ 20\\ 20\\ 20\ebm +
         \bbm 16\\ 12 \\ 8 \\ 4 \ebm = 
         \bbm 0\\ 0\\ 0\ebm,
     \]
     which is a specific example of a nontrivial linear relation.
    \item[(d)] This is true \mk.  One way to prove it is to take the vectors
     as the columns of a matrix, and start row-reducing it:
     \[ 
      \bbm 1 & 1 & 1 \\ 1 & 10 & 11 \\ 1 & 100 & 101 \ebm
      \to
      \bbm 1 & 1 & 1 \\ 0 & 9 & 10 \\ 0 & 99 & 100 \ebm
      \to
      \bbm 1 & 1 & 1 \\ 0 & 9 & 10 \\ 0 & 0 & -10 \ebm
     \]
     We now have an upper triangular matrix with nonzero entries on
     the diagonal, and any such matrix can be row-reduced to the
     identity.  It follows that the given list is a basis \mks{2}.
   \end{itemize}
  \item[(2)] 
   \begin{itemize}
    \item[(a)] The set $V_1$ is a subspace \mk.  Indeed, it is clear that
     the zero vector lies in $V_1$.  Next, suppose we have two elements
     $a=\bbm w&x&y&z\ebm^T$ and $a'=\bbm w'&x'&y'&z'\ebm^T$ in $V_1$, so
     $w+x+y+z=0$ and $w'+x'+y'+z'=0$.  By adding these equations, we see
     that $(w+w')+(x+x')+(y+y')+(z+z')=0$, so the vector
     $a+a'=\bbm w+w'&x+x'&y+y'&z+z'\ebm$ also lies in $V_1$.  This shows
     that $V_1$ is closed under addition.  Similarly, for any $t\in\R$
     we have $tw+tx+ty+tz=0$, showing that $ta\in V_1$.  This means that
     $V_1$ is closed under scalar multiplication and so is a subspace \mks{2}.
    \item[(b)] The set $V_2$ is a subspace \mk.  Indeed, as all squares are
     nonnegative, the only way we can have $w^2+x^2+y^2+z^2=0$ is if
     $w=x=y=z=0$.  Thus $V_2=\{0\}$, which is clearly a subspace \mk.
    \item[(c)] The set $V_3$ is not a subspace \mk.  Indeed, the vectors
     $a=\bbm 1&-1&0&0\ebm^T$ and $a'=\bbm 1&0&-1&0\ebm$ lie in $V_3$,
     but the sum $a+a'=\bbm 2&-1&-1&0\ebm^T$ does not (because
     $2^3+(-1)^3+(-1)^3+0^3=6\neq 0$) so $V_3$ is not closed under
     addition \mk.
    \item[(d)] The set $V_4$ is not a subspace \mk, because it does not
     contain the zero vector \mk.
   \end{itemize}
  \item[(3)] 
   \begin{itemize}
    \item[(a)] The simplest example is the list
     $e_1,e_2,e_3,e_1+e_2+e_3$.  \mks{2}
    \item[(b)] The simplest example is to take $V=\spn(e_1,e_2,e_3)$
     and $W=\spn(e_1,e_2,e_4)$ so $V+W=\spn(e_1,e_2,e_3,e_4)$. \mks{2}
    \item[(c)] We need three planes that meet along a common line.
     One example is to take
     \[ P = \left\{\bbm x\\y\\z\ebm\in\R^3\st y=z\right\} \hspace{3em}
        Q = \left\{\bbm x\\y\\z\ebm\in\R^3\st y=0\right\} \hspace{3em}
        R = \left\{\bbm x\\y\\z\ebm\in\R^3\st y=-z\right\},
     \]
     so 
     \[ P\cap Q\cap R = 
         \left\{\bbm x\\y\\z\ebm\in\R^3\st y=z=0\right\} =
          \text{ the $x$-axis }.  \mks{3}
     \]
   \end{itemize}
 \end{itemize}
\end{solution}

\begin{problem}[Mock 1]
 \begin{itemize}
  \item[(a)]
   Are the following statements true or false?  Justify your answers.
   \mrks{9}
   \begin{itemize}
    \item[(i)] No list of four vectors in $\R^5$ can span $\R^5$.
    \item[(ii)] Every linearly independent list is a basis.
    \item[(iii)] Let $u$ and $v$ be eigenvectors for a square matrix
     $A$, with different eigenvalues; then $u$ and $v$ are orthogonal.
    \item[(iv)] Let $V$ and $W$ be subspaces of $\R^n$ with
     $V\cap W=\{0\}$; then $\dim(V+W)=\dim(V)+\dim(W)$.
   \end{itemize}
  \item[(b)]
   Which of the following sets is a subspace of $\R^4$?  Justify your
   answers. \mrks{9}
   \begin{align*}
    V_1 &= \{\bbm w&x&y&z\ebm^T \in\R^4 \st w^2-x^2+y^2-z^2=0\} \\
    V_2 &= \{\bbm w&x&y&z\ebm^T \in\R^4 \st 2w-3x+4y-5z=0\} \\
    V_3 &= \{\bbm w&x&y&z\ebm^T \in\R^4 \st 2w-3x+4y-5z=1\} \\
    V_4 &= \{\bbm w&x&y&z\ebm^T \in\R^4 \st w+x^2+y^3+z^4=0\}.
   \end{align*}
  \item[(c)] Give examples of the following. \mrks{7}
   \begin{itemize}
    \item[(i)] A list of four vectors in $\R^2$ such that any two of
     them form a basis.
    \item[(ii)] A list of four vectors in $\R^4$ such that any two of
     them are linearly independent, but any three of them are linearly
     dependent. 
    \item[(iii)] A $3\tm 3$ matrix that has only two distinct
     eigenvalues. 
   \end{itemize}
 \end{itemize}
\end{problem}
\begin{solution}
 \begin{itemize}
  \item[(a)]
   \begin{itemize}
    \item[(i)] This is true \mk.  A general theorem in the notes says
     that any list that spans $\R^n$ must have at least $n$ elements \mk.
    \item[(ii)] This is not true \mk.  For example, the vectors
     $u_1=\bbm 1&0&0\ebm^T$ and $u_2=\bbm 0&1&0\ebm^T$ give a linearly
     independent list $u_1,u_2$ in $\R^3$, but this list does not span
     $\R^3$ and so is not a basis. \mk
    \item[(iii)] This is false \mk.  For a concrete counterexample,
     consider the matrix $A=\bbm 0&1\\0&1\ebm$ and the vectors
     $u=\bbm 1\\0\ebm$ and $v=\bbm 1\\1\ebm$.  These are eigenvectors
     of eigenvalue $0$ and $1$ respectively, but they are not
     orthogonal.  The statement is true for symmetric matrices, but
     not more generally \mks{2}.
    \item[(iv)] This is true \mk.  It is a general theorem that for any
     subspaces $V,W\leq\R^n$ we have
     $\dim(V+W)+\dim(V\cap W)=\dim(V)+\dim(W)$.  In the case where
     $V\cap W=\{0\}$ we have $\dim(V\cap W)=0$ and so
     $\dim(V+W)=\dim(V)+\dim(W)$ \mk.  
   \end{itemize}
  \item[(b)] 
   \begin{itemize}
    \item The set $V_1$ is not a subspace \mk.  Indeed, the
     vectors $a=\bbm 1&1&0&0\ebm^T$ and $a'=\bbm 1&-1&0&0\ebm^T$ are
     elements of $V_1$ but the vector $a+a'=\bbm 2&0&0&0\ebm$ is not;
     so $V_1$ is not closed under addition and cannot be a subspace. \mk
    \item The set $V_2$ is a subspace \mk.  Indeed, it is clear that
     the zero vector lies in $V_2$.  Next, suppose we have two elements
     $a=\bbm w&x&y&z\ebm^T$ and $a'=\bbm w'&x'&y'&z'\ebm^T$ in $V_2$, so
     $2w-3x+4y-5z=0$ and $2w'-3x'+4y'-5z'=0$.  By adding these
     equations, we see that $2(w+w')-3(x+x')+4(y+y')-5(z+z')=0$, so
     the vector $a+a'=\bbm w+w'&x+x'&y+y'&z+z'\ebm$ also lies in
     $V_2$.  This shows that $V_2$ is closed under addition.
     Similarly, for any $t\in\R$ we have $2tw-3tx+4ty-5tz=0$, showing
     that $ta\in V_2$.  This means that $V_2$ is closed under scalar
     multiplication and so is a subspace \mks{2}. 
    \item The set $V_3$ is not a subspace \mk, because it does
     not contain the zero vector \mk.
    \item The set $V_4$ is not a subspace \mk.  Indeed, the vector
     $a=\bbm -1&1&0&0\ebm^T$ is an element of $V_4$, but the vector
     $-a=\bbm 1&-1&0&0\ebm^T$ is not, which shows that $V_4$ is not
     closed under scalar multiplication. \mk.
   \end{itemize}
  \item[(c)] 
   \begin{itemize}
    \item[(i)] The simplest example is the list
     $\bbm 1\\0\ebm,\bbm 1\\1\ebm,\bbm 1\\2\ebm,\bbm 1\\3\ebm$ \mks{2}.
    \item[(ii)] The simplest example is the list
     $\bbm 1\\0\\0\ebm,
      \bbm 1\\1\\0\ebm,
      \bbm 1\\2\\0\ebm,
      \bbm 1\\3\\0\ebm$ \mks{3}.
    \item[(iii)] The simplest example is the matrix 
     $A=\bbm 0&0&0\\0&0&0\\0&0&1\ebm$ \mks{2}.
   \end{itemize}
 \end{itemize}
\end{solution}

\begin{problem}[Mock 2]
 \begin{itemize}
  \item[(a)]
   Are the following statements true or false?  You do not need to justify your answers.
   \mrks{9}
   \begin{itemize}
    \item[(i)] Let $A$ be a $3\tm 5$ matrix with linearly independent
     rows.  Then the rank of $A$ is $5$.
    \item[(ii)] Let $A$ be a $5\tm 5$ matrix with $A^T=A$, and let $u$
     and $v$ be vectors with $Au=2u$ and $Av=3v$.  Then $u.v=0$.
    \item[(iii)] Let $U$ be a $4\tm 4$ matrix whose columns are all
     orthogonal to each other.  Then $U^TU=I$.
    \item[(iv)] The list $\bbm 1\\3\ebm,\bbm 7\\1\ebm,\bbm 4\\4\ebm$
     is linearly independent.
    \item[(v)] The list
     $\bbm 1\\1\\0\\0\ebm,\bbm 0\\1\\1\\0\ebm,\bbm 0\\0\\1\\1\ebm$
     spans $\R^4$.
    \item[(vi)] Let $A$ be a $5\tm 5$ matrix with $\rnk(A)=5$.  Then
     $A$ is invertible.
    \item[(vii)] Let $u_1,u_2,u_3,u_4,u_5,u_6$ be a linearly
     independent list of vectors in $\R^6$.  Then the list
     $u_1,u_3,u_5$ is also linearly independent.
    \item[(viii)] Let $u_1,u_2,u_3,u_4,u_5,u_6$ be a list of vectors
     that spans $\R^6$.  Then the list $u_1,u_3,u_5$ also spans
     $\R^6$.
    \item[(ix)] Every subspace of $\R^3$ is either a line or a plane.
   \end{itemize}
  \item[(b)]
   Which of the following sets is a subspace of $\R^4$?  Here you do
   need to justify your answers. \mrks{8}
   \begin{align*}
    V_1 &= \{\bbm w&x&y&z\ebm^T \in\R^4 \st w\geq x\} \\
    V_2 &= \{\bbm w&x&y&z\ebm^T \in\R^4 \st w=x=y=z\} \\
    V_3 &= \{\bbm w&x&y&z\ebm^T \in\R^4 \st w+x+y+z \text{ is an integer }\} \\
    V_4 &= \{\bbm w&x&y&z\ebm^T \in\R^4 \st wxyz=0\}.
   \end{align*}
  \item[(c)] Give examples of the following. \mrks{8}
   \begin{itemize}
    \item[(i)] A linearly dependent list that spans $\R^2$.
    \item[(ii)] A $2\tm 2$ matrix that is invertible but not orthogonal.
    \item[(iii)] A subspace of $\R^3$ that contains the vector
     $\bbm 1&2&3\ebm^T$ and has dimension $2$.
    \item[(iv)] A $4\tm 4$ matrix $A$ such that $A\neq I_4$ and
     $\chi_A(t)=(t-1)^4$. 
   \end{itemize}
 \end{itemize}
\end{problem}
\begin{solution}
 \begin{itemize}
  \item[(a)]
   \begin{itemize}
    \item[(i)] This is false \mk.  (The rank will be $3$ rather than
     $5$.) 
    \item[(ii)] This is true \mk.   (Any two eigenvectors of a
     symmetric matrix with distinct eigenvalues are orthogonal.)
    \item[(iii)] This is false \mk.  (The matrix $2I_4$ is a
     counterexample.  It would be true if we said that the columns are
     orthonormal rather than just orthogonal.)
    \item[(iv)] This is false \mk.  (No list of $3$ vectors in $\R^2$
     can be linearly independent.)
    \item[(v)] This is false \mk.  (No list of $3$ vectors can span
     $\R^4$.)
    \item[(vi)] This is true \mk. (Rank $5$ means that the $5$ columns
     are linearly independent vectors in $\R^5$, so they form a basis
     of $\R^5$, which implies that the matrix is invertible.)
    \item[(vii)] This is true \mk.  (Any linear relation
     $\lm_1u_1+\lm_2u_3+\lm_3u_5=0$ gives a linear relation 
     \[ \lm_1u_1+0u_2+\lm_2u_3+0u_4+\lm_3u_5+0u_6=0, \]
     which must be the trivial relation because $u_1,\dotsc,u_6$ is an
     independent list.  This implies that $\lm_1=\lm_2=\lm_3=0$, so
     the original relation is also trivial.)
    \item[(viii)] This is false \mk. (No list of length $3$ can ever
     span $\R^6$.)
    \item[(ix)] This is false \mk. (There are also the subspaces
     $\{0\}$ and $\R^3$ itself.)
   \end{itemize}
  \item[(b)] 
   \begin{itemize}
    \item The set $V_1$ is not a subspace \mk.  Indeed, the
     vector $a=\bbm 1&0&0&0\ebm^T$ is an element of $V_1$ but the
     vector $-a=\bbm -1&0&0&0\ebm$ is not; so $V_1$ is not closed
     under scalar multiplication and cannot be a subspace. \mk 
    \item The set $V_2$ is a subspace \mk.  Indeed, $V_2$ is just the
     set of multiples of the vector $b=\bbm 1&1&1&1\ebm^T$ or in other
     words $V_2=\spn(b)$, and the span of any list is a subspace. \mk
    \item The set $V_3$ is not a subspace \mk.  Indeed, the
     vector $a=\bbm 1&0&0&0\ebm^T$ is an element of $V_3$ but the
     vector $\half a=\bbm \half&0&0&0\ebm^T$ is not; so $V_3$ is not
     closed under scalar multiplication and cannot be a subspace. \mk
    \item The set $V_4$ is not a subspace \mk.  Indeed, the vectors
     $c=\bbm 1&1&0&0\ebm^T$ and $d=\bbm 0&0&1&1\ebm^T$ are elements of
     $V_4$ but the vector $c+d=\bbm 1&1&1&1\ebm^T$ is not, so $V_4$ is
     not closed under addition and cannot be a subspace. \mk
   \end{itemize}
  \item[(c)] 
   \begin{itemize}
    \item[(i)] One example is the list
     $\bbm 1\\0\ebm,\bbm 0\\1\ebm,\bbm 1\\1\ebm$ \mks{2}.
    \item[(ii)] One example is the matrix
     $2I_2=\bbm 2&0\\0&2\ebm$ \mks{2}.
    \item[(iii)] One example is the space
     $V=\spn(\bbm 1&0&0\ebm^T,\;\bbm 0&2&3\ebm^T)$ \mks{2}.
    \item[(iv)] One example is the matrix
     \[ A = \bbm 1&1&1&1 \\ 0&1&1&1 \\ 0&0&1&1 \\ 0&0&0&1 \ebm
          \mks{2}.
     \]
   \end{itemize}
 \end{itemize}
\end{solution}

%%%%%%%%%%%%%%%%%%%%%%%%%%%%%%%%%%%%%%%%%%%%%%%%%%%%%%%%%%%%%%%%%%%%%%

\section{Question 4}
\setcounter{probcounter}{0}

\begin{problem}[2013-14]
 Consider the vectors
 \[ 
  a_1 = \bbm  1\\ 1\\ 0\\  0 \ebm \quad
  a_2 = \bbm  0\\ 0\\ 1\\ -1 \ebm \quad
  v_1 = \bbm  1\\-1\\-1\\ -1 \ebm \quad
  v_2 = \bbm  1\\ 0\\ 0\\ -1 \ebm \quad
  c_1 = \bbm  1\\-1\\ 1\\  1 \ebm \quad
  c_2 = \bbm  1\\ 1\\ 1\\ -1 \ebm
 \]
 and the subspaces
 \[ U = \ann(a_1,a_2) \qquad
    V = \spn(v_1,v_2) \qquad
    W = \ann(c_1,c_2)
 \]
 in $\R^4$.
 \begin{itemize}
  \item[(a)] Find the canonical basis for $U$. \mrks{4}
  \item[(b)] Find the canonical basis for $V$. \mrks{3}
  \item[(c)] Find the canonical basis for $W$. \mrks{4}
  \item[(d)] Find the canonical basis for $U\cap V\cap W$. \mrks{5}
  \item[(e)] Find the canonical basis for $U+V+W$. \mrks{4}
 \end{itemize}
\end{problem}
\begin{solution}
 \begin{itemize}
  \item[(a)] For a vector $u=\bbm p&q&r&s\ebm^T$ to lie in $U$, it
   must satisfy $u.a_1=u.a_2=0$, or equivalently $p+q=r-s=0$ \mk.  This
   means that $u$ must have the form 
   \[ u = \bbm p \\ -p \\ r \\ r \ebm 
        = p \bbm 1 \\ -1 \\ 0 \\ 0 \ebm + 
          q \bbm 0 \\ 0 \\ 1 \\ 1 \ebm. \mks{2}
   \]
   From this it follows that the vectors 
   \[ u_1 = \bbm 1 & -1 & 0 & 0 \ebm^T \qquad \text{ and } \qquad
      u_2 = \bbm 0 & 0 & 1 & 1 \ebm^T 
   \]
   form the canonical basis for $U$. \mk
  \item[(b)] We use the following row-reduction:
   \[ \left[\begin{array}{ccc}
        & v_1^T & \\ \hline
        & v_2^T &
       \end{array}\right] \mk =
      \bbm 1 & -1 & -1 & -1 \\ 1 & 0 & 0 & -1 \ebm 
      \to 
      \bbm 1 & -1 & -1 & -1 \\ 0 & 1 & 1 &  0 \ebm 
      \to 
      \bbm 1 &  0 &  0 & -1 \\ 0 & 1 & 1 &  0 \ebm \mk
   \]
   It follows that the vectors $v'_1=\bbm 1&0&0&-1\ebm^T$ and 
   $v'_2=\bbm 0&1&1&0\ebm^T$ form the canonical basis for $V$. \mk
  \item[(c)] For a vector $w=\bbm p&q&r&s\ebm^T$ to lie in $U$, it
   must satisfy $w.c_1=w.c_2=0$, or equivalently $p-q+r+s=0$ and
   $p+q+r-s=0$. \mk  By adding and subtracting these equations, we obtain
   $p+r=0$ and $q-s=0$, so 
   \[ w= \bbm p\\q\\-p\\q\ebm =
       p \bbm 1\\0\\-1\\0\ebm +
       q \bbm 0\\1\\0\\1\ebm. \mks{2}
   \]
   It follows that the vectors 
   \[ w_1 = \bbm 1 \\ 0 \\ -1 \\ 0\ebm \qquad \text{ and } 
      w_2 = \bbm 0 \\ 1 \\  0 \\ 1 \ebm
   \]
   form the canonical basis for $W$. \mk
  \item[(d)] From~(b) it is clear that $V$ is the set of vectors of
   the form $v=\bbm w&x&x&-w\ebm^T$.\mks{2}  For $v$ to lie
   in $U\cap V\cap W$ it must also satisfy
   $v.a_1=v.a_2=v.c_1=v.c_2=0$, or in other words 
   \[ w+x = x-(-w) = w-x+x+(-w) = w+x+x-(-w) = 0. \mk \]
   This just reduces to $x=-w$ \mk, so 
   \[ v=\bbm w&-w&-w&-w\ebm^T = 
       w \;\bbm 1&-1&-1&-1\ebm^T = w\,v_1.
   \]
   We conclude that the vector $v_1=\bbm 1&-1&-1&-1\ebm^T$ (on its own) is
   the canonical basis for $U\cap V\cap W$. \mk
  \item[(e)] The space $U+V+W$ is spanned by $u_1,u_2,v'_1,v'_2,w_1$
   and $w_2$. \mks{2}  We form a matrix with these vectors as rows, and
   row-reduce it:
   {\[
     \bbm
     1&-1&0&0\\
     0&0&1&1\\
     1&0&0&-1\\
     0&1&1&0\\
     1&0&-1&0\\
     0&1&0&1\\
     \ebm
     \to
     \bbm
     1&-1&0&0\\
     0&0&1&1\\
     0&1&0&-1\\
     0&1&1&0\\
     0&1&-1&0\\
     0&1&0&1\\
     \ebm
     \to
     \bbm
     1&0&0&-1\\
     0&0&1&1\\
     0&1&0&-1\\
     0&0&1&1\\
     0&0&-1&1\\
     0&0&0&2\\
     \ebm
     \to
    \] \[
     \bbm
     1&0&0&-1\\
     0&0&1&1\\
     0&1&0&-1\\
     0&0&0&0\\
     0&0&0&2\\
     0&0&0&2\\
     \ebm
     \to
     \bbm
     1&0&0&0\\
     0&0&1&0\\
     0&1&0&0\\
     0&0&0&0\\
     0&0&0&1\\
     0&0&0&0\\
     \ebm
     \to
     \bbm
     1&0&0&0\\
     0&1&0&0\\
     0&0&1&0\\
     0&0&0&1\\
     0&0&0&0\\
     0&0&0&0\\
     \ebm \mk
   \]}
  From this we conclude that $U+V+W$ is all of $\R^4$, so the
  canonical basis is $e_1,e_2,e_3,e_4$. \mk Alternatively, we can observe
  that the vector $e_3$ is $u_2+v'_2-w_2$, so it lies in $U+V+W$.  It
  follows in turn that the vectors $e_1=w_1+e_3$ and $e_2=v'_2-e_3$
  and $e_4=u_2-e_3$ also lie in $U+V+W$, so again we see that $U+V+W$
  is all of $\R^4$.
 \end{itemize}
\end{solution}

\begin{problem}[2012-13 resit]
 Consider the vectors
 \[
   a_1 = \bbm 2  \\ -1 \\ -1 \\ -1 \ebm \hspace{4em}
   a_2 = \bbm 0  \\  1 \\ -1 \\ -3 \ebm \hspace{4em}
   b_1 = \bbm -1 \\  1 \\  2 \\  1 \ebm \hspace{4em}
   b_2 = \bbm -8 \\  5 \\  4 \\  2 \ebm,
 \]
 and put $V=\ann(a_1,a_2)$, and $W=\ann(b_1,b_2)$.
 \begin{itemize}
  \item[(a)] Find the canonical basis for $V$. \mrks{4}
  \item[(b)] Find the canonical basis for $W$. \mrks{4}
  \item[(c)] Find the canonical basis for $V\cap W$. \mrks{5}
  \item[(d)] Find the canonical basis for $V+W$. \mrks{5}
  \item[(e)] State the standard equation relating the dimensions
   of the above four spaces, and verify that it holds in this case.
   \mrks{2}
 \end{itemize}
\end{problem}
\begin{solution}
 \begin{itemize}
  \item[(a)] Consider a vector $x=\bbm x_1&x_2&x_3&x_4\ebm^T$.  This
   lies in $V$ iff $a_1.x=a_2.x=0$, or equivalently
   $2x_1-x_2-x_3-x_4=x_2-x_3-3x_4=0$.  Subtracting these gives 
   $2x_1-2x_2+2x_4=0$, so $x_4=x_2-x_1$.  Substituting this into the
   first equation gives $x_3=3x_1-2x_2$.  Note that we have
   written higher-numbered variables in terms of lower-numbered ones,
   as required when finding the canonical basis \mks{2}.  This gives
   \[ x = \bbm x_1 \\ x_2 \\ 3x_1-2x_2 \\ -x_1+x_2\ebm =
        x_1 \bbm 1\\0\\3\\-1\ebm + x_2 \bbm 0\\1\\-2\\1 \ebm,
   \]
   which shows that the vectors $v_1=\bbm 1&0&3&-1\ebm^T$ and
   $v_2=\bbm 0&1&-2&1 \ebm^T$ form the canonical basis for $V$ \mks{2}.
  \item[(b)] Similarly, we can write the equations $b_1.x=b_2.x=0$
   with the variables in reverse order as follows:
   \begin{align*}
     x_4 + 2x_3 +  x_2 -  x_1 &= 0 \\
    2x_4 + 4x_3 + 5x_2 - 8x_1 &= 0.
   \end{align*}
   These equations can be solved in the standard way to give
   \begin{align*}
    x_4 &= -2x_3-x_1 \\
    x_2 &= 2x_1 \mks{2}
   \end{align*}
   so 
   \[ x = \bbm x_1 \\ 2x_1 \\ x_3 \\ -2x_3-x_1 \ebm 
        = x_1 \bbm 1 \\ 2 \\ 0 \\ -1 \ebm +
          x_3 \bbm 0 \\ 0 \\ 1 \\ -2 \ebm,
   \]
   so the vectors $w_1=\bbm 1&-2&0&-1\ebm^T$ and
   $w_2=\bbm 0&0&1&-2\ebm^T$ form the canonical basis for $W$. \mks{2}
  \item[(c)] $V\cap W$ is the set of vectors $x$ satisfying
   $a_1.x=a_2.x=b_1.x=b_2.x=0$ \mk.  Part~(a) tells us that the equations
   $a_1.x=a_2.x=0$ are equivalent to $x_4=x_2-x_1$ and $x_3=3x_1-2x_2$.
   Part~(b) tells us that the equations $b_1.x=b_2.x=0$ are equivalent
   $x_4=-2x_3-x_1$ and $x_2=2x_1$.  Putting these together, we get
   $x_2=2x_1$ and and $x_3=-x_1$ and $x_4=x_1$ \mks{2}, so
   \[ x = \bbm x_1 \\ 2x_1 \\ -x_1 \\ x_1 \ebm = 
        x_1 \bbm 1 \\ 2 \\ -1 \\ 1 \ebm.
   \] 
   It follows that the vector $u=\bbm 1&2&-1&1\ebm^T$ is (on its own)
   the canonical basis for $V\cap W$ \mks{2}.
  \item[(d)] We have 
   \[ V+W = \spn(v_1,v_2) + \spn(w_1,w_2) = \spn(v_1,v_2,w_1,w_2) \mks{2}. \]
   To find the canonical basis for this space, we row-reduce the
   matrix $[v_1|v_2|w_1|w_2]^T$ \mk:
   \[
    \bbm
    1&0&3&-1\\
    0&1&-2&1\\
    1&2&0&-1\\
    0&0&1&-2\\
    \ebm
    \to
    \bbm
    1&0&3&-1\\
    0&1&-2&1\\
    0&2&-3&0\\
    0&0&1&-2\\
    \ebm
    \to
   \] \[
    \bbm
    1&0&3&-1\\
    0&1&-2&1\\
    0&0&1&-2\\
    0&0&1&-2\\
    \ebm
    \to
    \bbm
    1&0&0&5\\
    0&1&0&-3\\
    0&0&1&-2\\
    0&0&0&0\\
    \ebm
   \]
   The transposes of the nonzero rows of the last matrix are:
   \[ t_1 = \bbm 1 \\ 0 \\ 0 \\  5 \ebm \hspace{4em}
      t_2 = \bbm 0 \\ 1 \\ 0 \\ -3 \ebm \hspace{4em}
      t_3 = \bbm 0 \\ 0 \\ 1 \\ -2 \ebm.
   \]
   These form the canonical basis for $V+W$. \mks{2}
  \item[(e)] The standard relation says that
   $\dim(V)+\dim(W)=\dim(V+W)+\dim(V\cap W)$. \mk
   The above bases show that 
   \[ \dim(V) = 2 \hspace{4em}
      \dim(W) = 2 \hspace{4em}
      \dim(V\cap W) = 1 \hspace{4em}
      \dim(V+W) = 3,
   \]
   so the left hand side and the right hand side are both equal to
   $4$. \mk
 \end{itemize}
\end{solution}

\begin{problem}[2012-13]
 Consider the vectors
 \[ a_1 = \bbm  2 \\  0 \\  0 \\ -2 \ebm \hspace{4em}
    a_2 = \bbm  1 \\  2 \\ -2 \\ -1 \ebm \hspace{4em}
    b_1 = \bbm  2 \\ -1 \\ -1 \\  2 \ebm \hspace{4em}
    b_2 = \bbm -4 \\  2 \\ -4 \\  8 \ebm,
 \]
 and put $V=\ann(a_1,a_2)$, and $W=\ann(b_1,b_2)$.
 \begin{itemize}
  \item[(a)] Find the canonical basis for $V$. \mrks{4}
  \item[(b)] Find the canonical basis for $W$. \mrks{4}
  \item[(c)] Find the canonical basis for $V\cap W$. \mrks{5}
  \item[(d)] Find the canonical basis for $V+W$. \mrks{5}
  \item[(e)] Find a vector that lies in $V+W$ but does not lie in $V$
   or in $W$. \mrks{2}
 \end{itemize}
\end{problem}
\begin{solution}
 \begin{itemize}
  \item[(a)] Consider a vector $x=\bbm x_1&x_2&x_3&x_4\ebm^T$.  This
   lies in $V$ iff $a_1.x=a_2.x=0$, or equivalently
   $2x_1-2x_4=x_1+2x_2-2x_3-x_4=0$.  The solution is clearly $x_4=x_1$
   and $x_3=x_2$ (with $x_1$ and $x_2$ arbitrary).  Note that we have
   written higher-numbered variables in terms of lower-numbered ones,
   as required when finding the canonical basis \mks{2}.  This gives
   \[ x = \bbm x_1 \\ x_2 \\ x_2 \\ x_1\ebm =
        x_1 \bbm 1\\0\\0\\1\ebm + x_2 \bbm 0\\1\\1\\0 \ebm,
   \]
   which shows that the vectors $v_1=\bbm 1&0&0&1\ebm^T$ and
   $v_2=\bbm 0&1&1&0 \ebm^T$ form the canonical basis for $V$ \mks{2}.
  \item[(b)] Similarly, we can write the equations $b_1.x=b_2.x=0$
   with the variables in reverse order as follows:
   \begin{align*}
    2x_4 - x_3 - x_2 + 2x_1 &= 0 \\
    8x_4 - 4x_3 + 2x_2 - 4x_1 &= 0.
   \end{align*}
   From these equations we deduce
   \begin{align*}
    6x_2 - 12 x_1 &= 0 \\
    x_2 &= 2 x_1 \\
    x_4 &= \half x_3 \mks{2},
   \end{align*}
   so 
   \[ x = \bbm x_1 \\ 2x_1 \\ x_3 \\ \half x_3 \ebm 
        = x_1 \bbm 1 \\ 2 \\ 0 \\ 0 \ebm +
          x_2 \bbm 0 \\ 0 \\ 1 \\ 1/2 \ebm,
   \]
   so the vectors $w_1=\bbm 1&2&0&0\ebm^T$ and
   $w_2=\bbm 0&0&1&1/2\ebm^T$ form the canonical basis for $W$. \mks{2}
  \item[(c)] $V\cap W$ is the set of vectors $x$ satisfying
   $a_1.x=a_2.x=b_1.x=b_2.x=0$ \mk.  Part~(a) tells us that the equations
   $a_1.x=a_2.x=0$ are equivalent to $x_4=x_1$ and $x_3=x_2$.
   Part~(b) tells us that the equations $b_1.x=b_2.x=0$ are equivalent
   to $x_2=2x_1$ and $x_4=\half x_3$.  Putting these together, we get
   $x_2=x_3=2x_1$ and $x_4=x_1$ \mks{2}, so
   \[ x = \bbm x_1 \\ 2x_1 \\ 2x_1 \\ x_1 \ebm = 
        x_1 \bbm 1 \\ 2 \\ 2 \\ 1 \ebm.
   \] 
   It follows that the vector $u=\bbm 1&2&2&1\ebm^T$ is (on its own)
   the canonical basis for $V\cap W$ \mks{2}.
  \item[(d)] We have 
   \[ V+W = \spn(v_1,v_2) + \spn(w_1,w_2) = \spn(v_1,v_2,w_1,w_2) \mk. \]
   To find the canonical basis for this space, we row-reduce the
   matrix $[v_1|v_2|w_1|w_2]^T$ \mk:
   \[
    \bbm
    1&0&0&1\\
    0&1&1&0\\
    1&2&0&0\\
    0&0&1&1/2\\
    \ebm
    \to
    \bbm
    1&0&0&1\\
    0&1&1&0\\
    0&2&0&-1\\
    0&0&1&1/2\\
    \ebm
    \to
    \bbm
    1&0&0&1\\
    0&1&1&0\\
    0&0&-2&-1\\
    0&0&1&1/2\\
    \ebm
    \to 
   \] \[
    \bbm
    1&0&0&1\\
    0&1&1&0\\
    0&0&1&1/2\\
    0&0&1&1/2\\
    \ebm
    \to
    \bbm
    1&0&0&1\\
    0&1&1&0\\
    0&0&1&1/2\\
    0&0&0&0\\
    \ebm
    \to
    \bbm
    1&0&0&1\\
    0&1&0&-1/2\\
    0&0&1&1/2\\
    0&0&0&0\\
    \ebm \mks{2}.
   \]
   The transposes of the nonzero rows of the last matrix are:
   \[ t_1 = \bbm 1 \\ 0 \\ 0 \\ 1 \ebm \hspace{4em}
      t_2 = \bbm 0 \\ 1 \\ 0 \\ -1/2 \ebm \hspace{4em}
      t_3 = \bbm 0 \\ 0 \\ 1 \\ 1/2 \ebm.
   \]
   These form the canonical basis for $V+W$. \mk
  \item[(e)] The vector $t_2$ lies in $V+W$, but it does not lie in
   $V$ (because $a_2.t_2=1$) and it does not lie in $W$ (because
   $b_1.t_2=-2$) \mks{2}.  (Note that $t_1$ is not an example because
   it lies in $V$, and $t_3$ is not an example because it lies in
   $W$.)    
 \end{itemize}
\end{solution}


\begin{problem}[2011-12 resit]
 Put 
 \[ v_1 = \bbm  1\\  2\\  1\\ 1 \ebm \hspace{2em}
    v_2 = \bbm  2\\  4\\  1\\ 0 \ebm \hspace{2em}
    u_1 = \bbm  2\\ -1\\  0\\ 0 \ebm \hspace{2em}
    u_2 = \bbm  1\\  2\\ 10\\ 5 \ebm
 \]
 and $V=\spn(v_1,v_2)$ and $W=\ann(u_1,u_2)$.  
 \begin{itemize}
  \item[(a)] Find the canonical basis for $V$. \mrks{4}
  \item[(b)] Find the canonical basis for $W$. \mrks{6}
  \item[(c)] Find the canonical basis for $V+W$. \mrks{5}
  \item[(d)] Find vectors $c_1$ and $c_2$ such that
   $V=\ann(c_1,c_2)$. \mrks{5}
  \item[(e)] Find the canonical basis for $V\cap W$. \mrks{5}
 \end{itemize}
\end{problem}
\begin{solution}
 \begin{itemize}
  \item[(a)] 
   To find the canonical basis for $V$ we perform the following
   row-reduction:
   \[ \left[\begin{array}{ccc}
       & v_1^T & \\\hline 
       & v_2^T & 
      \end{array}\right] \mk
      =
      \bbm 1&2&1&1 \\ 2&4&1&0 \ebm
      \to 
      \bbm 1&2&1&1 \\ 0&0&-1&-2 \ebm
      \to 
      \bbm 1&2&0&-1 \\ 0&0&1&2 \ebm \mk
   \]
   From this we see that the canonical basis is $(a_1,a_2)$ where $a_1$
   and $a_2$ are the transposes of the rows of the above matrix, namely
   \[ a_1 = \bbm 1\\2\\0\\-1\ebm \hspace{4em} 
      a_2 = \bbm 0\\0\\1\\2\ebm. \mks{2}
   \]
  \item[(b)]
   Next, $W$ is the set of vectors $x$ satisfying $x.u_2=0$ and
   $x.u_1=0$, or equivalently 
   \begin{align*}
    5x_4 + 10x_3 + 2x_2 + x_1 &= 0 \\
     -x_2 +  2x_1 &= 0 \mks{2}
   \end{align*}
   Solving these in the standard way gives $x_2=2x_1$ and
   $x_4=-2x_3-x_1$ with $x_3$ and $x_1$ arbitrary \mk, so 
   \[ x = \bbm x_1 \\ 2x_1 \\ x_3 \\ -x_1-2x_3 \ebm 
        = x_1 \bbm 1\\2\\0\\-1\ebm +
          x_3 \bbm 0\\0\\1\\-2\ebm. \mk
   \]
   From this we see that the following matrices form the canonical basis
   for $W$:
   \[ b_1 = \bbm 1\\2\\0\\-1\ebm \hspace{4em}
      b_2 = \bbm 0\\0\\1\\-2\ebm. \mks{2}
   \]
 
  \item[(c)]
   It follows that $V+W=\spn(a_1,a_2,b_1,b_2)$ \mk, but we can omit $b_1$
   because it is the same as $a_1$.  To make this canonical we perform
   the following row-reduction:
   \[ \left[\begin{array}{ccc}
       & a_1^T & \\\hline 
       & a_2^T & \\\hline
       & b_2^T &
      \end{array}\right] \mks{2}
      =
      \bbm 1&2&0&-1 \\ 0&0&1&2 \\ 0&0&1&-2 \ebm
      \to 
      \bbm 1&2&0&-1 \\ 0&0&1&2 \\ 0&0&0&-4 \ebm
      \to 
      \bbm 1&2&0&0 \\ 0&0&1&0 \\ 0&0&0&1 \ebm. \mk
   \]
   This shows that the canonical basis for $V+W$ consists of the vectors
   \[
      \bbm 1&2&0&0\ebm^T \hspace{4em}
      \bbm 0&0&1&0\ebm^T \hspace{4em}
      \bbm 0&0&0&1\ebm^T. \mk
   \]
  \item[(d)] 
   If $x$ is a vector that annihilates the space $V=\spn(v_1,v_2)$, we
   must have $x.v_1=x.v_2=0$, or equivalently
   \begin{align*}
    x_4 + x_3 + 2x_2 + x_1 &= 0 \\
          x_3 + 4x_2 + 2x_1 &= 0.\mk
   \end{align*}
   Subtracting the second of these from the first gives
   $x_4-2x_2-x_1=0$.  We thus have $x_4=2x_2+x_1$ and $x_3=-4x_2-2x_1$
   with $x_1$ and $x_2$ arbitrary, so 
   \[ x = \bbm x_1 \\ x_2 \\ -2x_1-4x_2 \\ x_1+2x_2 \ebm 
        = x_1\bbm 1\\0\\-2\\1\ebm + x_2\bbm 0\\ 1\\-4\\2\ebm. \mks{2}
   \]
   The standard methods now tell us that $V=\ann(c_1,c_2)$, where 
   \[ c_1 = \bbm 1\\0\\-2\\1 \ebm \hspace{4em}
      c_2 = \bbm 0\\1\\-4\\2 \ebm. \mks{2}
   \]

  \item[(e)] We now have 
   $V\cap W=\ann(c_1,c_2)\cap\ann(u_1,u_2)=\ann(c_1,c_2,u_1,u_2)$ \mk.
   In other words, $V\cap W$ is the set of solutions to the equations
   $x.c_1=0$ and $x.c_2=0$ and $x.u_1=0$ and $x.u_2=0$, or
   equivalently 
   \begin{align*}
    x_4 - 2x_3 + x_1 &= 0 \\
    2x_4 -4x_3 + x_2 &= 0 \\
    -x_2 + 2x_1 &= 0 \\
    5x_4 + 10x_3 + 2x_2 + x_1 &= 0. \mk
   \end{align*}
   The third equation gives $x_2=2x_1$.  After substituting this in
   the remaining equations we get
   \begin{align*}
    x_4 - 2x_3 + x_1 &= 0 \\
    2x_4 -4x_3 + 2x_1 &= 0 \\
    5x_4 + 10x_3 + 5x_1 &= 0. \mk
   \end{align*}
   Here the second equation is twice the first equation so it can be
   ignored.  If we subtract five times the first equation from the
   third equation we find that $x_3=0$.   Given this, the first
   equation becomes $x_4=-x_1$.  We now have
   \[ x = \bbm x_1\\ x_2\\ x_3 \\ x_4 \ebm
        = \bbm x_1\\ 2x_1\\ 0 \\ -x_1 \ebm
        = x_1 \bbm 1 \\ 2 \\ 0 \\ -1 \ebm 
        = x_1 \; a_1.
   \]
   It follows that the vector $a_1$ on its own is the canonical basis
   for $V\cap W$. \mks{2} 

   (Note: this could have been obtained more directly.  From parts~(a)
   and~(b) it is clear that the vector $a_1$ (which is the same as
   $b_1$) lies in $V\cap W$, so the subspace $\R a_1$ is contained in
   $V\cap W$.  If $V\cap W$ were any bigger than this, it would have
   dimension $2$ and so would be the same as $V$ and $W$, so we would
   have $V=W$, which is false because the canonical bases of $V$ and
   $W$ are different.  We must therefore have $V\cap W=\R a_1$.)
 \end{itemize}
\end{solution}



\begin{problem}[2011-12]
 Put 
 \[ v_1 = \bbm  1\\ 1\\ 0\\ 1 \ebm \hspace{2em}
    v_2 = \bbm  2\\ 2\\-1\\ 0 \ebm \hspace{2em}
    u_1 = \bbm  1\\ 2\\ 2\\-1 \ebm \hspace{2em}
    u_2 = \bbm  2\\ 2\\ 2\\-1 \ebm
 \]
 and $V=\spn(v_1,v_2)$ and $W=\ann(u_1,u_2)$.  
 \begin{itemize}
  \item[(a)] Find the canonical basis for $V$. \mrks{4}
  \item[(b)] Find the canonical basis for $W$. \mrks{6}
  \item[(c)] Find the canonical basis for $V+W$. \mrks{5}
  \item[(d)] Find vectors $c_1$ and $c_2$ such that
   $V=\ann(c_1,c_2)$. \mrks{5}
  \item[(e)] Find the canonical basis for $V\cap W$. \mrks{5}
 \end{itemize}
\end{problem}
\begin{solution}
 \begin{itemize}
  \item[(a)] 
   To find the canonical basis for $V$ we perform the following
   row-reduction:
   \[ \left[\begin{array}{ccc}
       & v_1^T & \\\hline 
       & v_2^T & 
      \end{array}\right] \mk
      =
      \bbm 1&1&0&1 \\ 2&2&-1&0 \ebm
      \to 
      \bbm 1&1&0&1 \\ 0&0&-1&-2 \ebm
      \to 
      \bbm 1&1&0&1 \\ 0&0&1&2 \ebm \mk
   \]
   From this we see that the canonical basis is $(a_1,a_2)$ where $a_1$
   and $a_2$ are the transposes of the rows of the above matrix, namely
   \[ a_1 = \bbm 1\\1\\0\\1\ebm \hspace{4em} 
      a_2 = \bbm 0\\0\\1\\2\ebm. \mks{2}
   \]
  \item[(b)]
   Next, $W$ is the set of vectors $x$ satisfying $x.u_1=0$ and
   $x.u_2=0$, or equivalently 
   \begin{align*}
    -x_4 + 2x_3 + 2x_2 + 2x_1 &= 0 \\
    -x_4 + 2x_3 + 2x_2 +  x_1 &= 0 \mks{2}
   \end{align*}
   Solving these in the standard way gives $x_1=0$ and $x_4=2x_3+2x_2$
   with $x_3$ and $x_2$ arbitrary \mk, so 
   \[ x = \bbm 0 \\ x_2 \\ x_3 \\ 2x_3+2x_2 \ebm 
        = x_2 \bbm 0\\1\\0\\2\ebm +
          x_3 \bbm 0\\0\\1\\2\ebm. \mk
   \]
   From this we see that the following matrices form the canonical basis
   for $W$:
   \[ b_1 = \bbm 0\\1\\0\\2\ebm \hspace{4em}
      b_2 = \bbm 0\\0\\1\\2\ebm. \mks{2}
   \]
 
  \item[(c)]
   It follows that $V+W=\spn(a_1,a_2,b_1,b_2)$ \mk, but we can omit $b_2$
   because it is the same as $a_2$.  To make this canonical we perform
   the following row-reduction:
   \[ \left[\begin{array}{ccc}
       & a_1^T & \\\hline 
       & a_2^T & \\\hline
       & b_1^T &
      \end{array}\right] \mks{2}
      =
      \bbm 1&1&0&1 \\ 0&0&1&2 \\ 0&1&0&2 \ebm
      \to 
      \bbm 1&0&0&-1 \\ 0&0&1&2 \\ 0&1&0&2 \ebm
      \to 
      \bbm 1&0&0&-1 \\ 0&1&0&2 \\ 0&0&1&2 \ebm. \mk
   \]
   This shows that the canonical basis for $V+W$ consists of the vectors
   \[
      \bbm 1&0&0&-1\ebm^T \hspace{4em}
      \bbm 0&1&0&2\ebm^T \hspace{4em}
      \bbm 0&0&1&2\ebm^T. \mk
   \]
  \item[(d)] 
   If $x$ is a vector that annihilates the space $V=\spn(v_1,v_2)$, we
   must have $x.v_1=x.v_2=0$, or equivalently
   \begin{align*}
    x_4 + x_2 + x_1 &= 0 \\
    -x_3 + 2x_2 + 2x_1 &= 0.\mk
   \end{align*}
   This gives $x_4=-x_2-x_1$ and $x_3=2x_2+2x_1$ with $x_1$ and $x_2$
   arbitrary, so 
   \[ x = \bbm x_1 \\ x_2 \\ 2x_1+2x_2 \\ -x_1-x_2 \ebm 
        = x_1\bbm 1\\0\\2\\-1\ebm + x_2\bbm 0\\ 1\\ 2\\-1\ebm. \mks{2}
   \]
   The standard methods now tell us that $V=\ann(c_1,c_2)$, 
   where 
   \[ c_1 = \bbm 1\\0\\2\\-1 \ebm \hspace{4em}
      c_2 = \bbm 0\\1\\2\\-1 \ebm. \mks{2}
   \]

  \item[(e)] We now have 
   $V\cap W=\ann(c_1,c_2)\cap\ann(u_1,u_2)=\ann(c_1,c_2,u_1,u_2)$ \mk.
   In other words, $V\cap W$ is the set of solutions to the equations
   $x.c_1=0$ and $x.c_2=0$ and $x.u_1=0$ and $x.u_2=0$, or
   equivalently 
   \begin{align*}
    -x_4+2x_3+x_1 &= 0 \\
    -x_4+2x_3+x_2 &= 0 \\
    -x_4 + 2x_3 + 2x_2 + 2x_1 &= 0 \\
    -x_4 + 2x_3 + 2x_2 +  x_1 &= 0. \mk
   \end{align*}
   Subtracting the first two equations gives $x_1=x_2$, and
   subtracting the last two gives $x_1=0$, so we also have $x_2=0$.
   Given this, everything else reduces easily to the equation
   $x_4=2x_3$ \mk.  We thus have 
   \[ x = \bbm 0\\0\\x_3\\2x_3\ebm = x_3 \bbm 0\\0\\1\\2\ebm =
       x_3a_2. \mk
   \]
   It follows that the vector $a_2$ on its own is the canonical basis
   for $V\cap W$. \mk 

   (Note: this could have been obtained more directly.  From parts~(a)
   and~(b) it is clear that the vector $a_2$ (which is the same as
   $b_2$) lies in $V\cap W$, so the subspace $\R a_2$ is contained in
   $V\cap W$.  If $V\cap W$ were any bigger than this, it would have
   dimension $2$ and so would be the same as $V$ and $W$, so we would
   have $V=W$, which is false because the canonical bases of $V$ and
   $W$ are different.  We must therefore have $V\cap W=\R a_2$.)
 \end{itemize}
\end{solution}

\begin{problem}[Mock 1]
 Let $V$ be the set of all vectors in $\R^5$ of the form 
 \[ x = \bbm p & p+q & 0 & -p-q & -p \ebm^T \]
 (where $p,q\in\R$ are arbitrary).  Also, put 
 \[ W = \{x\in\R^5\st x_1+x_2+x_3+3x_4=x_1+x_2+x_3-x_4+2x_5=0\}. \]
 \begin{itemize}
  \item[(a)] Find the canonical basis for $V$. \mrks{5}
  \item[(b)] Find the canonical basis for $W$. \mrks{6}
  \item[(c)] Find the canonical basis for $V+W$. \mrks{5}
  \item[(d)] Write down a formula relating the dimensions of
   $V$, $W$, $V+W$ and $V\cap W$, and use it to determine
   $\dim(V\cap W)$. \mrks{3}
  \item[(e)] Find a basis for $V\cap W$ \mrks{6}.
 \end{itemize}
\end{problem}
\begin{solution}
 \begin{itemize}
  \item[(a)] The general element of $V$ can be written as
   $x=pu_1+qu_2$, where 
   \[ u_1 = \bbm 1&1&0&-1&-1\ebm^T \hspace{5em}
      u_2 = \bbm 0&1&0&-1&0\ebm^T,
   \] 
   so $V=\spn(u_1,u_2)$ \mks{2}.  Using the row-reduction
   \[ \left[\begin{array}{ccc}
       & u_1^T & \\ \hline
       & u_2^T & \end{array}\right] = 
       \bbm 1&1&0&-1&-1 \\ 0&1&0&-1&0 \ebm \to
       \bbm 1&0&0&0&-1 \\ 0&1&0&-1&0 \ebm \mks{2}
   \]
   we see that the vectors 
   \[ v_1 = \bbm 1&0&0& 0&-1\ebm^T \qquad\text{ and } \qquad
      v_2 = \bbm 0&1&0&-1&0\ebm^T
   \] 
   form the canonical basis for $V$ \mk.
  \item[(b)] To find the canonical basis for $W$ we write the defining
   equations with the variables in descending order, and find the
   solution in a form that expresses higher-numbered variables in
   terms of lower-numbered ones.  The equations are 
   \begin{align*}
    3x_4+x_3+x_2+x_1 &= 0 \\
    2x_5-x_4+x_3+x_2+x_1 &= 0. \mk
   \end{align*}
   The first equation gives
   $x_4=-\tfrac{1}{3}x_3-\tfrac{1}{3}x_2-\tfrac{1}{3}x_1$.  On the
   other hand, we can subtract the two equation to get $2x_5-4x_4=0$
   or in other words
   $x_5=2x_4=-\tfrac{2}{3}x_3-\tfrac{2}{3}x_2-\tfrac{2}{3}x_1$ \mks{2}.  We
   now have
   \[ x = \bbm x_1\\ x_2\\ x_3\\ x_4\\ x_5 \ebm =
       \bbm x_1\\x_2\\x_3\\
            -\tfrac{1}{3}x_3-\tfrac{1}{3}x_2-\tfrac{1}{3}x_1 \\
            -\tfrac{2}{3}x_3-\tfrac{2}{3}x_2-\tfrac{2}{3}x_1 \ebm =
       x_1 \bbm 1\\0\\0\\-1/3\\-2/3\ebm + 
       x_2 \bbm 0\\1\\0\\-1/3\\-2/3\ebm + 
       x_3 \bbm 0\\0\\1\\-1/3\\-2/3\ebm \mks{2}.
   \]
   It follows that the vectors
   \[ w_1 = \bbm 1\\0\\0\\-1/3\\-2/3\ebm \quad\text{ and }\quad
      w_2 = \bbm 0\\1\\0\\-1/3\\-2/3\ebm \quad\text{ and }\quad
      w_3 = \bbm 0\\0\\1\\-1/3\\-2/3\ebm 
   \]
   form the canonical basis for $W$ \mk.
  \item[(c)]
   It now follows that $V+W=\spn(v_1,v_2,w_1,w_2,w_3)$ \mk. 
   To make this canonical we perform the following row-reduction:
   \[ \left[\begin{array}{ccc}
       & v_1^T & \\\hline 
       & v_2^T & \\\hline
       & w_1^T & \\\hline
       & w_2^T & \\\hline
       & w_3^T & 
      \end{array}\right] \mk
      =
      \bbm
       1 & 0 & 0 &    0 &   -1 \\
       0 & 1 & 0 &   -1 &    0 \\
       1 & 0 & 0 & -1/3 & -2/3 \\
       0 & 1 & 0 & -1/3 & -2/3 \\
       0 & 0 & 1 & -1/3 & -2/3
      \ebm
      \to
      \bbm
       1 & 0 & 0 &    0 &   -1 \\
       0 & 1 & 0 &   -1 &    0 \\
       0 & 0 & 0 & -1/3 &  1/3 \\
       0 & 1 & 0 & -1/3 & -2/3 \\
       0 & 0 & 1 & -1/3 & -2/3
      \ebm
      \to
      \bbm
       1 & 0 & 0 &    0 &   -1 \\
       0 & 1 & 0 &   -1 &    0 \\
       0 & 0 & 0 & -1/3 &  1/3 \\
       0 & 0 & 0 &  1/3 & -1/3 \\
       0 & 0 & 1 & -1/3 & -2/3
      \ebm
    \] \[
      \to
      \bbm
       1 & 0 & 0 &    0 &   -1 \\
       0 & 1 & 0 &   -1 &    0 \\
       0 & 0 & 0 &    1 &   -1 \\
       0 & 0 & 0 & -1/3 &  1/3 \\
       0 & 0 & 1 & -1/3 & -2/3
      \ebm
      \to
      \bbm
       1 & 0 & 0 &    0 &   -1 \\
       0 & 1 & 0 &    0 &   -1 \\
       0 & 0 & 0 &    1 &   -1 \\
       0 & 0 & 0 &    0 &    0 \\
       0 & 0 & 1 &    0 &   -1
      \ebm
      \to
      \bbm
       1 & 0 & 0 &    0 &   -1 \\
       0 & 1 & 0 &    0 &   -1 \\
       0 & 0 & 1 &    0 &   -1 \\
       0 & 0 & 0 &    1 &   -1 \\
       0 & 0 & 0 &    0 &    0
      \ebm \mks{2}
   \]
   This shows that the canonical basis for $V+W$ consists of the vectors
   \[
      \bbm 1\\0\\0\\0\\-1\ebm \hspace{4em}
      \bbm 0\\1\\0\\0\\-1\ebm \hspace{4em}
      \bbm 0\\0\\1\\0\\-1\ebm \hspace{4em}
      \bbm 0\\0\\0\\1\\-1\ebm \mk.
   \]
  \item[(d)] The general dimension formula says that
   $\dim(V+W)+\dim(V\cap W)=\dim(V)+\dim(W)$ \mk, so
   $\dim(V\cap W)=\dim(V)+\dim(W)-\dim(V+W)$.  The dimension of a
   subspace is just the number of vectors in the canonical basis, so
   $\dim(V)=2$ and $\dim(W)=3$ and $\dim(V+W)=4$ \mk.  This gives
   $\dim(V\cap W)=2+3-4=1$ \mk. 
  \item[(e)] This problem can be done in several different ways, but
   the following is probably the easiest.  Any element $x\in V\cap W$
   must have the form  
   \[ x = \bbm x_1&x_2&x_3&x_4&x_5\ebm^T
        = \bbm p & p+q & 0 & -p-q & -p \ebm^T
   \]
   and must satisfy 
   \[ x_1+x_2+x_3+3x_4=x_1+x_2+x_3-x_4+2x_5=0. \mks{2} \]
   This reduces to 
   \[ p+(p+q)+0+3(-p-q) = p+(p+q)+0-(-p-q)+2(-p)=0 \]
   or in other words $-p-2q=p+2q=0$, so $p=-2q$ \mks{2}.  This in turn gives 
   \[ x = \bbm p \\ p+q \\ 0 \\ -p-q \\ -p \ebm
        = \bbm -2q \\ -q \\ 0 \\ q \\ 2q \ebm 
        = q \bbm -2 \\ -1 \\ 0 \\ 1 \\ 2\ebm \mk.
   \]
   This shows that the vector $\bbm -2&-1&0&1&2\ebm$ is (on its own) a
   basis for $V\cap W$ \mk.  (To get the canonical basis, we would divide
   this vector by $-2$.) 
 \end{itemize}
\end{solution}

\begin{problem}[Mock 2]
 Consider the vectors
 \[ a_1 = \bbm 1\\2\\3\\1\ebm \hspace{3em}
    a_2 = \bbm 2\\4\\-2\\-6 \ebm \hspace{3em}
    a_3 = \bbm 3\\6\\0\\-6 \ebm \hspace{3em}
    p = \bbm 11\\22\\22\\0\ebm \hspace{3em}
    q = \bbm 222\\-111\\0\\0\ebm
 \]
 Put $V=\spn(a_1,a_2,a_3)$ and $W=\ann(a_1,a_2,a_3)$.
 \begin{itemize}
  \item[(a)] Find the canonical basis for $V$. \mrks{4}
  \item[(b)] Find the canonical basis for $W$. \mrks{8}
  \item[(c)] Find the canonical basis for $V+W$. \mrks{5}
  \item[(d)] Use the dimension formula to determine $V\cap W$. \mrks{4}
  \item[(e)] Prove that $p\in V$ and $q\in W$. \mrks{4}
 \end{itemize}
\end{problem}
\begin{solution}
 \begin{itemize}
  \item[(a)] Using the row-reduction
   \[ \left[\begin{array}{ccc}
       & a_1^T & \\ \hline
       & a_2^T & \\ \hline
       & a_3^T & \end{array}\right] = 
       \bbm 1&2&3&1 \\
            2&4&-2&-6 \\
            3&6&0&-6 \ebm \to
       \bbm 1&2&3&1 \\
            0&0&-8&-8 \\
            0&0&-9&-9 \ebm \to
       \bbm 1&2&0&-2 \\
            0&0&1&1 \\
            0&0&0&0 \ebm \mks{2}
   \]
   we see that the vectors 
   \[ v_1 = \bbm 1&2&0&-2\ebm^T \qquad\text{ and } \qquad
      v_2 = \bbm 0&0&1&1\ebm^T
   \] 
   form the canonical basis for $V$ \mks{2}.
  \item[(b)] To find the canonical basis for $W$ we write the 
   equations $a_3.x=0$, $a_2.x=0$ and $a_1.x=0$ with the variables in
   descending order, and find the solution in a form that expresses
   higher-numbered variables in terms of lower-numbered ones.  The
   equations are  
   \begin{align*}
    -6x_4+6x_2+3x_1 &= 0 \\
    -6x_4-2x_3+4x_2+2x_1 &= 0 \\
    x_4+3x_3+2x_2+x_1 &= 0 \mks{2}.
   \end{align*}
   We can write down the matrix of coefficients and row-reduce it as
   follows:
   \[ \bbm -6 &  0 &  6 &  3 \\
           -6 & -2 &  4 &  2 \\
            1 &  3 &  2 &  1 \ebm \to
      \bbm  0 & 18 & 18 &  9 \\
            0 & 16 & 16 &  8 \\
            1 &  3 &  2 &  1 \ebm \to
      \bbm  1 &  3 &  2 &  1 \\
            0 &  1 &  1 & 1/2 \\
            0 &  0 &  0 &  0 \ebm \to
      \bbm  1 &  0 & -1 &-1/2 \\
            0 &  1 &  1 & 1/2 \\
            0 &  0 &  0 &  0 \ebm \mks{2}
   \]
   It follows that our original equations are equivalent to the
   equations $x_4-x_2-x_1/2=0$ and $x_3+x_2+x_1/2=0$.  Using this we
   get 
   \[ x = 
      \bbm x_1\\ x_2\\ x_3\\ x_4 \ebm = 
      \bbm x_1\\ x_2\\-x_1/2-x_2\\x_1/2+x_2 \ebm = 
      x_1 \bbm 1\\ 0 \\ -1/2 \\ 1/2 \ebm + 
      x_2 \bbm 0\\ 1 \\ -1 \\ 1\ebm \mks{2}. 
   \]
   It follows that the vectors
   \[ w_1 = \bbm 1\\ 0 \\ -1/2 \\ 1/2 \ebm \qquad \text{ and } \qquad
      w_2 = \bbm 0\\ 1 \\ -1 \\ 1\ebm
   \]
   form the canonical basis for $W$ \mks{2}.
  \item[(c)] We now have $V+W=\spn(v_1,v_2,w_1,w_2)$.  To find the
   canonical basis for this, we perform the row-reduction
   \[ \left[\begin{array}{ccc}
       & v_1^T & \\\hline 
       & v_2^T & \\\hline
       & w_1^T & \\\hline
       & w_2^T & 
      \end{array}\right] \mk
      =
      \bbm
        1 &  2 &    0 &  -2 \\
        0 &  0 &    1 &   1 \\
        1 &  0 & -1/2 & 1/2 \\
        0 &  1 &   -1 &   1
      \ebm 
      \to 
      \bbm
        1 &  2 &    0 &  -2 \\
        0 &  0 &    1 &   1 \\
        0 & -2 & -1/2 & 3/2 \\
        0 &  1 &   -1 &   1
      \ebm 
      \to 
    \] \[
      \bbm
        1 &  0 &   -2 &  -4 \\
        0 &  0 &    1 &   1 \\
        0 &  0 & -5/2 & 5/2 \\
        0 &  1 &   -1 &   1
      \ebm 
      \to 
      \bbm
        1 &  0 &    0 &  -2 \\
        0 &  1 &    0 &   2 \\
        0 &  0 &    1 &   1 \\
        0 &  0 &    0 &   5 
      \ebm 
      \to 
      \bbm
        1 &  0 &    0 &   0 \\
        0 &  1 &    0 &   0 \\
        0 &  0 &    1 &   0 \\
        0 &  0 &    0 &   1
      \ebm \mks{2}
   \]
   This shows that the canonical basis for $V+W$ consists of the
   standard basis vectors $e_1,e_2,e_3,e_4$ and so $V+W$ is all of
   $\R^4$ \mks{2}. 
  \item[(d)] The general dimension formula says that
   $\dim(V+W)+\dim(V\cap W)=\dim(V)+\dim(W)$ \mk, so
   $\dim(V\cap W)=\dim(V)+\dim(W)-\dim(V+W)$.  The dimension of a
   subspace is just the number of vectors in the canonical basis, so
   $\dim(V)=\dim(W)=2$ and $\dim(V+W)=4$ \mks{2}.  This gives
   $\dim(V\cap W)=0$ and so $V\cap W=\{0\}$ \mk. 
  \item[(e)] By inspection we have $p=11v_1+22v_2$.  As $v_1$ and
   $v_2$ span $V$ it follows that $p\in V$ \mks{2}.  Similarly, we have  
   $q=222w_1-111w_2$ and it follows that $q\in W$ \mks{2}.
 \end{itemize}
\end{solution}

%%%%%%%%%%%%%%%%%%%%%%%%%%%%%%%%%%%%%%%%%%%%%%%%%%%%%%%%%%%%%%%%%%%%%%

\section{Question 5}
\setcounter{probcounter}{0}

\begin{problem}[2013-14]
 Consider the matrix
 $\displaystyle 
     A= \frac{1}{27}\bbm
     9&8&-8\\
     8&23&0\\
     -8&0&-5
     \ebm
 $.
 \begin{itemize}
  \item[(a)] State the main results about eigenvalues and eigenvectors
   of symmetric matrices. \mrks{4}
  \item[(b)] Show that the following are eigenvectors of $A$: \mrks{2}
   \[ u_1 = \bbm 7 \\ -4 \\ -4 \ebm 
      \qquad
      u_2 = \bbm 4 \\ -1 \\ 8 \ebm.
   \]
  \item[(c)] Find an orthogonal matrix $U$ and a diagonal matrix $D$
   such that $A=UDU^T$. \mrks{9} \\
   \textbf{Hint:} For any square matrix, the sum of the eigenvalues is 
   the same as the sum of the diagonal entries.  Because of this, you
   do not need to calculate the characteristic polynomial.
  \item[(d)] Find $\lim_{n\to \infty}A^n$. \mrks{5}
 \end{itemize}
\end{problem}
\begin{solution}
 \begin{itemize}
  \item[(a)] For any real symmetric matrix, all eigenvalues are real \mks{2},
   and eigenvectors corresponding to different eigenvectors are
   orthogonal \mks{2}.  (For the second part, two marks will also be
   given for saying that there is an orthonormal basis of
   eigenvectors.)
  \item[(b)] We have 
   \begin{align*}
    Au_1 &= \frac{1}{27} \bbm 9&8&-8\\ 8&23&0\\ -8&0&-5 \ebm
                         \bbm 7 \\ -4 \\ -4 \ebm 
          = \frac{1}{27} \bbm 63 \\ -36 \\ 36 \ebm 
          = \frac{1}{3} \bbm 7 \\ -4 \\ 4 \ebm 
          = \frac{1}{3} u_1 \\
    Au_2 &= \frac{1}{27} \bbm 9&8&-8\\ 8&23&0\\ -8&0&-5 \ebm
                         \bbm 4 \\ -1 \\ 8 \ebm 
          = \frac{1}{27} \bbm -36 \\ 9 \\ -72 \ebm 
          = \frac{1}{3} \bbm -4 \\ 1 \\ -8 \ebm 
          = -\frac{1}{3} u_2.
   \end{align*}
   This shows that $u_1$ is an eigenvector of eigenvalue $\lm_1=1/3$,
   and $u_2$ is an eigenvector of eigenalue $\lm_2=-1/3$. \mks{2}
  \item[(c)] If we let $\lm_3$ denote the third eigenvalue, then the
   hint tells us that 
   \[ \lm_1+\lm_2+\lm_3=\frac{1}{27}(9+23-5)=1. \]
   As $\lm_1=1/3$ and $\lm_2=-1/3$, this just gives $\lm_3=1$ \mks{2}.  To
   find the corresponding eigenvector $u_3$, we row-reduce the matrix $A-I$: 
   \[
    \bbm
    -2/3&8/27&-8/27\\
    8/27&-4/27&0\\
    -8/27&0&-32/27\\
    \ebm
    \to
    \bbm
    1&-4/9&4/9\\
    8/27&-4/27&0\\
    -8/27&0&-32/27\\
    \ebm
    \to
   \] \[
    \bbm
    1&-4/9&4/9\\
    0&-4/243&-32/243\\
    0&-32/243&-256/243\\
    \ebm
    \to
    \bbm
    1&-4/9&4/9\\
    0&1&8\\
    0&-32/243&-256/243\\
    \ebm
    \to
    \bbm
    1&0&4\\
    0&1&8\\
    0&0&0\\
    \ebm \mks{2}
   \]
   This tells us that $u_3$ must have the form $\bbm x&y&z\ebm^T$ with
   $x+4z=y+8z=0$, so $u_3=z\bbm -4&-8&1\ebm^T$ for some arbitrary
   nonzero scalar $z$.  We choose $z=-1$ giving
   $u_3=\bbm 4&8&-1\ebm^T$. \mk

   Next, as $A$ is symmmetric and the eigenvectors $u_k$ have
   different eigenvalues, they are automatically orthogonal.  (It is
   also not hard to check that directly, by calculating that
   $u_1.u_2=u_1.u_3=u_2.u_3=0$.)  However, they are not orthonormal,
   because 
   \begin{align*}
    \|u_1\|^2 &= 7^2 + (-4)^2 + (-4)^2 = 81 \\
    \|u_2\|^2 &= 4^2 + (-1)^2 + 8^2 = 81 \\
    \|u_3\|^2 &= 4^2 + 8^2 + (-1)^2 = 81,
   \end{align*}
   so $\|u_1\|=\|u_2\|=\|u_3\|=\sqrt{81}=9$ \mks{2}.  It follows that the
   vectors $u_k/9$ form an orthonormal basis.  We therefore obtain
   an orthogonal diagonalisation $A=UDU^T$ with 
   \begin{align*}
    U &= \left[ \begin{array}{c|c|c}
          \frac{u_1}{9} & \frac{u_2}{9} & \frac{u_3}{9}
         \end{array}\right] 
       = \frac{1}{9} \bbm 
          7&4&4\\
          -4&-1&8\\
          -4&8&-1\\
         \ebm \\
    D &= \bbm \lm_1 & 0 & 0 \\ 0 & \lm_2 & 0 \\ 0 & 0 & \lm_3 \ebm
       = \bbm 1/3 & 0 & 0 \\ 0 & -1/3 & 0 \\ 0 & 0 & 1 \ebm. \mks{2}
   \end{align*}
   Recall also that $U$ is an orthogonal matrix, so $U^{-1}=U^T$.
  \item[(d)] We now have $A^n=UD^nU^T$ for all $n\geq 0$ \mk.  Now put 
   \[ D^\infty = \lim_{n\to\infty} D^n = \lim_{n\to\infty}
      \bbm (1/3)^n & 0 & 0 \\ 0 & (-1/3)^n & 0 \\ 0 & 0 & 1 \ebm
      =
      \bbm 0 & 0 & 0 \\ 0 & 0 & 0 \\ 0 & 0 & 1 \ebm. \mks{2}
   \]
   It follows that 
   \begin{align*}
    \lim_{n\to\infty} A^n 
     &= U D^\infty U^T \mk \\
     &= \frac{1}{81} 
        \bbm 7&4&4\\ -4&-1&8\\ -4&8&-1 \ebm
        \bbm 0&0&0 \\ 0&0&0 \\ 0&0&1 \ebm 
        \bbm 7&-4&-4 \\ 4&-1&8 \\ 4&8&-1 \ebm \\
     &= \frac{1}{81} 
        \bbm 0&0&4\\ 0&0&8\\ 0&0&-1 \ebm
        \bbm 7&-4&-4 \\ 4&-1&8 \\ 4&8&-1 \ebm 
      = \frac{1}{81}
        \bbm 16 & 32 & -4 \\ 32 & 64 & -8 \\ -4 & -8 & 1 \ebm \mk
   \end{align*}
 \end{itemize}
\end{solution}

\begin{problem}[2012-13 resit]
 Consider the matrix
 $\displaystyle M=\bbm 0&2&0&0 \\ 2&0&3&0 \\ 0&3&0&2 \\ 0&0&2&0 \ebm$
 \begin{itemize}
  \item[(a)] Find the eigenvalues and eigenvectors of $M$.  To
   simplify the calculations you may use the following fact: for this
   particular matrix, if $\bbm w&x&y&z\ebm^T$ is an eigenvector of
   eigenvalue $\lm$, then $\bbm w&-x&y&-z\ebm^T$ is an eigenvector of
   eigenvalue $-\lm$.
   \mrks{14}
  \item[(b)] Find an orthogonal matrix $P$ and a diagonal matrix $D$
   such that $M=PDP^T$. \mrks{6}
 \end{itemize}
\end{problem}
\begin{solution}
 \begin{itemize}
  \item[(a)] We first find the characteristic polynomial:
   \begin{align*}
    \chi_M(t) &= \det\bbm -t&2&0&0 \\ 2&-t&3&0 \\ 0&3&-t&2 \\ 0&0&2&-t \ebm\\
    &= -t\det\bbm -t&3&0 \\ 3&-t&2 \\ 0&2&-t \ebm 
       -2 \det\bbm 2&3&0 \\ 0&-t&2 \\ 0&2&-t \ebm \mks{2} \\
    \det\bbm -t&3&0 \\ 3&-t&2 \\ 0&2&-t \ebm &= 
       -t(t^2-4) -3 (-3t-0) = -t^3+13t \\
    \det\bbm 2&3&0 \\ 0&-t&2 \\ 0&2&-t \ebm &=
       2(t^2-4) - 3(0) = 2t^2 - 8 \\
    \chi_M(t) &= -t(-t^3+13t) - 2(2t^2-8) = t^4 - 17t^2 + 16 \mks{2} \\
      &= (t^2-1)(t^2-16) = (t-1)(t+1)(t-4)(t+4). \mk
   \end{align*}
   Thus, the eigenvalues are 
   \[ \lm_1 =  1 \hspace{3em}
      \lm_2 = -1 \hspace{3em}
      \lm_3 =  4 \hspace{3em}
      \lm_4 = -4. \mk
   \]

   To find an eigenvector $u_1$ of eigenvalue $1$, we row-reduce $M-I$:
   \[
    \bbm
    -1&2&0&0\\
    2&-1&3&0\\
    0&3&-1&2\\
    0&0&2&-1\\
    \ebm
    \to
    \bbm
    1&-2&0&0\\
    0&3&3&0\\
    0&3&-1&2\\
    0&0&2&-1\\
    \ebm
    \to
    \bbm
    1&0&2&0\\
    0&1&1&0\\
    0&0&-4&2\\
    0&0&2&-1\\
    \ebm
    \to
    \bbm
    1&0&0&1\\
    0&1&0&1/2\\
    0&0&1&-1/2\\
    0&0&0&0\\
    \ebm.
   \]
   We conclude that $u_1=\bbm w&x&y&z\ebm^T$ with $w+z=x+z/2=y-z/2=0$.
   We can take $z=-2$ giving $u_1=\bbm 2&1&-1&-2 \ebm^T$ \mks{3}.  The
   hint now tells us that the vector $u_2=\bbm 2&-1&-1&2\ebm$ is an
   eigenvector of eigenvalue $-1$. \mk

   To find an eigenvector $u_3$ of eigenvalue $4$, we row-reduce $M-4I$:
   \[
    \bbm
    -4&2&0&0\\
    2&-4&3&0\\
    0&3&-4&2\\
    0&0&2&-4\\
    \ebm
    \to
    \bbm
    1&-1/2&0&0\\
    0&-3&3&0\\
    0&3&-4&2\\
    0&0&2&-4\\
    \ebm
    \to
    \bbm
    1&0&-1/2&0\\
    0&1&-1&0\\
    0&0&-1&2\\
    0&0&2&-4\\
    \ebm
    \to
    \bbm
    1&0&0&-1\\
    0&1&0&-2\\
    0&0&1&-2\\
    0&0&0&0\\
    \ebm
   \]
   We conclude that $u_3=\bbm w&x&y&z\ebm^T$ with $w-z=x-2z=y-2z=0$.  We can
   take $z=1$ giving $u_3=\bbm 1&2&2&1 \ebm^T$ \mks{3}.  The
   hint now tells us that the vector $u_4=\bbm 1&-2&2&-1\ebm$ is an
   eigenvector of eigenvalue $-4$. \mk
  \item[(b)] We now need to find an orthonormal basis for $\R^4$
   consisting of eigenvectors for $M$.  As $M$ is symmetric and the
   eigenvalues $\lm_i$ are distinct, the eigenvectors $u_i$ are
   automatically orthogonal.  Alternatively, it is easy to check
   directly that 
   \[ u_1.u_2=u_1.u_3=u_1.u_4=u_2.u_3=u_2.u_4=u_3.u_4=0. \mk \]
   However, they are not unit vectors, so they do not form an
   orthonormal basis.  Indeed, we have 
   \[ \|u_1\| = \sqrt{4+1+1+4} = \sqrt{10}, \]
   and similarly $\|u_2\|=\|u_3\|=\|u_4\|=\sqrt{10}$ \mk.  Thus, the
   vectors $v_i=u_i/\sqrt{10}$ form an orthonormal basis for 
   $\R^4$ consisting of eigenvectors for $M$. \mk  We now put
   \begin{align*}
    P &= [v_1|v_2|v_3|v_4] 
       = \frac{1}{\sqrt{10}}
          \bbm
          2&2&1&1\\
          1&-1&2&-2\\
          -1&-1&2&2\\
          -2&2&1&-1\\
          \ebm \mk \\ 
    D &= \bbm \lm_1 & 0 & 0 & 0 \\
              0 & \lm_2 & 0 & 0 \\
              0 & 0 & \lm_3 & 0 \\
              0 & 0 & 0 & \lm_4 \ebm
       = \bbm 1 &  0 & 0 &  0 \\
              0 & -1 & 0 &  0 \\
              0 &  0 & 4 &  0 \\
              0 &  0 & 0 & -4 \ebm. \mk
   \end{align*}
   The general theory tells us that $P$ is an orthogonal matrix (so
   $P^{-1}=P^T$) and that $M=PDP^{-1}=PDP^T$. \mk
 \end{itemize}
\end{solution}

\begin{problem}[2012-13]
 Consider the matrix
 $\displaystyle M=\bbm 8&2&2 \\ 2&-4&5 \\ 2&5&-4 \ebm$.
 \begin{itemize}
  \item[(a)] Find the eigenvalues and eigenvectors of $M$. \mrks{14}
  \item[(b)] Find an orthogonal matrix $P$ and a diagonal matrix $D$
   such that $M=PDP^T$. \mrks{6}
 \end{itemize}
\end{problem}
\begin{solution}
 \begin{itemize}
  \item[(a)] We first find the characteristic polynomial:
   \begin{align*}
    \chi_M(t) &= \det\bbm 8-t&2&2 \\ 2&-4-t&5 \\ 2&5&-4-t \ebm \\
    &= (8-t)\det\bbm -4-t&4 \\ 5&-4-t\ebm 
       -2 \det\bbm 2&5 \\ 2&-4-t \ebm 
       +2 \det\bbm 2&-4-t \\ 2&5 \ebm \mks{2} \\
    \det\bbm -4-t&5 \\ 5&-4-t\ebm &= (-4-t)^2 - 25 = t^2+8t-9 \\ 
    \det\bbm 2&5 \\ 2&-4-t \ebm &= -8-2t-10 = -2t-18 \\
    \det\bbm 2&-4-t \\ 2&5 \ebm &= 10-(-8-2t) = 2t+18 \\
    \chi_M(t) &= (8-t)(t^2+8t-9)+4t+36+4t+36 
               = -t^3+81t \mks{2} = -t(t-9)(t+9).
   \end{align*}
   Thus, the eigenvalues are $\lm_1=0$, $\lm_2=9$ and $\lm_3=-9$.\mk

   To find an eigenvector $u_1$ of eigenvalue $0$, we row-reduce $A$:
   \[
    \bbm
    8&2&2\\
    2&-4&5\\
    2&5&-4\\
    \ebm
    \to
    \bbm
    1&1/4&1/4\\
    2&-4&5\\
    2&5&-4\\
    \ebm
    \to
    \bbm
    1&1/4&1/4\\
    0&-9/2&9/2\\
    0&9/2&-9/2\\
    \ebm
    \to
    \bbm
    1&1/4&1/4\\
    0&1&-1\\
    0&9/2&-9/2\\
    \ebm
    \to
    \bbm
    1&0&1/2\\
    0&1&-1\\
    0&0&0\\
    \ebm.
   \]
   We conclude that $u_1=\bbm x&y&z\ebm^T$ with $x+z/2=y-z=0$.  We can
   take $z=2$ giving $u_1=\bbm -1&2&2 \ebm^T$ \mks{3}.

   To find an eigenvector $u_2$ of eigenvalue $9$, we row-reduce $A-9I$:
   \[
    \bbm
    -1&2&2\\
    2&-13&5\\
    2&5&-13\\
    \ebm
    \to
    \bbm
    1&-2&-2\\
    0&-9&9\\
    0&9&-9\\
    \ebm
    \to
    \bbm
    1&-2&-2\\
    0&1&-1\\
    0&9&-9\\
    \ebm
    \to
    \bbm
    1&0&-4\\
    0&1&-1\\
    0&0&0\\
    \ebm
   \]
   We conclude that $u_2=\bbm x&y&z\ebm^T$ with $x-4z=y-z=0$.  We can
   take $z=1$ giving $u_2=\bbm 4&1&1 \ebm^T$ \mks{3}.
   
   To find an eigenvector $u_3$ of eigenvalue $-9$, we row-reduce
   $A+9I$:
   \[
    \bbm
    17&2&2\\
    2&5&5\\
    2&5&5\\
    \ebm
    \to
    \bbm
    17&2&2\\
    1&5/2&5/2\\
    2&5&5\\
    \ebm
    \to
    \bbm
    0&-81/2&-81/2\\
    1&5/2&5/2\\
    0&0&0\\
    \ebm
    \to
    \bbm
    0&1&1\\
    1&5/2&5/2\\
    0&0&0\\
    \ebm
    \to
    \bbm
    1&0&0\\
    0&1&1\\
    0&0&0\\
    \ebm
   \]
   We conclude that $u_3=\bbm x&y&z\ebm^T$ with $x=y+z=0$.  We can
   take $z=-1$ giving $u_3=\bbm 0&1&-1 \ebm^T$ \mks{3}.
  \item[(b)] We now need to find an orthonormal basis for $\R^3$
   consisting of eigenvectors for $M$.  As $M$ is symmetric and the
   eigenvalues $\lm_i$ are distinct, the eigenvectors $u_i$ are
   automatically orthogonal.  (Alternatively, it is easy to check
   directly that $u_1.u_2=u_1.u_3=u_2.u_3=0$.)  However, they are not
   unit vectors, so they do not form an orthonormal basis.  Indeed, we
   have 
   \begin{align*}
    \|u_1\| &= \sqrt{1+4+4} = 3 \\
    \|u_2\| &= \sqrt{16+1+1} = 3\sqrt{2} \\
    \|u_3\| &= \sqrt{0+1+1} = \sqrt{2} \mks{2}.
   \end{align*}
   The vectors $v_i=u_i/\|u_i\|$ form an orthonormal basis for
   $\R^3$ consisting of eigenvectors for $M$.  We now put
   \begin{align*}
    P &= [v_1|v_2|v_3] 
       = \bbm
          -1/3 & 2\sqrt{2}/3 &  0          \\
           2/3 &  \sqrt{2}/6 &  \sqrt{2}/2 \\
           2/3 &  \sqrt{2}/6 & -\sqrt{2}/2
         \ebm \mks{2} \\
    D &= \bbm \lm_1 & 0 & 0 \\ 0 & \lm_2 & 0 \\ 0 & 0 & \lm_3 \ebm
       = \bbm 0 & 0 & 0 \\ 0 & 9 & 0 \\ 0 & 0 & -9 \ebm. \mks{2}
   \end{align*}
   The general theory tells us that $P$ is an orthogonal matrix (so
   $P^{-1}=P^T$) and that $M=PDP^{-1}=PDP^T$.
 \end{itemize}
\end{solution}


\begin{problem}[2011-12 resit]
 Consider the matrix
 \[ A = \bbm 1&1&1&4 \\ 1&1&1&4 \\ 1&1&1&4 \\ 4&4&4&1 \ebm. \]
 You may assume that $\det(tI-A)=t^4-4t^3-45t^2$.
 \begin{itemize}
  \item[(a)] Find the eigenvalues of $A$. \mrks{2}
  \item[(b)] Find an eigenvector $u_1$ of eigenvalue $0$ with
   $\|u_1\|=1$.  \mrks{3}
  \item[(c)] Find another eigenvector $u_2$ of eigenvalue $0$ with
   $u_1.u_2=0$ and $\|u_2\|=1$. \mrks{4} 
  \item[(d)] Find an orthonormal basis of $\R^4$ consisting of
   eigenvectors of $A$. \mrks{10}
  \item[(e)] Find an orthogonal matrix $P$ and a diagonal matrix $D$
   such that $A=PDP^T$. \mrks{3}
  \item[(f)] Express the quadratic form
   \[ Q = w^2+x^2+y^2+z^2 + 2(wx+wy+xy) + 8(wz+xz+yz) \] 
   as $Q=F^2-G^2$, where $F$ and $G$ are linear forms.  \mrks{3}
 \end{itemize}
\end{problem}
\begin{solution}
 \begin{itemize}
  \item[(a)] The characteristic polynomial factorises as
   $\chi_A(t)=t^2(t^2-4t-45)=t^2(t-9)(t+5)$, so the eigenvalues are
   $0$, $9$ and $-5$. \mks{2}
  \item[(b)] It is easy to see that a vector $\bbm w&x&y&z\ebm^T$ is
   an eigenvector of eigenvalue $0$ if and only if
   $w+x+y+4z=4w+4x+4y+z=0$ \mk, which reduces to $w+x+y=0$ and
   $z=0$ \mk.  One solution is $\bbm 1&-1&0&0\ebm^T$.  To make this a
   unit vector, we need to divide by $\sqrt{2}$, giving 
   \[ u_1 = \bbm \frac{1}{\sqrt{2}} & -\frac{1}{\sqrt{2}} & 0 & 0 \ebm^T.
      \mk
   \]
  \item[(c)] We again need $u_2=\bbm w&x&y&0\ebm^T$ with $w+x+y=0$\mk.
   We also want this to be orthogonal to $u_1$, so we must have
   $w=x$ \mk.   The equation $w+x+y=0$ now becomes $y=-2x$, so
   $u_2=\bbm x&x&-2x&0\ebm^T$.  This gives 
   \[ \|u_2\|^2 = x^2+x^2+4x^2+0 = 6x^2 \mk. \]
   We want $u_2$ to be a unit vector, so we must have $x=1/\sqrt{6}$,
   giving 
   \[ u_2 = \bbm \frac{1}{\sqrt{6}} & -\frac{1}{\sqrt{6}}
                  & \frac{-2}{\sqrt{6}} & 0 \ebm^T \mk.
   \]
   (\textbf{Any approach to~(b) and~(c) that ends up with two orthonormal 
   vectors in the kernel will be given 7/7.})
  \item[(d)]
   Next, to find an eigenvector of eigenvalue $9$ we row-reduce the
   matrix $A-9I$ \mk:
   \[
   \bbm
   -8&1&1&4\\
   1&-8&1&4\\
   1&1&-8&4\\
   4&4&4&-8\\
   \ebm
   \to
   \bbm
   0&-63&9&36\\
   1&-8&1&4\\
   0&9&-9&0\\
   0&36&0&-24\\
   \ebm
   \to
   \bbm
   0&-63&9&36\\
   1&-8&1&4\\
   0&1&-1&0\\
   0&36&0&-24\\
   \ebm
   \to
  \] \[
   \bbm
   0&0&-54&36\\
   1&0&-7&4\\
   0&1&-1&0\\
   0&0&36&-24\\
   \ebm
   \to
   \bbm
   0&0&-54&36\\
   1&0&-7&4\\
   0&1&-1&0\\
   0&0&1&-2/3\\
   \ebm
   \to
   \bbm
   0&0&0&0\\
   1&0&0&-2/3\\
   0&1&0&-2/3\\
   0&0&1&-2/3\\
   \ebm \mks{2}
   \]
   From this we see that the eigenvectors of eigenvalue $9$ are 
   multiples of the vector $a=\bbm 2&2&2&3\ebm^T$ \mk.  We note that
   $\|a\|^2=4+4+4+9=21$.  To
   get a unit vector we therefore take 
   \[ u_3 = a/\sqrt{21} = 
       \bbm \frac{2}{\sqrt{21}} &
            \frac{2}{\sqrt{21}} &
            \frac{2}{\sqrt{21}} &
            \frac{3}{\sqrt{21}} \ebm^T. \mk
   \]

   Finally, to find an eigenvector of eigenvalue $-5$ we
   row-reduce the matrix $A+5I$ \mk:
   \[
   \bbm
   6&1&1&4\\
   1&6&1&4\\
   1&1&6&4\\
   4&4&4&6\\
   \ebm
   \to
   \bbm
   0&-5&-35&-20\\
   0&5&-5&0\\
   1&1&6&4\\
   0&0&-20&-10\\
   \ebm
   \to
   \bbm
   0&-5&-35&-20\\
   0&1&-1&0\\
   1&1&6&4\\
   0&0&-20&-10\\
   \ebm
   \to
  \] \[
   \bbm
   0&0&1&1/2\\
   0&1&0&1/2\\
   1&0&7&4\\
   0&0&0&0\\
   \ebm
   \to
   \bbm
   0&0&1&1/2\\
   0&1&0&1/2\\
   1&0&0&1/2\\
   0&0&0&0\\
   \ebm \mks{2}
   \]

   From this we see that the eigenvectors of eigenvalue $-5$ are 
   multiples of the vector $b=\bbm 1&1&1&-2\ebm^T$ \mk.  We note that
   $\|b\|^2=1+1+1+4=7$.  To
   get a unit vector we therefore take 
   \[ u_3 = a/\sqrt{7} = 
       \bbm \frac{1}{\sqrt{7}} &
            \frac{1}{\sqrt{7}} &
            \frac{1}{\sqrt{7}} &
            \frac{-2}{\sqrt{7}} \ebm^T \mk.
   \]

   It is standard that when $s$ and $t$ are eigenvectors of a
   symmetric matrix with distinct eigenvalues, we have $s.t=0$.  This
   gives $u_1.u_3=u_1.u_4=u_2.u_3=u_2.u_4=u_3.u_4=0$, and the
   remaining identity $u_1.u_2=0$ is clear by inspection (as are the
   others, in fact).  Thus, we have an orthonormal basis of
   eigenvectors. 
  \item[(e)] The general theory tells us that we can take 
   \[ P = \left[\begin{array}{c|c|c|c}
           &&& \\ u_1 & u_2 & u_3 & u_4 \\ &&&
          \end{array}\right]
        = \bbm 
            1/\sqrt{2} & 1/\sqrt{6} &  2/\sqrt{21} &  1/\sqrt{7} \\
           -1/\sqrt{2} & 1/\sqrt{6} &  2/\sqrt{21} &  1/\sqrt{7} \\
           0 &          -2/\sqrt{6} &  2/\sqrt{21} &  1/\sqrt{7} \\
           0 &           0          &  3/\sqrt{21} & -2/\sqrt{7}
          \ebm \mks{2}
   \]
   \[ D = \bbm \lm_1 & 0 & 0 & 0 \\ 
               0 & \lm_2 & 0 & 0 \\
               0 & 0 & \lm_3 & 0 \\
               0 & 0 & 0 & \lm_4 \ebm 
        = \bbm 0 & 0 & 0 & 0 \\ 
               0 & 0 & 0 & 0 \\
               0 & 0 & 9 & 0 \\
               0 & 0 & 0 & -5 \ebm \mk
   \]
  \item[(f)] We have $Q=a^TAa$, where $a=\bbm w&x&y&z\ebm^T$.  It is
   standard that with an orthonormal sequence of eigenvectors as
   above, we have $Q=\sum_i\lm_i(u_i.a)^2$.  In our case
   $\lm_1=\lm_2=0$ so this reduces to 
   \begin{align*}
     Q &= 9(u_3.a)^2 - 5(u_4.a)^2 \mk =
          (3u_3.a)^2 - (\sqrt{5}u_4.a)^2 \\
       &= \left(\frac{6w+6x+6y+9z}{\sqrt{21}}\right)^2 - 
          \left(\sqrt{5}\frac{w+x+y-2z}{\sqrt{7}}\right)^2. \mk
   \end{align*} 
   We may thus take
   \begin{align*}
    F &= (6w+6x+6y+9z)/\sqrt{21} \\
    G &= (w+x+y-2z)\sqrt{5/7}. \mk
   \end{align*}
 \end{itemize}
\end{solution}


\begin{problem}[2011-12]
 Consider the matrix
 \[ A = \bbm 1&1&2&2 \\ 1&1&2&2 \\ 2&2&1&1 \\ 2&2&1&1 \ebm. \]
 You may assume that $\det(A-tI)=t^4-4t^3-12t^2$.
 \begin{itemize}
  \item[(a)] Find the eigenvalues of $A$. \mrks{2}
  \item[(b)] Find an orthonormal basis of $\R^4$ consisting of
   eigenvectors of $A$. \mrks{14}
  \item[(c)] Find an orthogonal matrix $P$ and a diagonal matrix $D$
   such that $A=PDP^T$. \mrks{4}
  \item[(d)] Express the quadratic form
   \[ Q = w^2+x^2+y^2+z^2 + 2(wx+yz) + 4(wy+wz+xy+xz) \] 
   as $Q=F^2-G^2$, where $F$ and $G$ are linear forms.  Hence express
   $Q$ as a product of two linear forms.  \mrks{5}
 \end{itemize}
\end{problem}
\begin{solution}
 \begin{itemize}
  \item[(a)] The characteristic polynomial factorises as
   $\chi_A(t)=t^2(t^2-4t-12)=t^2(t-6)(t+2)$, so the eigenvalues are
   $0$, $6$ and $-2$. \mks{2}
  \item[(b)] It is easy to see that a vector $\bbm w&x&y&z\ebm^T$ is
   an eigenvector of eigenvalue $0$ if and only if
   $w+x+2y+2z=2w+2x+y+z=0$ \mks{2}, which reduces to $x=-w$ and $z=-y$
   \mk.  If we  put 
   \[ u_1 = \bbm 1/\sqrt{2}&-1/\sqrt{2}&0&0\ebm^T \hspace{4em}
      u_2 = \bbm 0&0&1/\sqrt{2}&-1/\sqrt{2}\ebm^T
   \]
   then $u_1$ and $u_2$ are orthonormal and are eigenvectors of
   eigenvalue $0$ \mks{2}. 

   Next, to find an eigenvector of eigenvalue $6$ we row-reduce the
   matrix $A-6I$ \mk:
   \[ 
    \bbm -5&1&2&2 \\ 1&-5&2&2 \\ 2&2&-5&1 \\ 2&2&1&-5 \ebm
    \to
    \bbm 0&-24&12&12 \\ 1&-5&2&2 \\ 0&12&-9&-3 \\ 0&12&-3&-9 \ebm
    \to
    \bbm 0&-24&12&12 \\ 1&-5&2&2 \\ 0&12&-9&-3 \\ 0&0&6&-6 \ebm
   \] \[
    \to
    \bbm 0&-24&0&24 \\ 1&-5&0&4 \\ 0&12&0&-12 \\ 0&0&1&-1 \ebm
    \to
    \bbm 0&1&0&-1 \\ 1&-5&0&4 \\ 0&0&0&0 \\ 0&0&1&-1 \ebm
    \to
    \bbm 1&0&0&-1 \\ 0&1&0&-1 \\ 0&0&1&-1 \\ 0&0&0&0  \ebm \mks{2}.
   \]
   From this we see that the eigenvectors of eigenvalue $6$ are the
   vectors of the form $a=\bbm w&w&w&w\ebm^T$ \mk.  We note that
   $\|a\|=\sqrt{w^2+w^2+w^2+w^2}=\sqrt{4w^2}=2|w|$.  To get a unit
   vector we take $w=1/2$ giving 
   \[ u_3 = \bbm \half & \half & \half & \half \ebm^T. \mk \]

   Finally, to find an eigenvector of eigenvalue $-2$ we
   row-reduce the matrix $A+2I$:
   \[ 
    \bbm 3&1&2&2 \\ 1&3&2&2 \\ 2&2&3&1 \\ 2&2&1&3 \ebm
    \to
    \bbm 0&-8&-4&-4 \\ 1&3&2&2 \\ 0&-4&-1&-3 \\ 0&-4&-3&-1 \ebm
    \to
    \bbm 0&1&1/2&1/2 \\ 1&3&2&2 \\ 0&0&1&-1 \\ 0&0&-1&1 \ebm
   \] \[
    \to
    \bbm 0&1&0&1 \\ 1&3&0&4 \\ 0&0&1&-1 \\ 0&0&0&0 \ebm
    \to
    \bbm 1&0&0&1 \\ 0&1&0&1 \\ 0&0&1&-1 \\ 0&0&0&0 \ebm \mks{2}
   \]
   From this we see that the eigenvectors of eigenvalue $-2$ are the
   vectors $\bbm w&x&y&z\ebm^T$ satisfying $w+z=x+z=y-z=0$, or in
   other words the vectors of the form $\bbm w&w&-w&-w\ebm^T$ \mk.  To get
   a unit vector we again take $w=1/2$, giving 
   \[ u_4 = \bbm \half & \half & -\half & -\half\ebm^T. \mk \]
   It is standard that when $a$ and $b$ are eigenvectors of a
   symmetric matrix with distinct eigenvalues, we have $a.b=0$.  This
   gives $u_1.u_3=u_1.u_4=u_2.u_3=u_2.u_4=u_3.u_4=0$, and the
   remaining identity $u_1.u_2=0$ is clear by inspection (as are the
   others, in fact).  Thus, we have an orthonormal basis of
   eigenvectors. 
  \item[(c)] The general theory tells us that we can take 
   \[ P = \left[\begin{array}{c|c|c|c}
           &&& \\ u_1 & u_2 & u_3 & u_4 \\ &&&
          \end{array}\right]
        = \bbm 
            1/\sqrt{2} & 0 &  1/2 &  1/2 \\
           -1/\sqrt{2} & 0 &  1/2 &  1/2 \\
           0 &  1/\sqrt{2} &  1/2 & -1/2 \\
           0 & -1/\sqrt{2} &  1/2 & -1/2 \\
          \ebm \mks{2}
   \]
   \[ D = \bbm \lm_1 & 0 & 0 & 0 \\ 
               0 & \lm_2 & 0 & 0 \\
               0 & 0 & \lm_3 & 0 \\
               0 & 0 & 0 & \lm_4 \ebm 
        = \bbm 0 & 0 & 0 & 0 \\ 
               0 & 0 & 0 & 0 \\
               0 & 0 & 6 & 0 \\
               0 & 0 & 0 & -2 \ebm \mks{2}
   \]
  \item[(d)] We have $Q=a^TAa$, where $a=\bbm w&x&y&z\ebm^T$.  It is
   standard that with an orthonormal sequence of eigenvectors as
   above, we have $Q=\sum_i\lm_i(u_i.a)^2$.  In our case
   $\lm_1=\lm_2=0$ so this reduces to 
   \[ Q = 6(u_3.a)^2 - 2(u_4.a)^2 \mks{2} =
       (\sqrt{6}(w+x+y+z)/2)^2 - (\sqrt{2}(w+x-y-z)/2)^2. \mk
   \] 
   We may thus take
   \begin{align*}
    F &= \sqrt{6}(w+x+y+z)/2 \\
    G &= \sqrt{2}(w+x-y-z)/2. \mk
   \end{align*}
   This in turn gives $Q=LM$, where 
   \begin{align*}
    L &= F+G = \frac{\sqrt{6}+\sqrt{2}}{2}(w+x)+\frac{\sqrt{6}-\sqrt{2}}{2}(y+z) \\
    M &= F-G = \frac{\sqrt{6}-\sqrt{2}}{2}(w+x)+\frac{\sqrt{6}+\sqrt{2}}{2}(y+z). \mk
   \end{align*}
 \end{itemize}
\end{solution}

\begin{problem}[Mock 1]
 Put 
 \[ A = \bbm
         9&6&2&3\\
         6&0&0&2\\
         2&0&0&6\\
         3&2&6&9
        \ebm \qquad
    u_1 = \bbm 1\\-2\\2\\-1 \ebm \qquad
    u_2 = \bbm 1\\-2\\-2\\1 \ebm \qquad
    u_3 = \bbm 2\\1\\-1\\-2 \ebm \qquad
    u_4 = \bbm 2\\1\\1\\2 \ebm
 \]
 \begin{itemize}
  \item[(a)] Show that $\det(A)=1024=2^{10}$. \mrks{4}
  \item[(b)] Show that the vectors $u_i$ are eigenvectors for $A$, and
   determine the corresponding eigenvalues.  \mrks{6} \\
   \textbf{Note:} You do not need to find the characteristic
   polynomial or perform any row-reduction.
  \item[(c)] Find an orthogonal matrix $P$ and a diagonal matrix $D$
   such that $A=PDP^T$. \mrks{6}
  \item[(d)] Express the quadratic form
   \[ Q = 9(w^2+z^2)+12(wx+yz)+6wz+4(wy+xz) \]
   as $Q=F^2+G^2-H^2-J^2$, where $F,G,H$ and $J$ are linear forms.
   \mrks{6} 
  \item[(e)] What are the rank and signature of $Q$? \mrks{3}
 \end{itemize}
\end{problem}
\begin{solution}
 \begin{itemize}
  \item[(a)] We expand the determinant along the second row.  The $6$
   at the beginning of the row is in the $(2,1)$ slot and so has a
   sign $(-1)^{2+1}=-1$.  The $2$ at the end of the row is in the
   $(2,4)$ slot and so has a sign $(-1)^{2+4}=+1$.  This gives
   \[ \det(A) = -6\det\bbm 6&2&3 \\ 0&0&6 \\ 2&6&9 \ebm 
                +2\det\bbm 9&6&2 \\ 2&0&0 \\ 3&2&6 \ebm. \mk
   \]
   Each of the above $3\tm 3$ determinants can be expanded along the
   middle row:
   \begin{align*}
    \det\bbm 6&2&3 \\ 0&0&6 \\ 2&6&9 \ebm &=
       -6\det\bbm 6&2 \\ 2&6\ebm = -6\tm 32 = -192 \mk \\
    \det\bbm 9&6&2 \\ 2&0&0 \\ 3&2&6 \ebm &= 
       -2\det\bbm 6&2\\2&6\ebm = -2\tm 32 = -64. \mk
   \end{align*}
   Putting this together we get
   \[ \det(A) = -6\tm(-192) + 2\tm(-64) = 1024. \mk \]
  \item[(b)] We have
   \begin{align*}
    Au_1 &= 
     \bbm 9&6&2&3\\ 6&0&0&2\\ 2&0&0&6\\ 3&2&6&9 \ebm
     \bbm 1\\-2\\2\\-1 \ebm = 
     \bbm -2\\4\\-4\\2 \ebm = -2u_1 \\
    Au_2 &=
     \bbm 9&6&2&3\\ 6&0&0&2\\ 2&0&0&6\\ 3&2&6&9 \ebm
     \bbm 1\\-2\\-2\\1 \ebm =
     \bbm -4\\8\\8\\-4 \ebm = -4u_2 \\
    Au_3 &=
     \bbm 9&6&2&3\\ 6&0&0&2\\ 2&0&0&6\\ 3&2&6&9 \ebm
     \bbm 2\\1\\-1\\-2 \ebm =
     \bbm 16\\8\\-8\\-16\ebm = 8u_3 \\
    Au_4 &=
     \bbm 9&6&2&3\\ 6&0&0&2\\ 2&0&0&6\\ 3&2&6&9 \ebm
     \bbm 2\\1\\1\\2 \ebm =
     \bbm 32\\16\\16\\32 \ebm = 16u_4 \mks{4}
   \end{align*}
   From this we see that the vectors $u_i$ are eigenvctors for $A$,
   with eigenvalues $\lm_1=-2$, $\lm_2=-4$, $\lm_3=8$ and $\lm_4=16$
   \mks{2}. 
  \item[(c)] We need to find an orthonormal basis for $\R^4$
   consisting of eigenvectors for $A$ \mk.  The matrix $A$ is symmetric,
   and the vectors $u_i$ are eigenvectors for $A$ with distinct
   eigenvalues, so the general theory tells us that they are all
   orthogonal to each other.  (Alternatively, we can just check that
   directly $u_i.u_j=0$ for all $i\neq j$.) \mk  However, we do not have
   an orthonormal basis because $u_i$ is not a unit vector.  Instead,
   we have $u_1.u_1=1^2+(-2)^2+2^2+(-1)^2=10$ \mk, and similar
   calculations show that $u_2.u_2=u_3.u_3=u_4.u_4=10$ as well.  It
   follows that if we put $v_i=u_i/\sqrt{10}$ then the list
   $v_1,v_2,v_3,v_4$ is an orthonormal basis as required \mk.

   The general theory now tells us that we can take 
   \[ P = \left[\begin{array}{c|c|c|c}
           &&& \\ v_1 & v_2 & v_3 & v_4 \\ &&&
          \end{array}\right]
        = \bbm 
            1/\sqrt{10} &  1/\sqrt{10} &  2/\sqrt{10} &  2/\sqrt{10} \\
           -2/\sqrt{10} & -2/\sqrt{10} &  1/\sqrt{10} &  1/\sqrt{10} \\
            2/\sqrt{10} & -2/\sqrt{10} & -1/\sqrt{10} &  1/\sqrt{10} \\
           -1/\sqrt{10} &  1/\sqrt{10} & -2/\sqrt{10} &  2/\sqrt{10}
          \ebm \mk
   \]
   \[ D = \bbm \lm_1 & 0 & 0 & 0 \\ 
               0 & \lm_2 & 0 & 0 \\
               0 & 0 & \lm_3 & 0 \\
               0 & 0 & 0 & \lm_4 \ebm 
        = \bbm -2 &  0 & 0 & 0 \\ 
                0 & -4 & 0 & 0 \\
                0 &  0 & 8 & 0 \\
                0 &  0 & 0 & 16 \ebm \mk
   \]
  \item[(d)] We have $Q=a^TAa$, where $a=\bbm w&x&y&z\ebm^T$ \mk.  It is
   standard that with an orthonormal sequence of eigenvectors as
   above, we have $Q=\sum_i\lm_i(v_i.a)^2$ \mk.  In our case this gives
   \begin{align*}
    Q &= -2(a.v_1)^2-4(a.v_2)^2+8(a.v_3)^2+16(a.v_4)^2 \mk \\
      &= (a.\sqrt{8}v_3)^2 + (a.4v_4)^2 - (a.\sqrt{2}v_1)^2 - (a.2v_2)^2 \mk.
   \end{align*}
   In other words, we have $Q=F^2+G^2-H^2-J^2$, where 
   \begin{align*}
    F &= a.\sqrt{8}v_3 = 4w/\sqrt{5}  + 2x/\sqrt{5}  - 2y/\sqrt{5}  - 4z/\sqrt{5} \\
    G &= a.4v_4        = 8w/\sqrt{10} + 4x/\sqrt{10} + 4y/\sqrt{10} + 8z/\sqrt{10} \\
    H &= a.\sqrt{2}v_1 =  w/\sqrt{5}  - 2x/\sqrt{5}  + 2y/\sqrt{5}  -  z/\sqrt{5} \\
    J &= a.2v_2        = 2w/\sqrt{10} - 4x/\sqrt{10} - 4y/\sqrt{10} + 2z/\sqrt{10}
         \mks{2}
   \end{align*}
  \item[(e)] The rank is the number of nonzero eigenvalues, which is
   $4$ \mk.  The signature is the number of positive eigenvalues minus the
   number of negative eigenvalues, which is $0$ \mks{2}.
 \end{itemize}
\end{solution}

\begin{problem}[Mock 2]
 Put 
 \[ A = \bbm 2&3&7 \\ 3&2&3 \\ 7&3&2 \ebm. 
 \]
 You may assume the following row-reductions.  (Some of them are
 useful, and some of them are not.)
 \[
   \bbm
   2&3&7\\
   3&2&3\\
   7&3&2\\
   \ebm \to 
   \bbm
   1&0&-1\\
   0&1&3\\
   0&0&0\\
   \ebm \qquad
   \bbm
   7&3&7\\
   3&7&3\\
   7&3&7\\
   \ebm \to
   \bbm
   1&0&1\\
   0&1&0\\
   0&0&0\\
   \ebm \qquad
   \bbm
   9&3&7\\
   3&9&3\\
   7&3&9\\
   \ebm \to
   \bbm
   1&0&0\\
   0&1&0\\
   0&0&1\\
   \ebm
  \] \[
   \hspace{-3em}
   \bbm
   -2&3&7\\
   3&-2&3\\
   7&3&-2\\
   \ebm \to
   \bbm
   1&0&0\\
   0&1&0\\
   0&0&1\\
   \ebm \qquad
   \bbm
   -7&3&7\\
   3&-7&3\\
   7&3&-7\\
   \ebm \to
   \bbm
   1&0&0\\
   0&1&0\\
   0&0&1\\
   \ebm \qquad
   \bbm
   -9&3&7\\
   3&-9&3\\
   7&3&-9\\
   \ebm \to
   \bbm
   1&0&-1\\
   0&1&-2/3\\
   0&0&0\\
   \ebm
  \]
 \begin{itemize}
  \item[(a)] Find the eigenvalues of $A$. \mrks{6}
  \item[(b)] Find an orthogonal matrix $P$ and a diagonal matrix $D$
   such that $A=PDP^T$. \mrks{10}
  \item[(c)] Express the quadratic form
   \[ Q = 2x^2+2y^2+2z^2+6xy+6yz+14xz \]
   as $Q=L^2-M^2$, where $L$ and $M$ are linear forms.  Hence find
   linear forms $F$ and $G$ such that $Q=FG$.
   \mrks{7} 
  \item[(d)] What are the rank and signature of $Q$? \mrks{2}
 \end{itemize}
\end{problem}
\begin{solution}
 \begin{itemize}
  \item[(a)] The first of the given row-reductions shows that $A$
   reduces a matrix with a row of zeros.  It follows that
   $\det(A-0I)=\det(A)=0$, and thus that $0$ is an eigenvalue of $A$.
   Similarly, the second and sixth row-reductions shows that $A+5I$
   and $A-11I$ reduce to matrices with a row of zeros, so
   $\det(A+5I)=\det(A-11I)=0$, so $-5$ and $11$ are the other two
   eigenvalues.  

   For a more standard but less efficient approach, we can calculate
   the characteristic polynomial:
   \begin{align*}
    \chi_A(t) &= 
     \det\bbm 2-t&3&7\\ 3&2-t&3\\ 7&3&2-t \ebm \mk \\
     &= (2-t)\det\bbm 2-t&3 \\ 3&2-t\ebm 
       -3 \det\bbm 3&3\\ 7&2-t \ebm 
       +7 \det\bbm 3&2-t\\7&3 \ebm \mk \\
    \det\bbm 2-t&3 \\ 3&2-t\ebm &= 
     (2-t)^2-9 = t^2-4t-5 \\
    \det\bbm 3&3\\ 7&2-t \ebm &= 
     6-3t-21 = -3t-15 \\
    \det\bbm 3&2-t\\7&3 \ebm &=
     9 - (14-7t) = 7t - 5 \mks{2} \\
    \chi_A(t) &= (2-t)(t^2-4t-5) -3(-3t-15) + 7(7t-5) \\
              &= 2t^2-8t-10-t^3+4t^2+5t+9t+45+49t-35 \\
     &= -t^3 +6t^2 +55t \mk = -(t^2-6t-55)t = -(t-11)(t+5)t. 
   \end{align*}
   We again see that the eigenvalues are $\lm_1=11$ and $\lm_2=-5$ and
   $\lm_3=0$ \mk. 
  \item[(b)] We first need to find a system of eigenvectors for $A$.
   Using the fact that
   \[ A-11I = 
       \bbm
       -9&3&7\\
       3&-9&3\\
       7&3&-9\\
       \ebm \to
       \bbm
       1&0&-1\\
       0&1&-2/3\\
       0&0&0\\
       \ebm
   \]
   we see that an eigenvector $u_1=\bbm x&y&z\ebm^T$ of eigenvalue
   $11$ must satisfy $x-z=y-(2/3)z=0$.  We take $z=3$ giving
   $u_1=\bbm 3&2&3\ebm^T$ \mks{2}.  Similarly, using the fact that 
   \[ A+5I = 
       \bbm
       7&3&7\\
       3&7&3\\
       7&3&7\\
       \ebm \to
       \bbm
       1&0&1\\
       0&1&0\\
       0&0&0\\
       \ebm
   \]
   we see that the vector $u_2=\bbm 1&0&-1\ebm^T$ is an eigenvector of
   eigenvalue $-5$ \mks{2}.  Finally, using the row-reduction 
   \[ A = 
       \bbm
       2&3&7\\
       3&2&3\\
       7&3&2\\
       \ebm \to 
       \bbm
       1&0&-1\\
       0&1&3\\
       0&0&0\\
       \ebm
   \]
   we see that the vector $u_3=\bbm 1&-3&1\ebm$ is an eigenvector of
   eigenvalue $0$ \mks{2}.  As $A$ is symmetric and these three eigenvectors
   have different eigenvalues, they are automatically orthogonal.
   Next, we have 
   \begin{align*}
    u_1.u_1 &= 3^2+2^2+3^2 = 22 \\
    u_2.u_2 &= 1^2+(-1)^2 = 2 \\
    u_3.u_3 &= 1^2+(-3)^2+1^2 = 11.
   \end{align*}
   It follows that the vectors 
   \[ v_1 = \frac{u_1}{\sqrt{22}} =
             \bbm 3/\sqrt{22}\\ 2/\sqrt{22} \\ 3/\sqrt{22}\ebm 
      \qquad
      v_2 = \frac{u_2}{\sqrt{2}} = 
             \bbm 0 \\ 1/\sqrt{2} \\ -1/\sqrt{2} \ebm \\
      \qquad
      v_3 = \frac{u_3}{\sqrt{11}} = 
             \bbm 1/\sqrt{11} \\ -3/\sqrt{11} \\ 1/\sqrt{11}\ebm
   \] 
   form an orthonormal basis for $\R^3$ consisting of eigenvectors of
   $A$ \mks{2}.  It follows in turn that $A=PDP^T$, where 
   \begin{align*}
    P &= \left[\begin{array}{c|c|c}
          && \\ v_1 & v_2 & v_3 \\ && 
         \end{array}\right]
       = \bbm 3/\sqrt{22} &  0          &  1/\sqrt{11} \\
              2/\sqrt{22} &  1/\sqrt{2} & -3/\sqrt{11} \\
              3/\sqrt{22} & -1/\sqrt{2} &  1/\sqrt{11} \ebm \\
    D &= \diag(\lm_1,\lm_2,\lm_3) = 
          \bbm 11 & 0 & 0 \\ 0 & -5 & 0 \\ 0 & 0 & 0 \ebm \mks{2}. 
   \end{align*}
  \item[(c)] We note that $Q=a^TAa$, where $a=\bbm x&y&z\ebm^T$ and
   $A$ is as above.  The general theory therefore tells us that 
   \[ Q = \sum_{i=1}^3 \lm_i(a.v_i)^2 = 11(a.v_1)^2 - 5 (a.v_2)^2 \mks{2}. \]
   Thus, if we put 
   \begin{align*}
    L &= \sqrt{11}\; a.v_1 
       = (3x+2y+3z)/\sqrt{2} \\
    M &= \sqrt{5}\; a.v_2 = \sqrt{5/2}(y-z)
   \end{align*}
   then $Q=L^2-M^2$ \mks{2}.  This can be factored as $Q=FG$, where 
   \begin{align*}
    F &= L+M = (3x+(2+\sqrt{5})y+(3-\sqrt{5}z)/\sqrt{2} \\
    G &= L-M = (3x+(2-\sqrt{5})y+(3+\sqrt{5}z)/\sqrt{2} \mks{3}.
   \end{align*}
  \item[(d)] The rank is the number of nonzero eigenvalues, which is
   $2$ \mk.  The signature is the number of positive eigenvalues minus the
   number of negative eigenvalues, which is $0$ \mk.
 \end{itemize}
\end{solution}

\end{document}
